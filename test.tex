
1. **Teste de Recursão**: Verifica a geração correta da sequência de Fibonacci até o 9º número. Por exemplo, verifica se o número de Fibonacci no índice 7 é calculado corretamente.

\begin{verbatim}
fib = fn(n)  
    float_close(n, 0) ? 
        0
    :float_close(n, 1) ?
        1
    :fib(n-1)  + fib(n-2);

fib(7)
\end{verbatim}

2. **Teste de Atribuição e Interpretação**: Avalia uma expressão envolvendo atribuições de variáveis e chamadas de função no mesmo escopo. Por exemplo, verifica se a função `val` calcula corretamente a subtração de seus argumentos.

\begin{verbatim}
a = 1;
b = 2;
c = 3;
d = 4;

val = fn(a,b,c,d) d-c-b-a; 
val(a,b,c,d, 2, 4 ,4)
\end{verbatim}

3. **Teste de Operador Relacional**: Verifica a precedência e associatividade dos operadores relacionais (`lt`, `gt`, `eq`) e lógicos (`or`, `and`). Por exemplo, verifica se a expressão `1 or 2 and 3 or 4` é corretamente parantezada.

\begin{verbatim}
1 or 2 and 3 or 4
\end{verbatim}

4. **Teste de Precedência de Operadores Unários e Binários**: Examina a precedência dos operadores unários e binários. Por exemplo, verifica se a expressão `!a + b` é corretamente interpretada com o operador unário `!` aplicado antes do binário `+`.

\begin{verbatim}
!a + b
\end{verbatim}

5. **Teste de Agrupamento**: Garante que as expressões entre parênteses sejam corretamente agrupadas. Por exemplo, verifica se a expressão `a + (b + c) + d` é parantezada conforme o esperado.

\begin{verbatim}
a + (b + c) + d
\end{verbatim}

Esses testes, junto com outros na suíte, ajudam a validar a correção da implementação do parser de Pratt, cobrindo várias características da linguagem e regras de precedência de operadores.
