%%%%%%%%%%%%%%%%%%%%%%%%%%%%%%%%%%%%%%%%%%%%%%%%%%%%%%%%%%%%%%%%%%%%%%%%%%

% abnTeX2: Modelo de Trabalho Acadêmico em conformidade com 
% as normas da ABNT

%%%%%%%%%%%%%%%%%%%%%%%%%%%%%%%%%%%%%%%%%%%%%%%%%%%%%%%%%%%%%%%%%%%%%%%%%%

\documentclass[english, 
               brazil, 
               bsc] %Opções bsc (TCC) e msc (Mestrado)
               {dcomp-abntex2}


%%%%%%%%%%%%%%%%%%%%%%%%%%%%%%%%%%%%%%%%%%%%%%%%%%%%%%%%%%%%%%%%%%%%%%%%%%
% Área para adição de pacotes extras
%%%%%%%%%%%%%%%%%%%%%%%%%%%%%%%%%%%%%%%%%%%%%%%%%%%%%%%%%%%%%%%%%%%%%%%%%%

\usepackage{lipsum} % Retirar para a versão final do documento

%Utilize aqui seu pacote preferido para algoritmos
\usepackage[linesnumbered]{algorithm2e}

%%%%%%%%%%%%%%%%%%%%%%%%%%%%%%%%%%%%%%%%%%%%%%%%%%%%%%%%%%%%%%%%%%%%%%%%%%

%Compila o indice
\makeindex

\begin{document}

% Seleciona o idioma do documento (conforme pacotes do babel)
\selectlanguage{brazil}

% Retira espaço extra obsoleto entre as frases.
\frenchspacing 

%%%%%%%%%%%%%%%%%%%%%%%%%%%%%%%%%%%%%%%%%%%%%%%%%%%%%%%%%%%%%%%%%%%%%%%%%%
% ELEMENTOS PRÉ-TEXTUAIS
%%%%%%%%%%%%%%%%%%%%%%%%%%%%%%%%%%%%%%%%%%%%%%%%%%%%%%%%%%%%%%%%%%%%%%%%%%

\pretextual


\titulo{Desenvolvimento de um Compilador de BRDFs em LaTeX para linguagem de shading GLSL, através da técnica Pratt Parsing } 
\autor{Everton Santos de Andrade Júnior}
\orientador{Beatriz Trinchão Andrade}
\coorientador{Gastao Florencio Miranda Junior}
\curso{Ciência da Computação}

\inserirInformacoesPDF

\imprimircapa
\imprimirfolhaderosto*

% \begin{dedicatoria}
   \vspace*{\fill}
   \centering
   \noindent
   \textit{Esta página foi deixada em branco, não-ironicamente, de propósito} \vspace*{\fill}
\end{dedicatoria}
% ---

% \include{Pre_Textual/Agradecimentos}
% \begin{epigrafe}[]
    \vspace*{\fill}
	\begin{flushright}
	
		\textit{Este trabalho, além de cultural, filosófico e pedagógico\\
				É também medicinal, preventivo e curativo\\
				Servindo entre outras coisas para pano branco e pano preto\\
				Curuba e ferida braba\\
				Piolho, chulé e caspa\\
				Cravo, espinha e berruga\\
				Panarismo e água na pleura\\
				Só não cura o velho chifre\\
				Por que não mata a raiz\\
				Pois fica ela encravada\\
				No fundo do coração\\
				(Falcão)}
		
	\end{flushright}
\end{epigrafe}
% ---

% % resumo em português
\setlength{\absparsep}{18pt} % ajusta o espaçamento dos parágrafos do resumo
\begin{resumo}
 
% Segundo a \citeonline[3.1-3.2]{NBR6028:2003}, o resumo deve ressaltar o objetivo, o método, os resultados e as conclusões do documento. A ordem e a extensão destes itens dependem do tipo de resumo (informativo ou indicativo) e do tratamento que cada item recebe no documento original. O resumo deve ser precedido da referência do documento, com exceção do resumo inserido no próprio documento. (\ldots) As palavras-chave devem figurar logo abaixo do resumo, antecedidas da expressão Palavras-chave:, separadas entre si por ponto e finalizadas também por ponto.


  O presente trabalho propõe o desenvolvimento de um compilador de funções de distribuição de reflexão bidirecional (BRDFs) expressas em \LaTeX{}  para a linguagem de \textit{shading} GLSL, utilizando a técnica de Pratt \textit{Parsing} e linguagem de programação Odin.
  O objetivo é automatizar o processo de tradução de funções complexas de materiais, descritas em equações \LaTeX{}, para o código GLSL utilizado na programação de \textit{shaders} para OpenGL.
  Ao fornecer essa ferramenta, pretende-se não apenas simplificar o trabalho dos desenvolvedores e pesquisadores na área de computação gráfica, mas também democratizar o acesso e compreensão de modelos de materiais complexos. Além disso, ao permitir que as BRDFs sejam expressas em uma forma mais familiar e acessível, como a notação matemática, o compilador reduz a barreira de entrada para aqueles que não estão familiarizados com linguagens programação, de modo a facilitar a colaboração interdisciplinar entre profissionais de diferentes áreas. A validação dos \textit{shaders} de saída do compilador proposto será feita através da ferramenta Disney BRDF Explorer, que possibilita a visualização e análise de BRDFs.

  \textbf{Palavras-chave}: Compilador, BRDFs, LaTeX, GLSL, Shading, Pratt \textit{Parsing}.
\end{resumo}

% \include{Pre_Textual/Abstract}


\mostrarlistadeILUSTRACOES
\mostrarlistadeQUADROS
\mostrarlistadeTABELAS
\mostrarlistadeCODIGOS
\mostrarlistadeALGORITMOS
 
\include{Pre_Textual/Abreviaturas}
% ---
% inserir lista de símbolos
% ---

\begin{simbolos}
  \item[$ \Gamma $] Letra grega Gama
  \item[$ \Lambda $] Lambda
  \item[$ \zeta $] Letra grega minúscula zeta
  \item[$ \in $] Pertence
\end{simbolos}
% ---

    
\mostrarSUMARIO

%%%%%%%%%%%%%%%%%%%%%%%%%%%%%%%%%%%%%%%%%%%%%%%%%%%%%%%%%%%%%%%%%%%%%%%%%%
% ELEMENTOS TEXTUAIS
%%%%%%%%%%%%%%%%%%%%%%%%%%%%%%%%%%%%%%%%%%%%%%%%%%%%%%%%%%%%%%%%%%%%%%%%%%

\textual

%%%%%%%%%%%%%%%%%%%%%%%%%%%%%%%%%%%%%%%%%%%%%%%%%%%%%%%%%%%%%%%%%%%%%%%%%%
% Introdução
%%%%%%%%%%%%%%%%%%%%%%%%%%%%%%%%%%%%%%%%%%%%%%%%%%%%%%%%%%%%%%%%%%%%%%%%%%
\chapter{Introdução}


\section{Contexto}

Na computação gráfica, a representação realista de cenas tridimensionais depende fortemente da modelagem da luz. A interação da luminosidade incidente no objeto, bem como os materiais que compõem esses objetos, são aspectos críticos a serem considerados na geração dessas cenas [referencia]. Na prática, essa interação é frequentemente modelada por meio de funções de distribuição de refletância bidirecional, conhecidas como BRDFs.


As BRDFs, essencialmente, calculam a proporção entre a energia luminosa que atinge um ponto na superfície e como essa energia é refletida, transmitida ou absorvida [referencia]. Na renderização, essas funções são implementadas por meio programas especializados das unidades de processamento gráfico (GPUs), esses programmas são chamados de shaders, e cada API de rederização disponibiliza etapas diferentes onde esses executaveis podem ser mudados durante o processo de renderização. Esses shaders concedem a capacidade de cada objeto renderizado ter sua aparência configurada por meio de um código que implementa uma BRDF.


\section{Motivação}

Apesar da disponibilidade de linguagens específicas para a programação de shaders, que possibilitam a modificação procedimentos que representam uma BRDF, a aplicação de BRDFs na geração de shaders requer conhecimento especializado em programação [referencia?]. Essa barreira técnica pode restringir a exploração dos efeitos visuais por profissionais de áreas não relacionadas à programação. Diante disso, surge a necessidade de ferramentas mais acessíveis para a criação de shaders.

No meio acadêmico, as BRDFs são, comumente, descritas por uma fórmula escrita em LaTeX, uma abordagem promissora para atender a essa necessidade é o desenvolvimento de um compilador capaz de traduzir BRDFs em LaTex para shaders, assim democratizando a visualização dessas BRDFs. Dado que as fórmulas são equações matemáticas, precisamos retrigir repretsentação da linguagem de entrada para o compilador afim de garatir um projeto útil em tempo ábil.

\section{Objetivo}
Este trabalho visa projetar e implementar um compilador que, a partir de funções de distribuição de refletância bidirecional escrita como equações em LaTeX, seja capaz de gerar código de shading na linguagem alvo da API OpenGL (referencia). A saída será um shader capaz de reproduzir as características de reflexão da função de refletância original, considerando a precedencia de operadores, em uma superfície tridimensional, ou, ao menos, alcançar uma aproximação satisfatória dessas características, considerando as limitações da linguagem de shading da API principalmente as representações de dados de forma discreta.

\section{Metodologia}
Para alcançar o objetivo, a sequencia das etapas adotadas serão as seguintes.


\begin{enumerate}

   \item Realizar uma análise abrangente das áreas relacionadas ao desenvolvimento da ferramenta proposta;
   \item Investigar o estado da arte no campo da compilação de BRDFs em linguagens de shading;
   \item Definir a linguagem de entrada e a linguagem de saída do compilador;
   \item Elaborar testes com equações LaTeX de entrada pareado com a saída em shader GLSL esperado;
   \item Implementar o compilador utilizando uma linguagem de programação e tecnicas recursivas de parsing
   \item Realizar a renderização de cenas utilizando o shader gerado pelo compilador.

\end{enumerate}

% Apesar da importância de usar técnicas confiáveis para avaliar um BRDF, há uma falta de trabalhos na literatura que reúnam e comparem essas técnicas.
% Este artigo propõe uma compilação de técnicas usadas para avaliar representações de BRDF, juntamente com suas definições formais. Essas técnicas foram classificadas em três grupos diferentes - funções de comparação, imagens renderizadas e gráficos - e, para ilustrar seu uso, três modelos clássicos e amplamente adotados e uma representação de BRDF de ponta foram avaliados quanto à sua capacidade de preservar a aparência de materiais medidos. Com base em nossa pesquisa sobre funções de comparação, uma técnica de avaliação de BRDF estável e robusta é proposta. Observou-se tanto durante a revisão da literatura quanto nos experimentos que cada grupo de técnicas fornece informações complementares sobre os BRDFs avaliados, o que sugere que pelo menos um modelo de cada categoria deve ser adotado durante a escolha de critérios para avaliar um BRDF.

%%%%%%%%%%%%%%%%%%%%%%%%%%%%%%%%%%%%%%%%%%%%%%%%%%%%%%%%%%%%%%%%%%%%%%%%%%
% Revisão Bibliográfica 
%%%%%%%%%%%%%%%%%%%%%%%%%%%%%%%%%%%%%%%%%%%%%%%%%%%%%%%%%%%%%%%%%%%%%%%%%%
\chapter{Revisão Bibliográfica}

Para esta seção, será conduzida uma revisão literária abrangente com o objetivo de explorar trabalhos relacionados ao desenvolvimento de compiladores para tradução de BRDFs expressas em LaTeX para a linguagem de shading, empregando, técnicas de parsing.
O processo de busca será conduzido em duas etapas distintas. Primeiramente, será realizado um levantamento dos trabalhos existentes nas bases IEEE Xplorer Digital Library, Elsevier Scopus e Google Acadêmico, esse foram escolhidos por serem acessíveis gratuitamente pela afiliação à Universidade Federal de Sergipe, já o google scholar foi escolhido para agregar pesquisas em outras bases que possam ter trabalhos relevantes.

Por fim, será realizada uma busca por produtos ou ferramentas similares no mercado, utilizando strings de busca específicas em repositórios digitais, especificamente GitHub, e GitLab. Esses processos de busca permitirão identificar referências relevantes e estabelecer um panorama do estado da arte no campo dos compiladores de BRDFs para shaders, contribuindo para a compreensão do contexto acadêmico e prático no qual este trabalho se insere.

\subsection{Mapeamento Sistemático}

Foram elaboradas questões de pesquisa específicas e conduzidas buscas em diversas bases de dados, utilizando frases-chave que refletiam os principais aspectos do tema em questão. A partir desse processo de busca criteriosa, foram identificados e selecionados os trabalhos que melhor atendiam às questões propostas, garantindo maior relevância para o estudo em questão.


\subsection{Questões de Pequisa}

\begin{enumerate}
  \item Quais são as abordagens mais comuns utilizadas na criação de compiladores para tradução de BRDFs expressas em alguma linguagem de texto, com LaTeX, para shaders?

  \item Quais as técnicas de parsing que têm sido aplicadas no desenvolvimento de compiladores para linguagens matemáticas como LaTeX?

  \item O trabalho utiliza arvores, ou gramáticas livre de contexto para representar uma BRDF?

 \item Quais são os principais desafios enfrentados ao traduzir funções matemáticas complexas, como as BRDFs, em shaders?

\end{enumerate}

Quais são as ferramentas e recursos disponíveis para auxiliar no desenvolvimento de compiladores para BRDFs e shaders, e como elas podem ser integradas ao processo de desenvolvimento?

%
(("Full Text \& Metadata":brdf) OR  ("Full Text \& Metadata":reflectance))
AND (("Full Text \& Metadata":shader) OR  ("Full Text \& Metadata":shading))
AND (("Full Text \& Metadata":compiler) OR  ("Full Text \& Metadata":parsing) OR  ("Full Text \& Metadata":parser) OR  ("Full Text \& Metadata":grammar))

%
% ("Full Text & Metadata":brdf)
% AND (("Full Text & Metadata":shader) OR  ("Full Text & Metadata":shading))
% AND (("Full Text & Metadata":compiler) OR  ("Full Text & Metadata":parsing) OR  ("Full Text & Metadata":parser) OR  ("Full Text & Metadata":grammar))
%


\subsection{Descrição dos Trabalhos Relacionados}

\subsection{Pesquisa por Repositórios online}


\chapter{Conceitos}

\section{BRDFs}
% \url{https://www.youtube.com/watch?v=kPIqO929pIc&list=PL2zRqk16wsdpyQNZ6WFlGQtDICpzzQ925&index=3}

\section{Compilador}
\section{Análise Léxica}
\section{Análise Sintática}
\subsection{Pratt Parsing}
\subsubsection{precedencia de expressões}
\section{Análise Sintática}
\section{Pratt Parsing}
\section{Shaders}


% \chapter{Resultados de comandos}\label{cap_exemplos}

\chapterprecis{Isto é uma sinopse de capítulo. A ABNT não traz nenhuma
normatização a respeito desse tipo de resumo, que é mais comum em romances 
e livros técnicos.}\index{sinopse de capítulo}

% ---
\section{Codificação dos arquivos: UTF8}
% ---

A codificação de todos os arquivos do \abnTeX\ é \texttt{UTF8}. É necessário que
você utilize a mesma codificação nos documentos que escrever, inclusive nos
arquivos de base bibliográficas |.bib|.

% ---
\section{Citações diretas}
\label{sec-citacao}
% ---

\index{citações!diretas}Utilize o ambiente \texttt{citacao} para incluir citações diretas com mais de três linhas:

\begin{citacao}
As citações diretas, no texto, com mais de três linhas, devem ser
destacadas com recuo de 4 cm da margem esquerda, com letra menor que a do texto utilizado e sem as aspas. No caso de documentos datilografados, deve-se observar apenas o recuo \cite[5.3]{NBR10520:2002}.
\end{citacao}

Use o ambiente assim:

\begin{verbatim}
\begin{citacao}
As citações diretas, no texto, com mais de três linhas [...] 
deve-se observar apenas o recuo \cite[5.3]{NBR10520:2002}.
\end{citacao}
\end{verbatim}

O ambiente \texttt{citacao} pode receber como parâmetro opcional um nome de
idioma previamente carregado nas opções da classe (\autoref{sec-hifenizacao}). Nesse
caso, o texto da citação é automaticamente escrito em itálico e a hifenização é
ajustada para o idioma selecionado na opção do ambiente. Por exemplo:

\begin{verbatim}
\begin{citacao}[english]
Text in English language in italic with correct hyphenation.
\end{citacao}
\end{verbatim}

Tem como resultado:

\begin{citacao}[english]
Text in English language in italic with correct hyphenation.
\end{citacao}

\index{citações!simples}Citações simples, com até três linhas, devem ser
incluídas com aspas. Observe que em \LaTeX as aspas iniciais são diferentes das
finais: ``Amor é fogo que arde sem se ver''.

% ---
\section{Notas de rodapé}
% ---

As notas de rodapé são detalhadas pela NBR 14724:2011 na seção 5.2.1\footnote{As
notas devem ser digitadas ou datilografadas dentro das margens, ficando
separadas do texto por um espaço simples de entre as linhas e por filete de 5
cm, a partir da margem esquerda. Devem ser alinhadas, a partir da segunda linha
da mesma nota, abaixo da primeira letra da primeira palavra, de forma a destacar
o expoente, sem espaço entre elas e com fonte menor
\citeonline[5.2.1]{NBR14724:2011}.}\footnote{Caso uma série de notas sejam
criadas sequencialmente, o \abnTeX\ instrui o \LaTeX\ para que uma vírgula seja
colocada após cada número do expoente que indica a nota de rodapé no corpo do
texto.}\footnote{Verifique se os números do expoente possuem uma vírgula para
dividi-los no corpo do texto.}. 


% ---
\section{Tabelas}
% ---

\index{tabelas}A \autoref{tab-nivinv} é um exemplo de tabela construída em
\LaTeX.

\begin{table}[htb]
\ABNTEXfontereduzida
\caption[Níveis de investigação]{Níveis de investigação.}
\label{tab-nivinv}
\begin{tabular}{p{2.6cm}|p{6.0cm}|p{2.25cm}|p{3.40cm}}
  %\hline
   \textbf{Nível de Investigação} & \textbf{Insumos}  & \textbf{Sistemas de Investigação}  & \textbf{Produtos}  \\
    \hline
    Meta-nível & Filosofia\index{filosofia} da Ciência  & Epistemologia &
    Paradigma  \\
    \hline
    Nível do objeto & Paradigmas do metanível e evidências do nível inferior &
    Ciência  & Teorias e modelos \\
    \hline
    Nível inferior & Modelos e métodos do nível do objeto e problemas do nível inferior & Prática & Solução de problemas  \\
   % \hline
\end{tabular}
\legend{Fonte: \citeonline{van86}}
\end{table}

Já a \autoref{tabela-ibge} apresenta uma tabela criada conforme o padrão do
\citeonline{ibge1993} requerido pelas normas da ABNT para documentos técnicos e
acadêmicos.

\begin{table}[htb]
\IBGEtab{%
  \caption{Um Exemplo de tabela alinhada que pode ser longa
  ou curta, conforme padrão IBGE.}%
  \label{tabela-ibge}
}{%
  \begin{tabular}{ccc}
  \toprule
   Nome & Nascimento & Documento \\
  \midrule \midrule
   Maria da Silva & 11/11/1111 & 111.111.111-11 \\
  \midrule 
   João Souza & 11/11/2111 & 211.111.111-11 \\
  \midrule 
   Laura Vicuña & 05/04/1891 & 3111.111.111-11 \\
  \bottomrule
\end{tabular}%
}{%
  \fonte{Produzido pelos autores.}%
  \nota{Esta é uma nota, que diz que os dados são baseados na
  regressão linear.}%
  \nota[Anotações]{Uma anotação adicional, que pode ser seguida de várias
  outras.}%
  }
\end{table}


% ---
\section{Figuras}
% ---

\index{figuras}Figuras podem ser criadas diretamente em \LaTeX,
como o exemplo da \autoref{fig_circulo}.

\begin{figure}[htb]
	\caption{\label{fig_circulo}A delimitação do espaço}
	\begin{center}
	    \setlength{\unitlength}{5cm}
		\begin{picture}(1,1)
		\put(0,0){\line(0,1){1}}
		\put(0,0){\line(1,0){1}}
		\put(0,0){\line(1,1){1}}
		\put(0,0){\line(1,2){.5}}
		\put(0,0){\line(1,3){.3333}}
		\put(0,0){\line(1,4){.25}}
		\put(0,0){\line(1,5){.2}}
		\put(0,0){\line(1,6){.1667}}
		\put(0,0){\line(2,1){1}}
		\put(0,0){\line(2,3){.6667}}
		\put(0,0){\line(2,5){.4}}
		\put(0,0){\line(3,1){1}}
		\put(0,0){\line(3,2){1}}
		\put(0,0){\line(3,4){.75}}
		\put(0,0){\line(3,5){.6}}
		\put(0,0){\line(4,1){1}}
		\put(0,0){\line(4,3){1}}
		\put(0,0){\line(4,5){.8}}
		\put(0,0){\line(5,1){1}}
		\put(0,0){\line(5,2){1}}
		\put(0,0){\line(5,3){1}}
		\put(0,0){\line(5,4){1}}
		\put(0,0){\line(5,6){.8333}}
		\put(0,0){\line(6,1){1}}
		\put(0,0){\line(6,5){1}}
		\end{picture}
	\end{center}
	\legend{Fonte: os autores}
\end{figure}

Ou então figuras podem ser incorporadas de arquivos externos, como é o caso da
\autoref{fig_grafico}. Se a figura que for incluída se tratar de um diagrama, um
gráfico ou uma ilustração que você mesmo produza, priorize o uso de imagens
vetoriais no formato PDF. Com isso, o tamanho do arquivo final do trabalho será
menor, e as imagens terão uma apresentação melhor, principalmente quando
impressas, uma vez que imagens vetorias são perfeitamente escaláveis para
qualquer dimensão. Nesse caso, se for utilizar o Microsoft Excel para produzir
gráficos, ou o Microsoft Word para produzir ilustrações, exporte-os como PDF e
os incorpore ao documento conforme o exemplo abaixo. No entanto, para manter a
coerência no uso de software livre (já que você está usando \LaTeX e \abnTeX),
teste a ferramenta \textsf{InkScape}\index{InkScape}
(\url{http://inkscape.org/}). Ela é uma excelente opção de código-livre para
produzir ilustrações vetoriais, similar ao CorelDraw\index{CorelDraw} ou ao Adobe
Illustrator\index{Adobe Illustrator}. De todo modo, caso não seja possível
utilizar arquivos de imagens como PDF, utilize qualquer outro formato, como
JPEG, GIF, BMP, etc. Nesse caso, você pode tentar aprimorar as imagens
incorporadas com o software livre \textsf{Gimp}\index{Gimp}
(\url{http://www.gimp.org/}). Ele é uma alternativa livre ao Adobe
Photoshop\index{Adobe Photoshop}.

\begin{figure}[htb]
	\caption{\label{fig_grafico}Gráfico produzido em Excel e salvo como PDF}
	\begin{center}
	    \includegraphics[scale=0.5]{Imagens/abntex2-modelo-img-marca.pdf}
	\end{center}
	\legend{Fonte: \citeonline[p. 24]{araujo2012}}
\end{figure}

% ---
\subsection{Figuras em \emph{minipages}}
% ---

\emph{Minipages} são usadas para inserir textos ou outros elementos em quadros
com tamanhos e posições controladas. Veja o exemplo da
\autoref{fig_minipage_imagem1} e da \autoref{fig_minipage_grafico2}.

\begin{figure}[htb]
 \label{teste}
 \centering
  \begin{minipage}{0.4\textwidth}
    \centering
    \caption{Imagem 1 da minipage} \label{fig_minipage_imagem1}
    \includegraphics[scale=0.9]{Imagens/abntex2-modelo-img-marca.pdf}
    \legend{Fonte: Produzido pelos autores}
  \end{minipage}
  \hfill
  \begin{minipage}{0.4\textwidth}
    \centering
    \caption{Grafico 2 da minipage} \label{fig_minipage_grafico2}
    \includegraphics[scale=0.2]{Imagens/abntex2-modelo-img-grafico.pdf}
    \legend{Fonte: \citeonline[p. 24]{araujo2012}}
  \end{minipage}
\end{figure}

Observe que, segundo a \citeonline[seções 4.2.1.10 e 5.8]{NBR14724:2011}, as
ilustrações devem sempre ter numeração contínua e única em todo o documento:

\begin{citacao}
Qualquer que seja o tipo de ilustração, sua identificação aparece na parte
superior, precedida da palavra designativa (desenho, esquema, fluxograma,
fotografia, gráfico, mapa, organograma, planta, quadro, retrato, figura,
imagem, entre outros), seguida de seu número de ordem de ocorrência no texto,
em algarismos arábicos, travessão e do respectivo título. Após a ilustração, na
parte inferior, indicar a fonte consultada (elemento obrigatório, mesmo que
seja produção do próprio autor), legenda, notas e outras informações
necessárias à sua compreensão (se houver). A ilustração deve ser citada no
texto e inserida o mais próximo possível do trecho a que se
refere. \cite[seções 5.8]{NBR14724:2011}
\end{citacao}

% ---
\section{Expressões matemáticas}
% ---

\index{expressões matemáticas}Use o ambiente \texttt{equation} para escrever
expressões matemáticas numeradas:

\begin{equation}
  \forall x \in X, \quad \exists \: y \leq \epsilon
\end{equation}

Escreva expressões matemáticas entre \$ e \$, como em $ \lim_{x \to \infty}
\exp(-x) = 0 $, para que fiquem na mesma linha.

Também é possível usar colchetes para indicar o início de uma expressão
matemática que não é numerada.

\[
\left|\sum_{i=1}^n a_ib_i\right|
\le
\left(\sum_{i=1}^n a_i^2\right)^{1/2}
\left(\sum_{i=1}^n b_i^2\right)^{1/2}
\]

Consulte mais informações sobre expressões matemáticas em
\url{https://github.com/abntex/abntex2/wiki/Referencias}.

% ---
\section{Enumerações: alíneas e subalíneas}
% ---

\index{alíneas}\index{subalíneas}\index{incisos}Quando for necessário enumerar
os diversos assuntos de uma seção que não possua título, esta deve ser
subdividida em alíneas \cite[4.2]{NBR6024:2012}:

\begin{alineas}

  \item os diversos assuntos que não possuam título próprio, dentro de uma mesma
  seção, devem ser subdivididos em alíneas; 
  
  \item o texto que antecede as alíneas termina em dois pontos;
  \item as alíneas devem ser indicadas alfabeticamente, em letra minúscula,
  seguida de parêntese. Utilizam-se letras dobradas, quando esgotadas as
  letras do alfabeto;

  \item as letras indicativas das alíneas devem apresentar recuo em relação à
  margem esquerda;

  \item o texto da alínea deve começar por letra minúscula e terminar em
  ponto-e-vírgula, exceto a última alínea que termina em ponto final;

  \item o texto da alínea deve terminar em dois pontos, se houver subalínea;

  \item a segunda e as seguintes linhas do texto da alínea começa sob a
  primeira letra do texto da própria alínea;
  
  \item subalíneas \cite[4.3]{NBR6024:2012} devem ser conforme as alíneas a
  seguir:

  \begin{alineas}
     \item as subalíneas devem começar por travessão seguido de espaço;

     \item as subalíneas devem apresentar recuo em relação à alínea;

     \item o texto da subalínea deve começar por letra minúscula e terminar em
     ponto-e-vírgula. A última subalínea deve terminar em ponto final, se não
     houver alínea subsequente;

     \item a segunda e as seguintes linhas do texto da subalínea começam sob a
     primeira letra do texto da própria subalínea.
  \end{alineas}
  
  \item no \abnTeX\ estão disponíveis os ambientes \texttt{incisos} e
  \texttt{subalineas}, que em suma são o mesmo que se criar outro nível de
  \texttt{alineas}, como nos exemplos à seguir:
  
  \begin{incisos}
    \item \textit{Um novo inciso em itálico};
  \end{incisos}
  
  \item Alínea em \textbf{negrito}:
  
  \begin{subalineas}
    \item \textit{Uma subalínea em itálico};
    \item \underline{\textit{Uma subalínea em itálico e sublinhado}}; 
  \end{subalineas}
  
  \item Última alínea com \emph{ênfase}.
  
\end{alineas}

% ---
\section{Espaçamento entre parágrafos e linhas}
% ---

\index{espaçamento!dos parágrafos}O tamanho do parágrafo, espaço entre a margem
e o início da frase do parágrafo, é definido por:

\begin{verbatim}
   \setlength{\parindent}{1.3cm}
\end{verbatim}

\index{espaçamento!do primeiro parágrafo}Por padrão, não há espaçamento no
primeiro parágrafo de cada início de divisão do documento
(\autoref{sec-divisoes}). Porém, você pode definir que o primeiro parágrafo
também seja indentado, como é o caso deste documento. Para isso, apenas inclua o
pacote \textsf{indentfirst} no preâmbulo do documento:

\begin{verbatim}
   \usepackage{indentfirst}      % Indenta o primeiro parágrafo de cada seção.
\end{verbatim}

\index{espaçamento!entre os parágrafos}O espaçamento entre um parágrafo e outro
pode ser controlado por meio do comando:

\begin{verbatim}
  \setlength{\parskip}{0.2cm}  % tente também \onelineskip
\end{verbatim}

\index{espaçamento!entre as linhas}O controle do espaçamento entre linhas é
definido por:

\begin{verbatim}
  \OnehalfSpacing       % espaçamento um e meio (padrão); 
  \DoubleSpacing        % espaçamento duplo
  \SingleSpacing        % espaçamento simples	
\end{verbatim}

Para isso, também estão disponíveis os ambientes:

\begin{verbatim}
  \begin{SingleSpace} ...\end{SingleSpace}
  \begin{Spacing}{hfactori} ... \end{Spacing}
  \begin{OnehalfSpace} ... \end{OnehalfSpace}
  \begin{OnehalfSpace*} ... \end{OnehalfSpace*}
  \begin{DoubleSpace} ... \end{DoubleSpace}
  \begin{DoubleSpace*} ... \end{DoubleSpace*} 
\end{verbatim}

Para mais informações, consulte \citeonline[p. 47-52 e 135]{memoir}.

% ---
\section{Inclusão de outros arquivos}\label{sec-include}
% ---

É uma boa prática dividir o seu documento em diversos arquivos, e não
apenas escrever tudo em um único. Esse recurso foi utilizado neste
documento. Para incluir diferentes arquivos em um arquivo principal,
de modo que cada arquivo incluído fique em uma página diferente, utilize o
comando:

\begin{verbatim}
   \include{documento-a-ser-incluido}      % sem a extensão .tex
\end{verbatim}

Para incluir documentos sem quebra de páginas, utilize:

\begin{verbatim}
   \input{documento-a-ser-incluido}      % sem a extensão .tex
\end{verbatim}

% ---
\section{Compilar o documento \LaTeX}
% ---

Geralmente os editores \LaTeX, como o
TeXlipse\footnote{\url{http://texlipse.sourceforge.net/}}, o
Texmaker\footnote{\url{http://www.xm1math.net/texmaker/}}, entre outros,
compilam os documentos automaticamente, de modo que você não precisa se
preocupar com isso.

No entanto, você pode compilar os documentos \LaTeX usando os seguintes
comandos, que devem ser digitados no \emph{Prompt de Comandos} do Windows ou no
\emph{Terminal} do Mac ou do Linux:

\begin{verbatim}
   pdflatex ARQUIVO_PRINCIPAL.tex
   bibtex ARQUIVO_PRINCIPAL.aux
   makeindex ARQUIVO_PRINCIPAL.idx 
   makeindex ARQUIVO_PRINCIPAL.nlo -s nomencl.ist -o ARQUIVO_PRINCIPAL.nls
   pdflatex ARQUIVO_PRINCIPAL.tex
   pdflatex ARQUIVO_PRINCIPAL.tex
\end{verbatim}

% ---
\section{Remissões internas}
% ---

Ao nomear a \autoref{tab-nivinv} e a \autoref{fig_circulo}, apresentamos um
exemplo de remissão interna, que também pode ser feita quando indicamos o
\autoref{cap_exemplos}, que tem o nome \emph{\nameref{cap_exemplos}}. O número
do capítulo indicado é \ref{cap_exemplos}, que se inicia à
\autopageref{cap_exemplos}\footnote{O número da página de uma remissão pode ser
obtida também assim:
\pageref{cap_exemplos}.}.
Veja a \autoref{sec-divisoes} para outros exemplos de remissões internas entre
seções, subseções e subsubseções.

O código usado para produzir o texto desta seção é:

\begin{verbatim}
Ao nomear a \autoref{tab-nivinv} e a \autoref{fig_circulo}, apresentamos um
exemplo de remissão interna, que também pode ser feita quando indicamos o
\autoref{cap_exemplos}, que tem o nome \emph{\nameref{cap_exemplos}}. O número
do capítulo indicado é \ref{cap_exemplos}, que se inicia à
\autopageref{cap_exemplos}\footnote{O número da página de uma remissão pode ser
obtida também assim:
\pageref{cap_exemplos}.}.
Veja a \autoref{sec-divisoes} para outros exemplos de remissões internas entre
seções, subseções e subsubseções.
\end{verbatim}

% ---
\section{Divisões do documento: seção}\label{sec-divisoes}
% ---

Esta seção testa o uso de divisões de documentos. Esta é a
\autoref{sec-divisoes}. Veja a \autoref{sec-divisoes-subsection}.

\subsection{Divisões do documento: subseção}\label{sec-divisoes-subsection}

Isto é uma subseção. Veja a \autoref{sec-divisoes-subsubsection}, que é uma
\texttt{subsubsection} do \LaTeX, mas é impressa chamada de ``subseção'' porque
no Português não temos a palavra ``subsubseção''.

\subsubsection{Divisões do documento: subsubseção}
\label{sec-divisoes-subsubsection}

Isto é uma subsubseção.

\subsubsection{Divisões do documento: subsubseção}

Isto é outra subsubseção.

\subsection{Divisões do documento: subseção}\label{sec-exemplo-subsec}

Isto é uma subseção.

\subsubsection{Divisões do documento: subsubseção}

Isto é mais uma subsubseção da \autoref{sec-exemplo-subsec}.


\subsubsubsection{Esta é uma subseção de quinto
nível}\label{sec-exemplo-subsubsubsection}

Esta é uma seção de quinto nível. Ela é produzida com o seguinte comando:

\begin{verbatim}
\subsubsubsection{Esta é uma subseção de quinto
nível}\label{sec-exemplo-subsubsubsection}
\end{verbatim}

\subsubsubsection{Esta é outra subseção de quinto nível}\label{sec-exemplo-subsubsubsection-outro}

Esta é outra seção de quinto nível.


\paragraph{Este é um parágrafo numerado}\label{sec-exemplo-paragrafo}

Este é um exemplo de parágrafo nomeado. Ele é produzida com o comando de
parágrafo:

\begin{verbatim}
\paragraph{Este é um parágrafo nomeado}\label{sec-exemplo-paragrafo}
\end{verbatim}

A numeração entre parágrafos numeradaos e subsubsubseções são contínuas.

\paragraph{Esta é outro parágrafo numerado}\label{sec-exemplo-paragrafo-outro}

Esta é outro parágrafo nomeado.

% ---
\section{Este é um exemplo de nome de seção longo. Ele deve estar
alinhado à esquerda e a segunda e demais linhas devem iniciar logo abaixo da
primeira palavra da primeira linha}
% ---

Isso atende à norma \citeonline[seções de 5.2.2 a 5.2.4]{NBR14724:2011} 
 e \citeonline[seções de 3.1 a 3.8]{NBR6024:2012}.

% ---
\section{Diferentes idiomas e hifenizações}
\label{sec-hifenizacao}
% ---

Para usar hifenizações de diferentes idiomas, inclua nas opções do documento o
nome dos idiomas que o seu texto contém. Por exemplo (para melhor
visualização, as opções foram quebras em diferentes linhas):

\begin{verbatim}
\documentclass[
	12pt,
	openright,
	twoside,
	a4paper,
	english,
	french,
	spanish,
	brazil
	]{abntex2}
\end{verbatim}

O idioma português-brasileiro (\texttt{brazil}) é incluído automaticamente pela
classe \textsf{abntex2}. Porém, mesmo assim a opção \texttt{brazil} deve ser
informada como a última opção da classe para que todos os pacotes reconheçam o
idioma. Vale ressaltar que a última opção de idioma é a utilizada por padrão no
documento. Desse modo, caso deseje escrever um texto em inglês que tenha
citações em português e em francês, você deveria usar o preâmbulo como abaixo:

\begin{verbatim}
\documentclass[
	12pt,
	openright,
	twoside,
	a4paper,
	french,
	brazil,
	english
	]{abntex2}
\end{verbatim}

A lista completa de idiomas suportados, bem como outras opções de hifenização,
estão disponíveis em \citeonline[p.~5-6]{babel}.

Exemplo de hifenização em inglês\footnote{Extraído de:
\url{http://en.wikibooks.org/wiki/LaTeX/Internationalization}}:

\begin{otherlanguage*}{english}
\textit{Text in English language. This environment switches all language-related
definitions, like the language specific names for figures, tables etc. to the other
language. The starred version of this environment typesets the main text
according to the rules of the other language, but keeps the language specific
string for ancillary things like figures, in the main language of the document.
The environment hyphenrules switches only the hyphenation patterns used; it can
also be used to disallow hyphenation by using the language name
`nohyphenation'.}
\end{otherlanguage*}

O idioma geral do texto por ser alterado como no exemplo seguinte:

\begin{verbatim}
  \selectlanguage{english}
\end{verbatim}

Isso altera automaticamente a hifenização e todos os nomes constantes de
referências do documento para o idioma inglês. Consulte o manual da classe
\cite{abntex2classe} para obter orientações adicionais sobre internacionalização de
documentos produzidos com \abnTeX.

A \autoref{sec-citacao} descreve o ambiente \texttt{citacao} que pode receber
como parâmetro um idioma a ser usado na citação.

% ---
\section{Consulte o manual da classe \textsf{abntex2}}
% ---

Consulte o manual da classe \textsf{abntex2} \cite{abntex2classe} para uma
referência completa das macros e ambientes disponíveis. 

Além disso, o manual possui informações adicionais sobre as normas ABNT
observadas pelo \abnTeX\ e considerações sobre eventuais requisitos específicos
não atendidos, como o caso da \citeonline[seção 5.2.2]{NBR14724:2011}, que
especifica o espaçamento entre os capítulos e o início do texto, regra
propositalmente não atendida pelo presente modelo.

% ---
\section{Referências bibliográficas}
% ---

A formatação das referências bibliográficas conforme as regras da ABNT são um
dos principais objetivos do \abnTeX. Consulte os manuais
\citeonline{abntex2cite} e \citeonline{abntex2cite-alf} para obter informações
sobre como utilizar as referências bibliográficas.

%-
\subsection{Acentuação de referências bibliográficas}
%-

Normalmente não há problemas em usar caracteres acentuados em arquivos
bibliográficos (\texttt{*.bib}). Porém, como as regras da ABNT fazem uso quase
abusivo da conversão para letras maiúsculas, é preciso observar o modo como se
escreve os nomes dos autores. Na ~\autoref{tabela-acentos} você encontra alguns
exemplos das conversões mais importantes. Preste atenção especial para `ç' e `í'
que devem estar envoltos em chaves. A regra geral é sempre usar a acentuação
neste modo quando houver conversão para letras maiúsculas.

\begin{table}[htbp]
\caption{Tabela de conversão de acentuação.}
\label{tabela-acentos}

\begin{center}
\begin{tabular}{ll}\hline\hline
acento & \textsf{bibtex}\\
à á ã & \verb+\`a+ \verb+\'a+ \verb+\~a+\\
í & \verb+{\'\i}+\\
ç & \verb+{\c c}+\\
\hline\hline
\end{tabular}
\end{center}
\end{table}


% ---
\section{Precisa de ajuda?}
% ---

Consulte a FAQ com perguntas frequentes e comuns no portal do \abnTeX:
\url{https://github.com/abntex/abntex2/wiki/FAQ}.

Inscreva-se no grupo de usuários \LaTeX:
\url{http://groups.google.com/group/latex-br}, tire suas dúvidas e ajude
outros usuários.

Participe também do grupo de desenvolvedores do \abnTeX:
\url{http://groups.google.com/group/abntex2} e faça sua contribuição à
ferramenta.

% ---
\section{Você pode ajudar?}
% ---

Sua contribuição é muito importante! Você pode ajudar na divulgação, no
desenvolvimento e de várias outras formas. Veja como contribuir com o \abnTeX\
em \url{https://github.com/abntex/abntex2/wiki/Como-Contribuir}.

% ---
\section{Quer customizar os modelos do \abnTeX\ para sua instituição ou universidade?}
% ---

Veja como customizar o \abnTeX\ em: \\
\url{https://github.com/abntex/abntex2/wiki/ComoCustomizar}.

% \chapter{Conteúdos específicos do modelo de trabalho acadêmico}\label{cap_trabalho_academico}

\section{Quadros}

Este modelo vem com o ambiente \texttt{quadro} e impressão de Lista de quadros 
configurados por padrão. Verifique um exemplo de utilização:

\begin{quadro}[htb]
\caption{\label{quadro_exemplo}Exemplo de quadro}
\begin{tabular}{|c|c|c|c|}
	\hline
	\textbf{Pessoa} & \textbf{Idade} & \textbf{Peso} & \textbf{Altura} \\ \hline
	Marcos & 26    & 68   & 178    \\ \hline
	Ivone  & 22    & 57   & 162    \\ \hline
	...    & ...   & ...  & ...    \\ \hline
	Sueli  & 40    & 65   & 153    \\ \hline
\end{tabular}
\fonte{Autor.}
\end{quadro}

Este parágrafo apresenta como referenciar o quadro no texto, requisito
obrigatório da ABNT. 
Primeira opção, utilizando \texttt{autoref}: Ver o \autoref{quadro_exemplo}. 
Segunda opção, utilizando  \texttt{ref}: Ver o Quadro \ref{quadro_exemplo}.

% % ---
% Capitulo de revisão de literatura
% ---
\chapter{Lorem ipsum dolor sit amet}
% ---

% ---
\section{Aliquam vestibulum fringilla lorem}
% ---

\lipsum[1]

\lipsum[2-3]

% ---
% primeiro capitulo de Resultados
% ---
\chapter{Lectus lobortis condimentum}
% ---

% ---
\section{Vestibulum ante ipsum primis in faucibus orci luctus et ultrices
posuere cubilia Curae}
% ---

\lipsum[21-22]

% ---
% segundo capitulo de Resultados
% ---
\chapter{Nam sed tellus sit amet lectus urna ullamcorper tristique interdum
elementum}
% ---

% ---
\section{Pellentesque sit amet pede ac sem eleifend consectetuer}
% ---

\lipsum[24]
% \chapter{Customização DCOMP}

\section{Lista de códigos}

Usado para criar a lista de códigos, adicionar sintaxe highlight, enumerar as linhas e colorir o fundo, para dar destaque a implementação.

Sintaxe básica:
\begin{verbatim}
\begin{codigo}[!htb]
    \caption{Espaço para o título do código}
    \label{Espaço para o label do código, para ser usado na referência}  
    \begin{lstlisting}[language = Linguagem de programação a ser usada]
        <CÓDIGO>
    \end{lstlisting}
\end{codigo}
\end{verbatim}

\begin{codigo}[htb]
  \caption{Código PHP}
  \label{codigophp}
  \begin{lstlisting}[language = php]
       <?php

       echo '%*Olá mundo*)!';
       print '%*Olá mundo*)!';
  \end{lstlisting}
\end{codigo}

\begin{codigo}
  \caption{Código python}
  \label{codigopython}
  \begin{lstlisting}[language = python]
    import numpy as np
 
    def incmatrix(genl1, genl2):
        m = len(genl1)
        n = len(genl2)
        M = None #to become the incidence matrix
        VT = np.zeros((n*m,1), int)  #dummy variable
 
        #compute the bitwise xor matrix
        M1 = bitxormatrix(genl1)
        M2 = np.triu(bitxormatrix(genl2),1) 
 
        for i in range(m-1):
            for j in range(i+1, m):
                [r,c] = np.where(M2 == M1[i,j])
                for k in range(len(r)):
                    VT[(i)*n + r[k]] = 1;
                    VT[(i)*n + c[k]] = 1; 
                    VT[(j)*n + r[k]] = 1;
                    VT[(j)*n + c[k]] = 1;
 
                    if M is None:
                        M = np.copy(VT)
                    else:
                        M = np.concatenate((M, VT), 1)
 
                    VT = np.zeros((n*m,1), int)
 
        return M
\end{lstlisting}
\end{codigo}

\begin{codigo}
  \caption{Codigo Java}
  \begin{lstlisting}[language = Java]
    public class Factorial{
        public static void main(String[] args){   
            final int NUM_FACTS = 100;
            for(int i = 0; i < NUM_FACTS; i++)
                System.out.println( i + "! is " + factorial(i) + factorial(i) factorial(i) factorial(i));
        }

        public static int factorial(int n){
            int result = 1;
            for(int i = 2; i <= n; i++)
                result *= i;
            return result;
        }
    }
\end{lstlisting}
\end{codigo}



\section{Lista de Algoritmos}

Usado para criar a lista de algoritmos ou pseudocodigos.

Sintaxe básica:
\begin{verbatim}
\begin{algoritmo}[!htb]
    \caption{Espaço para o título do algoritmo ou pseudocodigo}
    \label{label do do algoritmo ou pseudocodigo, para ser usado na referência}  
    <ESPAÇO RESERVADO PARA USAR SEU PACOTE FAVORITO DE CÓDIGOS>
\end{algoritmo}
\end{verbatim}


\begin{algoritmo}[htb]
	\caption{Algoritmo exemplo}
	\label{alg1}
	\begin{algorithm}[H]
 	\KwData{this text}
 	\KwResult{how to write algorithm with \LaTeX2e }
 	initialization\;
 	\While{not at end of this document}{
  		read current\;
  		\eIf{understand}{
   			go to next section\;
   			current section becomes this one\;
   		}{
   			go back to the beginning of current section\;
  		}
 	}
	\end{algorithm}
\end{algoritmo}

% \chapter{Conclusão}

\lipsum[31-33]

\phantompart
\bibliography{Bibliografia}

%%%%%%%%%%%%%%%%%%%%%%%%%%%%%%%%%%%%%%%%%%%%%%%%%%%%%%%%%%%%%%%%%%%%%%%%%%
% ELEMENTOS PÓS-TEXTUAIS
%%%%%%%%%%%%%%%%%%%%%%%%%%%%%%%%%%%%%%%%%%%%%%%%%%%%%%%%%%%%%%%%%%%%%%%%%%

\postextual

\renewcommand{\chapnumfont}{\chaptitlefont}
\renewcommand{\afterchapternum}{}
\include{Pos_Textual/Apendices}
\include{Pos_Textual/Anexos}

\end{document}
