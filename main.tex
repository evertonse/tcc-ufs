%v%%%%%%%%%%%%%%%%%%%%%%%%%%%%%%%%%%%%%%%%%%%%%%%%%%%%%%%%%%%%%%%%%%%%%%%%%


% abnTeX2: Modelo de Trabalho Acadêmico em conformidade com 
% as normas da ABNT


%%%%%%%%%%%%%%%%%%%%%%%%%%%%%%%%%%%%%%%%%%%%%%%%%%%%%%%%%%%%%%%%%%%%%%%%%%


\documentclass[english,
               brazil,
               bsc] %Opções bsc (TCC) e msc (Mestrado)
               {dcomp-abntex2}




%%%%%%%%%%%%%%%%%%%%%%%%%%%%%%%%%%%%%%%%%%%%%%%%%%%%%%%%%%%%%%%%%%%%%%%%%%
% Área para adição de pacotes extras
%%%%%%%%%%%%%%%%%%%%%%%%%%%%%%%%%%%%%%%%%%%%%%%%%%%%%%%%%%%%%%%%%%%%%%%%%%


% \usepackage{lipsum} % Retirar para a versão final do documento
\usepackage{float}
\usepackage{pgfgantt}
\usepackage{lscape}
\usepackage{makecell}
\usepackage{amsmath}
\usepackage{tabularx}
\usepackage[bottom]{footmisc}

% Not using any of these, because inkscape cli is not working as intended
% \usepackage[inkscapeformat=png]{svg}
% \usepackage{svg}

% \lstset{
%     literate=%
%     {á}{{\'a}}1
%     {č}{{\v{c}}}1
%     {ď}{{\v{d}}}1
%     {é}{{\'e}}1
%     {ě}{{\v{e}}}1
%     {í}{{\'i}}1
%     {ň}{{\v{n}}}1
%     {ó}{{\'o}}1
%     {ř}{{\v{r}}}1
%     {š}{{\v{s}}}1
%     {ť}{{\v{t}}}1
%     {ú}{{\'u}}1
%     {ů}{{\r{u}}}1
%     {ý}{{\'y}}1
%     {ž}{{\v{z}}}1
%     {Á}{{\'A}}1
%     {Č}{{\v{C}}}1
%     {Ď}{{\v{D}}}1
%     {É}{{\'E}}1
%     {Ě}{{\v{E}}}1
%     {Í}{{\'I}}1
%     {Ň}{{\v{N}}}1
%     {Ó}{{\'O}}1
%     {Ř}{{\v{R}}}1
%     {Š}{{\v{S}}}1
%     {Ť}{{\v{T}}}1
%     {Ú}{{\'U}}1
%     {Ů}{{\r{U}}}1
%     {Ý}{{\'Y}}1
%     {Ž}{{\v{Z}}}1
% }

\usepackage[T1]{fontenc}
\usepackage{listings}
\lstloadlanguages{haskell}
\lstloadlanguages{C}
\lstloadlanguages{tex}
\lstset{language=haskell}
\lstset{language=C}
\lstset{language=tex}
\lstset{%
        % basicstyle=\scriptsize,
        inputencoding=utf8,
        extendedchars=true,
        literate=%
        {é}{{\'{e}}}1
        {è}{{\`{e}}}1
        {ê}{{\^{e}}}1
        {ë}{{\¨{e}}}1
        {É}{{\'{E}}}1
        {Ê}{{\^{E}}}1
        {û}{{\^{u}}}1
        {ù}{{\`{u}}}1
        {ú}{{\'{u}}}1
        {â}{{\^{a}}}1
        {à}{{\`{a}}}1
        {á}{{\'{a}}}1
        {ã}{{\~{a}}}1
        {Á}{{\'{A}}}1
        {Â}{{\^{A}}}1
        {Ã}{{\~{A}}}1
        {ç}{{\c{c}}}1
        {Ç}{{\c{C}}}1
        {õ}{{\~{o}}}1
        {ó}{{\'{o}}}1
        {ô}{{\^{o}}}1
        {Õ}{{\~{O}}}1
        {Ó}{{\'{O}}}1
        {Ô}{{\^{O}}}1
        {î}{{\^{i}}}1
        {Î}{{\^{I}}}1
        {í}{{\'{i}}}1
        {Í}{{\~{Í}}}1
}

% Already Imported by dcomp-abntex2.cls:
% \usepackage[utf8]{inputenc}


\restylefloat{table}


%Utilize aqui seu pacote preferido para algoritmos
\usepackage[linesnumbered]{algorithm2e}


%%%%%%%%%%%%%%%%%%%%%%%%%%%%%%%%%%%%%%%%%%%%%%%%%%%%%%%%%%%%%%%%%%%%%%%%%%


%Compila o índice
\makeindex


\begin{document}


% Seleciona o idioma do documento (conforme pacotes do babel)
\selectlanguage{brazil}


% Retira espaço extra obsoleto entre as frases.
\frenchspacing 


%%%%%%%%%%%%%%%%%%%%%%%%%%%%%%%%%%%%%%%%%%%%%%%%%%%%%%%%%%%%%%%%%%%%%%%%%%
% ELEMENTOS PRÉ-TEXTUAIS
%%%%%%%%%%%%%%%%%%%%%%%%%%%%%%%%%%%%%%%%%%%%%%%%%%%%%%%%%%%%%%%%%%%%%%%%%%


\pretextual




% \titulo{Desenvolvimento de um Compilador de BRDFs em \LaTeX{}  para linguagem de shading GLSL, através da técnica Pratt Parsing}
% \titulo{Compilador de funções de distribuição de reflexão bidirecional escritas em \LaTeX{} para linguagem de shading GLSL, através da técnica Pratt Parsing}
\titulo{Compilador de Funções de Distribuição de Reflexão Bidirecional descritas em \LaTeX{} para Linguagem de Shading}


\autor{Everton Santos de Andrade Júnior}
\orientador{Dra. Beatriz Trinchão Andrade}
% \coorientador{Dr. Gastao Florencio Miranda Junior}
\curso{Ciência da Computação}


\inserirInformacoesPDF


\imprimircapa
\imprimirfolhaderosto*


% \begin{dedicatoria}
   \vspace*{\fill}
   \centering
   \noindent
   \textit{Esta página foi deixada em branco, não-ironicamente, de propósito} \vspace*{\fill}
\end{dedicatoria}
% ---

% \include{Pre_Textual/Agradecimentos}
% \include{Pre_Textual/Epigrafe}
% resumo em português
\setlength{\absparsep}{18pt} % ajusta o espaçamento dos parágrafos do resumo
\begin{resumo}
 
% Segundo a \citeonline[3.1-3.2]{NBR6028:2003}, o resumo deve ressaltar o objetivo, o método, os resultados e as conclusões do documento. A ordem e a extensão destes itens dependem do tipo de resumo (informativo ou indicativo) e do tratamento que cada item recebe no documento original. O resumo deve ser precedido da referência do documento, com exceção do resumo inserido no próprio documento. (\ldots) As palavras-chave devem figurar logo abaixo do resumo, antecedidas da expressão Palavras-chave:, separadas entre si por ponto e finalizadas também por ponto.

% O presente trabalho propõe o desenvolvimento de um compilador de funções de distribuição de reflexão bidirecional (BRDFs) expressas em LaTeX para a linguagem de shading GLSL, utilizando a técnica de parsing de Pratt. O objetivo é automatizar o processo de tradução de funções complexas de materiais, frequentemente descritas em LaTeX, para o código GLSL utilizado em programação de shaders para OpenGL. Para isso, será empregada a técnica de parsing de Pratt, uma abordagem eficiente e flexível para analisar e traduzir expressões matemáticas e lógicas. O trabalho incluirá a implementação do compilador, a análise de desempenho e precisão da tradução, e a comparação com métodos tradicionais de tradução manual. Ao final, espera-se oferecer uma ferramenta útil para desenvolvedores e pesquisadores na área de computação gráfica, facilitando a utilização e compreensão de modelos de materiais complexos em aplicações gráficas. Palavras-chave: Compilador, BRDFs, LaTeX, GLSL, Pratt Parsing.

O presente trabalho propõe o desenvolvimento de um compilador de funções de distribuição de reflexão bidirecional (BRDFs) expressas em LaTeX para a linguagem de shading GLSL, utilizando a técnica de parsing de Pratt. O objetivo é automatizar o processo de tradução de funções complexas de materiais, frequentemente descritas em LaTeX, para o código GLSL utilizado em programação de shaders para OpenGL. Ao fornecer essa ferramenta, pretende-se não apenas simplificar o trabalho dos desenvolvedores e pesquisadores na área de computação gráfica, mas também democratizar o acesso e compreensão de modelos de materiais complexos. Além disso, ao permitir que as BRDFs sejam expressas em uma forma mais familiar e acessível, como a notação matemática, o compilador reduz a barreira de entrada para aqueles que não estão familiarizados com linguagens programação. Isso pode facilitar a colaboração interdisciplinar entre profissionais de diferentes áreas, como artistas visuais, designers e cientistas de materiais, que desejam explorar e entender o comportamento visual de materiais em suas aplicações.

 \textbf{Palavras-chave}: Compilador, BRDFs, LaTeX, GLSL, Shading, Pratt Parsing.
\end{resumo}

% \include{Pre_Textual/Abstract}




\mostrarlistadeILUSTRACOES
% \mostrarlistadeQUADROS
\mostrarlistadeTABELAS
\mostrarlistadeCODIGOS
\mostrarlistadeALGORITMOS
 
% \include{Pre_Textual/Abreviaturas}
% \include{Pre_Textual/Simbolos}
    
\mostrarSUMARIO


%%%%%%%%%%%%%%%%%%%%%%%%%%%%%%%%%%%%%%%%%%%%%%%%%%%%%%%%%%%%%%%%%%%%%%%%%%
% ELEMENTOS TEXTUAIS
%%%%%%%%%%%%%%%%%%%%%%%%%%%%%%%%%%%%%%%%%%%%%%%%%%%%%%%%%%%%%%%%%%%%%%%%%%


\textual


%%%%%%%%%%%%%
% Introdução
%%%%%%%%%%%%%
\chapter{Introdução}
\label{introduction}


Na computação gráfica, a representação realista de cenas tridimensionais depende fortemente da modelagem da luz e dos materiais que compõem os objetos na cena. A interação da luminosidade incidente com esses materiais é crucial para a geração de imagens fiéis à realidade. Uma abordagem fundamental para modelar essa interação é por meio das funções de distribuição de refletância bidirecional, conhecidas como BRDFs (do inglês, \textit{Bidirectional Reflectance Distribution Functions}).




As BRDFs, essencialmente, calculam a proporção entre a energia luminosa que atinge um ponto na superfície e como essa energia é refletida, transmitida ou absorvida \cite{pbr}. Na renderização, essas funções são implementadas por meio de programas especializados nas unidades de processamento gráfico (GPUs), chamados de \textit{shaders}. Cada interface de programação, do inglês \textit{ Application Programming Interface} (API),  disponibiliza etapas diferentes onde esses executáveis podem ser programados durante o processo de renderização. Esses \textit{shaders} concedem a capacidade de cada objeto renderizado ter sua aparência configurada por meio de um código que implementa uma BRDF.




\section{Motivação}




Existem linguagens específicas para a programação de \textit{shaders}, as quais permitem a criação e edição de procedimentos que representam uma BRDF. No entanto, essa aplicação requer conhecimento especializado em programação. Essa barreira técnica pode restringir a exploração dos efeitos visuais por profissionais de áreas não relacionadas à programação. Diante disso, surge a necessidade de ferramentas mais acessíveis para a criação de \textit{shaders}.


No meio acadêmico, as BRDFs são comumente descritas por fórmulas escritas em \LaTeX{}\footnote{\LaTeX{} foi desenvolvido por Leslie Lamport \cite{leslie1994latex}, baseado no sistema \TeX{} de Donald Knuth, e permite a representação de equações matemáticas complexas.}, que é um sistema de preparação de documentos para alta qualidade tipográfica, geralmente utilizado para a criação de artigos científicos. Desta forma, uma abordagem promissora para simplificar a criação de \textit{shaders} é o desenvolvimento de um compilador capaz de traduzir BRDFs   escritas em \LaTeX{} para \textit{shaders}. Isso permitiria uma maior acessibilidade e democratização na criação de efeitos visuais complexos.


\section{Objetivo}
%%
Este trabalho visa projetar e implementar um compilador que, a partir de BRDFs escritas como equações em \LaTeX{}, seja capaz de gerar código de sombreamento (\textit{shading})\footnote{O termo \textit{shading} refere-se ao processo computacional realizado por um \textit{shader}, ou seja, o cálculo de cor, iluminação e textura para cada ponto de uma superfície em uma cena gráfica.} na linguagem alvo da API OpenGL. O resultado será um \textit{shader} capaz de reproduzir as características de reflexão da BRDF original ou, ao menos, alcançar uma aproximação satisfatória dessas características, levando em conta as limitações da linguagem de \textit{shading} da API, principalmente as representações de dados de forma discreta.


\section{Estrutura do Documento}
No \autoref{conceitos}, descrevemos os conhecimentos necessários para entender BRDFs, incluindo quantificação de luminosidade e radiação, e  conceitos de compiladores, como tokenização e construção da árvore sintática.


O \autoref{revisao} faz um mapeamento sistemático, utilizando termos de busca para identificar trabalhos relevantes sobre o desenvolvimento de compiladores para traduzir BRDFs de \LaTeX{}  para \textit{shaders}. Os critérios de inclusão e exclusão são definidos para filtrar os resultados. Além disso, são descritos os resultados encontrados em diversas bases de dados, como IEEE Xplore, BDTD, CAPES, ACM Digital Library e Google Scholar, bem como a análise de repositórios online como GitHub e SourceForge.


No \autoref{metodologia} é descrito o método para desenvolver o compilador proposto. São definidas etapas para alcançar os objetivos especificados neste trabalho e casos de teste são projetados para validação. Esse capítulo também inclui as convenções matemáticas e operações que o compilador deve implementar.



O \autoref{chapter-dev} descreve o desenvolvimento deste trabalho, que consiste na implementação de um analisador léxico, sintático e semântico, além da geração de código GLSL, todos implementados na linguagem de programação Odin \footnote{\url{https://odin-lang.org/}}. O método de análise sintática utilizado é o de Pratt, que, juntamente com a inferência de tipos e a visualização da AST por meio de arquivos de imagem, compõe boa parte do projeto. A linguagem desenvolvida possui uma gramática restrita em comparação com \LaTeX{}, priorizando as construções necessárias para descrever BRDFs, o que torna viável a construção deste compilador.

O \autoref{chapter.resultados} apresenta os experimentos elaborados para validar a implementação. Nele, são apresentadas as equações que descrevem as BRDFs, acompanhadas do código gerado pelo compilador desenvolvido. Além disso, são exibidas as renderizações de três objetos distintos para cada código de BRDF testado. Por último, no \autoref{chapter-conclusion}, são recapitulados os objetivos propostos e discutidas as potenciais direções futuras deste trabalho.

% O \autoref{chapter-dev} descreve o desenvolvimento deste trabalho, que consistem na implementação de um analisador léxico, sintático e semantico, e geração de código GLSL  implementados na linguagem de programação Odin \footnote{\url{https://odin-lang.org/}}, incluindo o método de análise sintática de Pratt, inferencia de tipos e visualização da AST por arquivo de imagem. A linguagem desenvolvida possui uma gramática restrita em comparação com \LaTeX{}, foi priorizado  as construções necessarias para descrever BRDFs de modo a deixar viavel a construção deste compilador.
%
% O \autoref{chapter.resultados} também descreve os experimentos elaborados para validar a implementação. Nele, é apresentado as equações que decrevem a BRDFs junto com seu  código gerado pelo compilador desnvolvido. Além da a renderização de tres objeto distintos para cada código de BRDFs testada.  Por útilmo, no \autoref{chapter-conclusion} é recapitulado sobre os objetivos proposoto e quais as potenciais direções deste trabalho.





%%%%%%%%%%%%%%%
%% Conceitos %%
%%%%%%%%%%%%%%%

\section{Conceitos}

\begin{frame}{Conceitos - BRDFs}

As BRDFs calculam a proporção entre a energia luminosa que atinge um ponto na superfície e como essa energia é:
    \begin{itemize}
        \item \textbf{Refletida}
        \item \textbf{Transmitida}
        \item \textbf{Absorvida}
    \end{itemize}
    \begin{figure}[H]
        \begin{center}
            \includegraphics[width=0.65\textwidth]{./Imagens/difusa-e-specular.png}
        \end{center}
    \end{figure}
\end{frame}


\begin{frame}{Equação de Renderização}
Um renderizador estima a ``quantidade'' luz $L_o$ que sai de um ponto em uma direção ${\omega}_o$ usando a equação de renderização (\cite{rendering_equation}). Nela, a BRDF é encontrada:

\begin{equation}\label{eq-rendering-equation}
\begin{aligned}
  &L_o(p, {\omega}_o) = L_e(p, {\omega}_o) +
\int_{H^2}f(p, {\omega}_i, {\omega}_o){L_i(p,{\omega}_i)\cos(\theta_i)d{\omega}_i}\\
    &L_o \text{ é radiância de saída (\textit{outgoing})}\\
    &L_e \text{ é radiância emitida pela superfície (i.e. fonte de luz)}\\
    &L_i \text{ é radiância incidente na superfície}\\
    &{\omega}_i \text{ é a direção do raio incidente}\\
    &{\omega}_o \text{ é a direção do raio refletido}\\
    % &H^2 \text{ são todas as direções no hemisfério no ponto $p$}\\
    % &\theta_i \text{ ângulo entre direção incidente e a normal da superfície}\\
    &f \text{ função de refletância}\\
\end{aligned}
\end{equation}
\end{frame}



% SLIDE 2
\begin{frame}{\textit{Pipeline} de GPU}
    Etapas Principais:
    \begin{enumerate}
        \item \textbf{\textit{Shader} de Vértice}: Processa e transforma vértices.
        \item \textbf{Rasterização}: Gera fragmentos a partir de primitivas.
        \item \textbf{\textit{Shader} de Fragmento}: Determina a cor final dos fragmentos.
    \end{enumerate}

    Fluxo de dados são:  CPU $\to$ GPU $\to$ \textit{Shaders} $\to$ Imagem final.
    \begin{figure}[H]
        % \centering
        \includegraphics[scale=0.25]{./Imagens/gpu_pipeline.png}
        % \caption{\small Representação do \textit{pipeline} de GPU.}
    \end{figure}
\end{frame}

% SLIDE 3
\begin{frame}[fragile]{\textit{Shader} de Vértice}
    \begin{itemize}
        \item Aplicação de transformações (ex.: rotação, projeção).
        \item Transmissão de dados (ex.: normais) ao \textit{shader} de fragmento.
    \end{itemize}

    \begin{clang}
#version 330 core
layout(location = 0) in vec3 inPosition;
layout(location = 1) in vec3 inNormal;

uniform mat4 modelViewProjection;

out vec3 fragNormal;

void main() {
    vec3 manipulatedPosition = inPosition + (sin(gl_VertexID * 0.1) * 0.1);
    fragNormal = inNormal;
    gl_Position = modelViewProjection * vec4(manipulatedPosition, 1.0);
}
    \end{clang}
\end{frame}

% SLIDE 4
\begin{frame}{\textit{Shader} de Fragmento}
    Nesse \textit{shader} é onde a cor final é calculada. Roda paralelamente 1 vez para cada ``pixel''.
    \begin{figure}[H]
        \centering
        \includegraphics[scale=0.5]{./Imagens/per_vertex_per_frag.png}
        \caption{\small Diferença entre \textit{shading} por vértice e por fragmento.}
    \end{figure}
\end{frame}

% \begin{frame}{Shaders e implementação}
%     \begin{itemize}
%         \item Na renderização, BRDFs são implementadas por \textit{shaders}, programas especializados executados na GPU.
%         \item APIs gráficas permitem programar esses shaders em diferentes etapas do processo de renderização ( OpenGL).
%         \item Com os shaders, cada objeto renderizado pode ter sua aparência configurada por meio de códigos que implementam BRDFs específicas.
%     \end{itemize}
% \end{frame}

\begin{frame}{Compilação: Visão Geral}
    Comilador $C: L_1 \rightarrow L_2$ mapeia programa de $L_1$ para $L_2$ preservando semântica; mantém mesmo significado algorítmico.

    \begin{enumerate}

        \item \textbf{Análise Léxica (Lexing):}
              \begin{itemize}
                  \item Divide entrada em tokens (palavras-chave, identificadores)
                  \item Reconhecível por máquinas de estado
              \end{itemize}
        \item \textbf{Análise Sintática (Parsing):}
              \begin{itemize}
                  \item Valida tokens segundo gramática
                  \item Constrói árvore sintática hierárquica
              \end{itemize}
        \item \textbf{Verificação de Tipos:}
              \begin{itemize}
                  \item Garante consistência de tipos
                  \item Previne erros em tempo de execução
              \end{itemize}
        \item \textbf{Emissão de Código:}
              \begin{itemize}
                  \item Gera código na linguagem alvo
              \end{itemize}
    \end{enumerate}
\end{frame}




%%%%%%%%%%%%%%%%%%%%%%%%%%%
%% Revisão Bibliográfica %%
%%%%%%%%%%%%%%%%%%%%%%%%%%%
%%%%%%%%%%%%%%%%%%%%%%%%%%%%%%%%%%%%%%%%%%%%%%%%%%%%%%%%%%%%%%%%%%%%%%%%%%
% Revisão Bibliográfica
%%%%%%%%%%%%%%%%%%%%%%%%%%%%%%%%%%%%%%%%%%%%%%%%%%%%%%%%%%%%%%%%%%%%%%%%%%

\chapter{Revisão Bibliográfica} \label{revisao}


Para esta seção, será conduzida uma revisão da literatura com o objetivo de explorar trabalhos relacionados ao desenvolvimento de compiladores para tradução de BRDFs expressas em \LaTeX{} para a linguagem de \textit{shading}, empregando técnicas de \textit{parsing}. O processo de busca será conduzido em duas etapas distintas. Inicialmente, será realizado um levantamento dos trabalhos existentes nas bases de dados  com relevantes periódicos, anais de eventos, artigos e trabalhos. Por fim, será realizada uma busca por produtos ou ferramentas similares no mercado, utilizando \textit{strings}\footnote{Durante todo este documento, uma \textit{string} é uma cadeia de caracteres.} de busca específicas em repositórios digitais, especificamente GitHub e SourceForge. Esses processos de busca permitirão identificar referências relevantes e estabelecer um panorama do estado da arte no campo dos compiladores de BRDFs  para \textit{shaders}, contribuindo para a compreensão do contexto acadêmico e prático no qual este trabalho se insere.


\section{Mapeamento Sistemático}


Com o intuito de obter resultados relevantes para a pesquisa, foram elaboradas frases de busca com base nos termos-chave relacionados ao tema deste trabalho. Também foram criadas questões de pesquisa para guiar a seleção dos trabalhos.


\subsection{Seleção das Bases}
As bases escolhidas foram: ACM Digital Library \footnote{\url{https://dl.acm.org/}},  IEEE Xplorer Digital Library \footnote{\url{https://ieeexplore.ieee.org/}},  Biblioteca Digital Brasileira de Teses e Dissertações (BDTD) \footnote{\url{https://bdtd.ibict.br/}}, Portal de Periódicos da CAPES \footnote{\url{https://www-periodicos-capes-gov-br.ezl.periodicos.capes.gov.br/index.php?}},  Google Acadêmico \footnote{\url{https://scholar.google.com/}}. Essas foram escolhidas por serem acessíveis gratuitamente pela afiliação à Universidade Federal de Sergipe, já o Google Scholar foi escolhido por agregar pesquisas em outras bases que possam ter trabalhos relevantes.


%
%


%
% \url{https://www-periodicos-capes-gov-br.ezl.periodicos.capes.gov.br/}








\subsection{Questões de Pequisa}  \label{questoes-pesquisa}


Foram elaboradas questões de pesquisa específicas que servem como guia para identificar e selecionar trabalhos científicos capazes de fornecer estratégias para o desenvolvimento do nosso projeto. Essas questões orientam a busca por artigos que apresentem soluções para desafios como: transformar descrições matemáticas de BRDFs em sequências computáveis adequadas para execução por renderizadores; representar modelos de \textit{shading} em estruturas hierárquicas, como árvores; e desenvolver estratégias de compilação de BRDFs para linguagens de \textit{shading}. A partir desse processo, foram identificados e selecionados os trabalhos que melhor atendem às questões propostas, garantindo maior relevância para este estudo.


\begin{enumerate}
  \item Quais são as abordagens mais comuns utilizadas na criação de compiladores para tradução de BRDFs expressas em alguma linguagem de texto, como \LaTeX{}, para \textit{shaders}?

  \item Quais as técnicas de \textit{parsing} têm sido aplicadas no desenvolvimento de compiladores para linguagens matemáticas?

  %% @@@ Novo deve ser referenciado
  \item O trabalho utiliza gramáticas ou árvores de sintaxe para gerar/representar BRDFs ?

  %% @@@ Novo deve ser referenciado
  \item De que maneira é possível transformar BRDFs em representações hierárquicas estruturadas?

  %% @@@ Novo deve ser referenciado
  \item Como operações matemáticas de BRDFs podem ser decompostas em cálculos modulares, permitindo sua implementação em sistemas computacionais?

 \item Quais são os principais desafios enfrentados ao traduzir funções matemáticas complexas, como as BRDFs, em \textit{shaders}?

 \item Quais são as ferramentas e recursos disponíveis para auxiliar no desenvolvimento de compiladores para BRDFs e \textit{shaders}, e como eles podem ser integrados ao processo de desenvolvimento?

\end{enumerate}




\subsection{Termos de Busca}
 As frases foram construídas considerando suas variações equivalentes através de operadores lógicos. Posteriormente, as frases de pesquisa foram adaptadas de acordo com as características individuais de cada base de dados utilizada. Os termos-chave escolhidos foram: (("shader" OR "shading") AND "BRDF" AND ("compiler"\ OR "parser"\ OR "grammar")). As adaptações para cada base e o número de trabalhos encontrados são listados na \autoref{tab-bases}.



\begin{table}[H]
\ABNTEXfontereduzida
\caption{\small Tabela de pesquisa.}
\label{tab-bases}
\begin{tabular}{p{2.6cm}|p{6.0cm}|p{2.25cm}|p{3.40cm}}
  %\hline
   \textbf{Bases} & \textbf{Termos de Pesquisa}  & \textbf{Resultados}\\
   \hline
    IEEE Xplore Digital Library
    &
    ("Full Text \& Metadata":brdf)
AND (("Full Text \& Metadata":shader) OR  ("Full Text \& Metadata":shading))
AND (("Full Text \& Metadata":compiler) OR  ("Full Text \& Metadata":parsing) OR  ("Full Text \& Metadata":parser) OR  ("Full Text \& Metadata":grammar))
   & 36
    \\ \hline


    BDTD
    & (Todos os campos:compiler OU Todos os campos:parsing OU Todos os campos:parser OU Todos os campos:compilador) E (Todos os campos:shader OU Todos os campos:shading) E (Todos os campos:brdf)
    & 0
    \\ \hline
    CAPES Periódico
    &  Qualquer campo contém brdf E 
 Qualquer campo contém compi* E shad*  
    & 0
    \\ \hline


  ACM Digital Library
  & AllField:((shader OR shading) AND brdf AND (compiler OR compiling) AND (parser OR grammar OR parsing))
  & 46
    \\ \hline


 Google Acadêmico
  & 
  ("BRDF" AND ("COMPILER" OR "COMPILING") AND( "PARSER" OR "PARSING") AND ("SHADER" OR "SHADING"))
  & 69
   % \hline
\end{tabular}
\end{table}


\subsection{Critérios}


Para garantir a relevância dos resultados obtidos, seguimos os critérios de inclusão e exclusão estabelecidos, de forma a filtrar os resultados. Ao fim desse procedimento, apenas os resultados com maior compatibilidade com este trabalho foram analisados e descritos de maneira detalhada. O resultados se encontram na \autoref{tab-result}.


\subsubsection{Critérios de Inclusão}


\begin{enumerate}
  \item Foram incluídos artigos relacionados às palavras-chaves;

  \item Foram incluídos artigos que de alguma forma citem a criação de um compilador ou um \textit{parser};

  \item Foram incluídos artigos que utilizam representações hierárquicas para BRDFs, pois contribuem para a compreensão do processo de decomposição dessas funções em estruturas adequadas ao nosso compilador, que utiliza árvores em suas análises e na geração de código.

  \item Foram incluídos artigos que apresentam linearização de BRDFs em sequências computáveis, relevantes para a geração de código a partir de árvores sintáticas.
\end{enumerate}


\subsubsection{Critérios de Exclusão}


\begin{enumerate}
  \item Foram excluídos artigos que dispunham de \textit{links} incorretos e ou quebrados;
  \item Foram excluídos artigos que não estão relacionados com as questões de pesquisa da \autoref{questoes-pesquisa};
  % \item Foram excluídos artigos que não têm como entrada uma BRDF no formato de equação, ou seja, utilizam a representação diretamente como código;
  \item Foram excluídos artigos que não abordam nenhum dos seguintes aspectos: a geração de \textit{shaders} como saída, a estruturação hierárquica de BRDFs ou a decomposição das operações matemáticas associadas a BRDFs;

  \item Foram excluídos artigos que não citam BRDFs e compilador ou árvores em seu resumo;
  \item Se, após a leitura completa, o artigo não concerne os interesses deste trabalho, esse foi excluído.
\end{enumerate}




\begin{table}[H]
\ABNTEXfontereduzida
  \caption{\small Resultados das bases após aplicar os critérios.}
\label{tab-result}
\begin{tabular}{p{6.6cm}|p{6.6cm}}
  %\hline
   \textbf{Bases}  & \textbf{Filtrados}\\
   \hline
    IEEE Xplore Digital Library
    % NOTE: Eram 2, mas removi 'Tree-Structured Shading Decomposition'.
    % NOTE: Decidi voltar para 2 e incluir novamente o artigo removido
   & 2
    \\ \hline
    BDTD
    & 0
    \\ \hline
    CAPES Periódico
    & 0
    \\ \hline


  ACM Digital Library
  & 1
    \\ \hline
 Google Acadêmico
  & 1
   % \hline
\end{tabular}
\end{table}





\subsection{Descrição dos Trabalhos Relacionados}


\subsubsection{genBRDF: Discovering New Analytic BRDFs with Genetic Programming}


Neste artigo é introduzido um \textit{framework} chamada genBRDF, a qual aplica técnicas de programação genética para explorar e descobrir novas BRDFs de maneira analítica \cite{genbrdf}. O processo inicia com a descrição analítica de uma BRDF existente\footnote{As BRDFs de entrada são escritas na linguagem definida pela gramática apresentada pelos autores, que permite operações matemáticas, como \verb"sin", \verb"cos", \verb"tan", \verb"exp", \verb"asin", \verb"acos", \verb"dot", \verb"+", \verb"sqrt". Como exemplo, os autores ilustram ${brdf}_{init} = \rho_s \cos(R \cdot V)^\alpha$, como uma BRDF válida.}, e interativamente aplica mutações e recombinações de partes das expressões matemáticas que compõem essas BRDFs à medida que novas gerações surgem.

Essas mutações são guiadas por uma função \textit{fitness}.
Esse tipo de função atua como o inverso de uma função de erro, sendo baseada em um \textit{dataset} de materiais previamente medidos. Ela direciona as mutações ao avaliar a qualidade das expressões geradas, permitindo que o \textit{framework} identifique as soluções viáveis após analisar milhares de variantes.

Os autores geraram uma gramática que inclui constantes e operadores matemáticos comuns encontrados em equações de BRDFs. A gramática é compilada, e a árvore sintática resultante passa por modificações realizadas pelo algoritmo genético. Nós na árvore podem ser trocados, substituídos, removidos e novos nós podem ser adicionados. Esse processo, após refinamento e análise, resulta em novas BRDFs. Alguns dos novos modelos de BRDF apresentados no documento incluem exemplares que superam os modelos existentes em termos de precisão e simplicidade.
 
Esse artigo se concentra em automaticamente encontrar novos modelos analíticos de BRDF, em vez de compilar diretamente equações BRDF em linguagens de \textit{shading}. Embora a representação das expressões das BRDFs possa potencialmente inspirar o nosso trabalho, o principal objetivo do artigo difere do nosso tema.


\subsubsection{Slang: language mechanisms for extensible real-time shading systems}


O artigo descreve a linguagem \texttt{Slang}, uma extensão da amplamente utilizada linguagem de \textit{shading} HLSL, projetada para aprimorar a modularidade e a extensibilidade \cite{slang}. A abordagem de \textit{design} da \texttt{Slang} é baseada em dois princípios fundamentais: garantir compatibilidade com o HLSL existente sempre que possível e introduzir recursos inspirados em linguagens de programação \textit{mainstream}, visando facilitar a familiaridade e a intuição dos desenvolvedores. O compilador desenvolvido neste estudo é capaz de gerar código de \textit{shading} em HLSL, que pode ser compilado por um compilador de HLSL existente.

%%
O autor enfatiza que cada extensão da \texttt{Slang} foi projetada para permitir uma transição gradual a partir do código HLSL já desenvolvido, garantindo a compatibilidade com a maioria das construções sintáticas e semânticas da linguagem HLSL. Isso elimina a necessidade de uma migração completa. Algumas dessas extensões incluem: funções genéricas, estruturas genéricas e tipos que implementam interfaces específicas, semelhantes às interfaces em \texttt{Java}.

Um exemplo do uso de interfaces e funções genéricas em \texttt{Slang} é apresentado no \autoref{cod-sland}. Ao declarar a conformidade com a interface \texttt{IFoo}, a definição da estrutura \texttt{MyType} deve incluir um método chamado \texttt{myMethod}, com a assinatura correspondente àquela definida na interface \texttt{IFoo}. Já na função \texttt{myGenericMethod}, o parâmetro \texttt{arg} é genérico, permitindo maior flexibilidade na implementação.

\begin{codigo}[H]
  \caption{\small Código de interface em \texttt{Slang}. }
  \label{cod-sland}
\begin{lstlisting}[language=C, frame=none, inputencoding=utf8]
interface IFoo {
    int myMethod(float arg);
}

struct MyType : IFoo {
    int myMethod(float arg) {
        return (int)arg + 1;
    }
}

int myGenericMethod<T>(T arg) where T : IFoo {
    return arg.myMethod(1.0);
}
\end{lstlisting}
\end{codigo}

Enquanto o artigo tenta melhorar a flexibilidade e a extensibilidade dos sistemas de \textit{shading} em tempo real, o nosso trabalho se concentra na compilação de equações BRDF em linguagens de \textit{shading}. Embora ambos os projetos envolvam o uso de \textit{shaders} e a geração de código de \textit{shading}, as abordagens e os focos são distintos

%%%%%%%%%%%%%%%%%%%%%%%%%%%%%%%%%%%%%%%%%%%%%%%
%%%%%%%%%%%%%%% REMOVED FOR NOW %%%%%%%%%%%%%%%
%%%%%%%%%%%%%%%%%%%%%%%%%%%%%%%%%%%%%%%%%%%%%%%
\subsubsection{Tree-Structured Shading Decomposition}


% Esse trabalho propõe uma abordagem para inferir uma representação de BRDF estruturada em árvore a partir de uma única imagem de entrada, usada para o sombreamento de objetos \cite{tree_decomposition}. Em vez de usar representações paramétricas, como é comum, é proposta uma abordagem que utiliza uma representação em árvore de \textit{shading}, combinando nós básicos e métodos para decompor o \textit{shading} da superfície do objeto, como partes albedo and reflectance,, representado na \autoref{fig_decomp}. Nessa imagem podemos observar o criação, que foi feita a partir de uma imagem de entrada e o trabalho dos atulres de compõe nessa arvore que criam nós que reprensentam parte do shading como a cor, representado pelo nó albedo, ou reflexão especular, chamado de Highlight na figura, ou a reflexão difusa chamada de Difuse na figura. Isso permite mairo facilidade na edição desso sahding. na imagem mostra uma troca no nó de albedo e todo o objeto pode ser renderizado com um nova cor (vista em azul), podemos também combinar com outras arvores geradas a partir da de composição de outra imagem. Essa edicação da arvore poderia ser feita com uma ferramenta visual de edição

Este trabalho propõe uma abordagem para inferir uma representação de BRDF  estruturada em forma de árvore, utilizando uma única imagem de entrada para sombreamento de objetos \cite{tree_decomposition}. Em vez de usar representações paramétricas, como é comum, a abordagem adota uma representação hierárquica, chamada de árvore de \textit{shading}, onde o sombreamento da superfície do objeto é decomposto em componentes, como albedo e refletância, conforme ilustrado na \autoref{fig_decomp}.

Nessa imagem, observa-se a árvore reconstruída a partir de uma imagem de entrada, onde os autores estruturam nós que representam diferentes propriedades do \textit{shading}. Por exemplo, o nó \texttt{Albedo} representa a cor da superfície, o nó \texttt{Highlight} descreve a reflexão especular, e o nó \texttt{Diffuse} corresponde à reflexão difusa. Essa estrutura facilita a manipulação e edição dos elementos do \textit{shading}.

A figura também demonstra uma aplicação prática dessa abordagem: a substituição do nó \texttt{Albedo} por uma nova cor (de vermelho para azul), resultando na renderização do objeto com a cor alterada. Além disso, é possível combinar árvores reconstruídas de outras imagens, substituindo uma subárvore ou utilizando o nó \texttt{Multiply}.

Essa abordagem aumenta a flexibilidade de edição, permitindo alterações mais intuitivas e rápidas na renderização de objetos que utilizam modelos BRDF.

\begin{figure}[H]
        \caption{\label{fig_decomp} \small Exemplo de decomposição de BRDFs em nós de uma árvore.}
        \begin{center}
            \includegraphics[scale=0.5]{./Imagens/tree-shading.png}
        \end{center}
        \legend{Fonte: \citeonline[]{radiometry_introduction}.}
\end{figure}




Assim como o nosso trabalho, esse artigo se concentra em facilitar o processo para usuários inexperientes, pois ambos visam fornecer ferramentas acessíveis para manipular representações de \textit{shading} sem exigir conhecimento avançado em programação. Esse artigo também emprega uma representação de BRDF em árvore, embora para um propósito diferente.
%%%%%%%%%%%%%%%%%%%%%%%%%%%%%%%%%%%%%%%%%%%%%%%
%%%%%%%%%%%%%%% REMOVED FOR NOW %%%%%%%%%%%%%%%
%%%%%%%%%%%%%%%%%%%%%%%%%%%%%%%%%%%%%%%%%%%%%%%


\subsubsection{A Real-Time Configurable Shader Based on Lookup Tables}


Esse trabalho propõe uma arquitetura de \textit{hardware} que permite cálculos de \textit{shading} por pixel em tempo real, utilizando \textit{lookup-tables} \cite{configurable}. Para isso, são projetados circuitos configuráveis baseados nessas tabelas, memórias de acesso aleatório (RAMs) e memórias somente leitura (ROMs). Vários circuitos base foram projetados para as operações mais comuns, como o cálculo do produto interno entre dois vetores e a rotação de um vetor por um ângulo. Um exemplo desses diagramas é representado na \autoref{fig_circuit}. Ademais, foi utilizada interpolação em um sistema de coordenadas polares em vez da interpolação vetorial convencional, com o objetivo de reduzir o tamanho dos circuitos e melhorar o desempenho.

% Vários circuitos base foram projetados para as operações mais comuns, como o cálculo do produto interno entre dois vetores e a rotação de um vetor por um ângulo. Um exemplo desses diagramas é representado na \autoref{fig_circuit}. Além disso, foi utilizada interpolação em um sistema de coordenadas polares, em vez da interpolação vetorial convencional, com o objetivo de reduzir o tamanho dos circuitos e melhorar o desempenho




\begin{figure}[H]
        \caption{\label{fig_circuit}\small Exemplo de circuito de produto interno entre vetores.}
        \begin{center}
            \includegraphics[scale=0.7]{./Imagens/rom-cos-lookup-table.png}
        \end{center}
  \legend{\small Fonte: \cite{configurable}.}
\end{figure}



Além disso, o circuito suporta diversas BRDFs, como Blinn-Phong, Cook-Torrance, Ward e modelos baseados em microfacetas, com tabelas específicas para cada modelo. O uso de tabelas de pesquisa permite a representação organizada da parametrização das BRDFs, tornando o processo de transformação de BRDF para \textit{shaders} mais acessível.

Este trabalho foi aceito por incluir o processo de tradução estruturada de BRDFs para uma sequência computável através de circuitos. Assim, o artigo demonstra como modularizar o cálculo das operações matemáticas contidas em BRDFs. A abordagem é útil para o nosso projeto, pois o processo de geração de \textit{shaders} a partir de uma árvore envolve linearizar as operações da BRDF em uma sequência correta, embora o propósito dessa tarefa em nosso projeto seja diferente do apresentado neste trabalho.


\section{Pesquisa por Repositórios Online}
Também foram analisados repositórios no GitHub e SourceForge, cada um com uma \textit{string} de busca específica. Os repositórios encontrados foram filtrados baseados em seus resumos. Caso não haja a menção da criação de um compilador ou não seja citada uma transformação de BRDF para outra estrutura, esse repositório foi excluído. O resultado se encontra na \autoref{tab-repo}.




% @IMPORTANT 'preciso incluir o trabalho de conclusão de tulasi, que tem um repositório associado. Você pode colocar em uma seção diferente aqui neste capítulo. também é importante mencionar no capítulo 1 que esse trabalho existe, e que a abordagem do seu difere nas técnicas usadas.


\begin{table}[H]
\ABNTEXfontereduzida
\caption[bases]{\small Resultados da pesquisa nos repositórios.}
\label{tab-repo}
\begin{tabular}{p{2.6cm}|p{6.0cm}|p{2.25cm}|p{3.40cm}}
  %\hline
   \textbf{Plataformas} & \textbf{Termos de Pesquisa}  & \textbf{Resultados}\\
   \hline
   GitHub
   &
   in:readme (GLSL AND BRDF AND  (compiler OR compilation) AND (shader OR shading))
   & 15
   \\ \hline
   SourceForge
   &
   compiler bdrf
   & 0
\end{tabular}
\end{table}




Após ler por completo os resumos dos repositórios do GitHub, é evidente que nenhum desses projetos é relacionado com o proposto neste trabalho. Apesar de comentarem sobre BRDFs, esses projetos não implementam compiladores, não fazem \textit{parsing} de equações de BRDFs e nem mesmo geram \textit{shaders} a partir de BRDFs.



%%%%%%%%%%%%%%%%
%% Metodologia %
%%%%%%%%%%%%%%%%
\chapter{Metodologia} \label{metodologia}

A metodologia para desenvolver o compilador proposto envolve uma abordagem prática. As suas principais etapas são: uma análise das informações pertinentes a BRDFs e compilação de \textit{shaders}; a exploração de técnicas existentes dentro do domínio; a especificação da linguagem subconjunto \LaTeX{} de entrada;  a implementação do compilador; a avaliação de seu desempenho por meio de experimentos de renderização.

Inicialmente, o método para realizar a análise e exploração das técnicas é descrito na \autoref{analise}. Em seguida, a especificação da linguagem de entrada e saída é definida na \autoref{especificacao-linguagem} @@@{link the grammar and explain}. Posteriormente, uma ideia de como o \textit{design} dos casos de teste devem ser elaborados para validar a correção e precisão do compilador é apresentado na \autoref{testes}. O método de implementação do compilador é detalhado na \autoref{compiladorimplementacao}. A \autoref{experimentos-renderizacao} planeja o método de avaliação dos experimentos de renderização quanto a qualidade visual dos \textit{shaders} compilados. Por fim, um plano de continuação é delineado, abordando as próximas etapas para completar o desenvolvimento do compilador proposto (\autoref{continuacao}).
Seguindo essa metodologia, a ferramenta proposta visa compilar efetivamente descrições de BRDF em \textit{shaders} GLSL.


\section{Análise e Técnicas} \label{analise}




O primeiro passo envolve a realização de uma análise detalhada das áreas relacionadas ao desenvolvimento da ferramenta proposta. Isso inclui a revisão da literatura (\autoref{revisao}) sobre BRDFs, linguagens de \textit{shaders}, \textit{design} de compiladores e técnicas de renderização gráfica. Além disso, envolve o estudo de ferramentas e bibliotecas pertinentes. Durante essa análise, foram estudados conceitos de radiometria para compreender tecnicamente as BRDFs. A principal fonte de informação sobre radiância e BRDFs foi o livro ``Physically Based Rendering: From Theory To Implementation'' \cite{pbr}. Esse livro foi importante para compreensão da equação de renderização (\autoref{eq-rendering-equation}). A leitura de exemplos práticos e leitura das código fonte da ferramente \autoref{fig-disney-tool} permitiu a familiarização com o desenvolvimento de BRDFs, fornecendo uma base sólida para a compreensão do mapeamento da equação para código, aspecto fundamental para o desenvolvimento do compilador proposto neste trabalho.@

Ademais, foram exploradas diversas técnicas para compilação, como o método de Pratt \textit{Parsing} para a construção de um compilador, conforme detalhado na \autoref{parser}, somado ao uso do conhecimento de recursividade e caminhada em arvóres para realizar a analise semantica e geração de código.


\section{Especificação da Linguagem}\label{especificacao-linguagem}

As especificações da linguagem de entrada e saída para o compilador são definidas. A linguagem de entrada é uma versão simplificada do \LaTeX{}, na qual as expressões matemáticas nos ambientes \texttt{equation} são suficientes para descrever BRDFs. O \LaTeX{}  é um sistema de composição amplamente utilizado para documentos matemáticos e científicos. O ambiente \texttt{equation} é especificamente projetado para exibir equações individuais. O \autoref{equation-latex} é um exemplo de código-fonte \LaTeX{}  usando o ambiente \texttt{equation}.


\begin{codigo}[H]
\caption{\small Código-fonte de função quadrática.}
\label{equation-latex}
\begin{lstlisting}
\begin{equation}
    g(x) = ax^2 + bx + c
\end{equation}
\end{lstlisting}
\end{codigo}




Este código representa a equação quadrática \( g(x) = ax^2 + bx + c \), onde \( a \), \( b \) e \( c \) são coeficientes. O código GLSL correspondente gerado a partir dessa equação pode ser o \autoref{cod-glsl-g}.

\begin{codigo}[H]
\caption{\small Código GLSL da função quadrática g.}
\label{cod-glsl-g}
\begin{lstlisting}
float g(float x, float a, float b, float c) {
    return a * x * x + b * x + c;
}
\end{lstlisting}
\end{codigo}

@@@
O ambiente de equações \LaTeX{} é amplo demais para o projeto, entre todas os as construções matemáticas representaveis por esse ambiente um subconjunto essencial para BRDFs deve ser escolhido. Ao analisar as principais BRDFs, como os ditos em \autoref{testes}, nota-se algumas contruções indispensaveis, essas devem ser reconhecidas e entendidas o suficiente para emitir código GLSL pelo nosso 
compilador, essas são enumeradas à seguir:

\label{subconjunto-latex-equantion} \begin{enumerate}
  \item principais funções trigonometricas \verb" \tan, \sin, \cos, \arctan, \arcsin, \arccos";

\item funcão raiz quadrada \verb"\sqrt" $\left(\sqrt{}\right)$;
\item funcão exponencial \verb"\sqrt" $\left(\sqrt{}\right)$;
\item funções utilitárias como $\max, \min$, ($\max, \min$);
\item definição de equações, por exemplo \verb"f = x" (rederizado fica $f = x$).
\item denifição de funções, por exemplo  \verb"f(x, y) = x^y" (rederizado fica $f(x, y) = x^y$) respectivamente;
\item constantes comuns como \verb"\pi" ($\pi$), \verb"\epsilon" ($\epsilon$);
\item constantes especificar \verb"\theta" ($\theta$), entre outros detalhados na @ref capitulo@;
\item indicador de vetor como \verb"\vec{}" (ex: $\vec{n}$);
\item identificadores aninhandos como \verb"f_{n_{i}}" ($f_{n_{i}}$).;
\item chamada de funções \verb"f(x+y)";
\item operadores de produto vetorial (\verb"x \times y", $x \times v$), soma ($+$), multiplicação ($x*y$ ou \verb"x \cdot y", $x \cdot y$), fração (\verb"\frac{x}{y}", $\frac{x}{y}$), divisão (\verb"{x}/{y}", ${x}/{y}$), power \verb"^", ($x^y$);

\end{enumerate}

Uma descrição completa dos simbolos reconhecidos são dados no @Desenvolvimento capitulo Lexer@. Construção completa da gramatica reconhecida pelo compilador é dado em @Capitulo Desenvolvimento Parser@. Note que do potno de vista do parser e lexer alguns simbolos são apenas reconhcidos, é citado que o compilador também precisa entenderlo, e para é preciso atribuir significado especifico à esses simbolos e construções, por exemplo $\omega_o$, que o angulo de saída da luz @@@ ou $f$ que é a brdf. Essa atribuição é feita em etapa de analise semantica, que vem após o parsing @ref@. Todos as convenções de simbolos que devem ser suportados pela linguagem é definido na tabela \autoref{tab-conventions} e seu significado pode ser encontrado

\begin{table}[h]
    \centering
    % \begin{tabular}{|c|l|}
    \begin{tabular}{cl}
        \hline
        \textbf{Símbolo} & \textbf{Descrição} \\
        \hline
        $\theta_i$ & Ângulo de elevação da direção da luz incidente \\
        \hline
        $\theta_o$ & Ângulo de elevação da direção da luz refletida \\
        \hline
        $\phi_i$ & Ângulo azimutal da direção da luz incidente \\
        \hline
        $\phi_o$ & Ângulo azimutal da direção da luz refletida \\
        \hline
        $\omega_i$ & Direção da luz incidente  \\
        \hline
        $\omega_o$ & Direção da luz refletida  \\
        \hline
        $f$ & BRDF de referência \\
        \hline
        $\vec{n}$ & Vetor normal à superfície \\
        \hline
        $\vec{h}$ & Vetor do meio entre $\omega_o$ e $\omega_i$ \\
        \hline
        $\theta_h$ & Ângulo entre $\vec{n}$ e $\vec{h}$ \\
        \hline
        $\theta_d$ & Ângulo entre $\omega_i$ e $\vec{h}$ \\
        \hline
    \end{tabular}
    % \caption{Tabela de simbolos e seu descrições}
    \label{tab-conventions}
\end{table}
%
%
\section{Design de Casos de Teste} \label{testes}
%
%
Os casos de teste são essenciais para validar a precisão e correção do processo de tradução do compilador. Eles estabelecem uma correspondência entre as equações \LaTeX{} de entrada, que descrevem as BRDFs, e o código de \textit{shader} GLSL esperado como saída. Um exemplo específico que demonstra a eficácia do compilador pode ser construído com a BRDF de Cook-Torrance. Sua função, \texttt{cook\_torrance}, é representada pela \autoref{eq-cook-torrance} (seu código-fonte está definido no \autoref{cod-input-latex}), onde \(D\) é a função de distribuição normal, \(G\) é a função de sombreamento geométrico e \(F\) é a função de Fresnel.

Embora as funções \(D\), \(G\), \(F\) não tenham sido definidas explicitamente, é importante ressaltar que, caso essas funções fossem definidas na equação \LaTeX{}, elas também devem ser definidas no \autoref{cod-glsl-esperado}, GLSL esperado de saída. Vale resaltar que nessa sessão de metodologia estamos dandos uma versão simplificada  de como o design de casos de teste ocorre para auxiliar entendimento. Na prática, unidades, como $\rho_d$, e funções, como $D,G$ e $F$, devem estar definidas. Casos de teste completos estão disponiveis no \autoref{resultados}.


Além disso, algumas variáveis, como a normal representada por \( n \), seriam passadas como entrada no \textit{shader} de fragmentos ou declaradas como variáveis uniformes, portanto não estão definidas explicitamente na função \texttt{cook\_torrance} no \autoref{cod-glsl-esperado}; elas são variáveis implícitas. Todas as variáveis implicitas se encontram na \autoref{tabela-variaveis}. Inicialmente, o foco é definir casos de teste para avaliar apenas a geração das operações e precedências. No entanto, é importante considerar que, posteriormente, o GLSL não deverá apenas gerar a função BRDF, mas sim o \textit{shader} completo, incluindo as variáveis uniformes e a passagem da cor calculada para as próximas etapas do \textit{pipeline} gráfico.


\begin{equation} \label{eq-cook-torrance}
  \text{cook\_torrance}(\omega_i, \omega_o) = \frac{D(h)F(\omega_i, h)G(\omega_i, \omega_o, h)}{4(\omega_i \cdot n)(\omega_o \cdot n)}
\end{equation}


\begin{codigo}[H]
\caption{\small Entrada em \LaTeX\  (Cook-Torrance BRDF).}
\label{cod-input-latex}
\begin{lstlisting}
  \text{cook\_torrance}(\omega_i, \omega_o)
      = \frac{D(h)F(\omega_i, h)G(\omega_i, \omega_o, h)}{4(\omega_i \cdot n)(\omega_o \cdot n)}
\end{lstlisting}
\end{codigo}


\begin{codigo}[H]
\caption{\small Saída em GLSL esperada (Cook-Torrance BRDF).}
\label{cod-glsl-esperado}
\begin{lstlisting}[language=C]
vec3 cook_torrance(vec3 wi, vec3 wo) {
    float D_RESULT = D(h);
    vec3  F_RESULT = F(wi, wo);
    float G_RESULT = G(wi, wo, h);
    float denominador = 4.0 * dot(n, wi) * dot(n, wo);
    return D_RESULT * F_RESULT * G_RESULT / denominador;
}
\end{lstlisting}
\end{codigo}


\section{Implementação do Compilador} \label{compiladorimplementacao}
% \begin{enumerate}
%
% Definir símbolos pré-definidos como constantes matemáticas e outras quantidades.
% Implementar o processo de geração de código GLSL.
%
% Expandir os casos de teste para cobrir uma ampla gama de BRDFs. Testar o código gerado quanto à correção, tanto em código quanto em visualização e corrigir se necessário.


Este trabalho envolve várias tarefas-chave destinadas a completar o desenvolvimento do compilador proposto para converter equações \LaTeX{}  que descrevem BRDFs em código de \textit{shader} GLSL. As tarefas incluem: Criar um \textit{lexer} e \textit{parser} para aceitar equações \LaTeX{}; testar o \textit{lexer} para garantir o reconhecimento correto dos \textit{tokens}; testar o \textit{parser} para garantir que a árvore sintática está com precedência correta; definir símbolos predefinidos e constantes matemáticas; implementar o processo de geração de código GLSL usando a árvore sintática com o padrão de \textit{design} visitante (\textit{Visitor}); definir os casos de teste para cobrir uma certa variedade de BRDFs; testar o código gerado quanto à correção, incluindo as visualizações das BRDFs em algumas cenas.

A implementação do compilador é realizada utilizando a linguagem de programação Odin, conhecida por ser uma linguagem de propósito geral com foco em programação orientada a dados. Sua escolha se deve à sua capacidade de oferecer controle de baixo nível e a sua adequação para o desenvolvimento de sistemas complexos. Além disso, nenhuma biblioteca externa foi utilizada, sendo usada apenas as bibliotecas padrões básicas que acompanham a instalação da linguagem.

Técnicas de análise recursiva são utilizadas, especificamente o Pratt \textit{Parsing}. Inicialmente, o \textit{lexer} e o \textit{parser} foram implementados para o subconjunto (\autoref{subconjunto-latex-equantion}) linguagem \LaTeX{} comentado em \autoref{}. Para garantir que os fundamentos do compilador estejam funcionais, considerando precedência totalmente testada para a árvore sintática, foram criados o pacote \texttt{walker}, que abstrai uma maneira de andar pela AST e valida algo, esse é usado para duas coisas, uma é para adicionar parenteses expoicitando a ordem de operação, outro é recursivamente inferir os tipos de cada expressão (nós que representam valores) presentes na AST. 
Também, é necessario cirar um passagem de analisise semantica, pacote chamado de "checker" onde iremos anotar a AST com todos os campos relevantes como tipos (função com seu dominio e contradominio, vetor real e sua dimensão, ou número $\in \mathbb{R} $). Por ultimo, já com o AST anotadas com outras informaçoes, realizamos através do pacote "emitter" a geração de cógido glsl, pronto para ser carregado e redenrizado pela ferramenta \autoref{disney-brdf-tool}.

\section{Experimentos de Renderização} \label{experimentos-renderizacao}


Por fim, experimentos de renderização são realizados usando os \textit{shaders} gerados pelo compilador. Isso permite a avaliação do desempenho e da qualidade visual das imagens renderizadas produzidas pelos \textit{shaders} compilados. A plataforma escolhida para os testes é a ferramenta \label{disney-brdf-tool} Disney BRDF \footnote{\url{https://github.com/wdas/brdf}}, compilada localmente para modificar e adicionar outros
\textit{shaders}.


Essa ferramenta é composta por um renderizador e uma interface que permite ajustar parâmetros de BRDFs através de controles deslizantes em tempo real, fornecendo uma visualização interativa do efeito das mudanças nos parâmetros que afetam a aparência do objeto renderizado, como ilustrado na \autoref{fig-disney-tool}. O código que informa à ferramenta qual a BRDF a ser renderizada e seus possíveis parâmetros pode ser visto na \autoref{fig-disney-code}. Esse código possui um formato específico, onde se encontram algumas seções. Existe a seção para código GLSL e outra seção delimitada por \texttt{::begin parameters} e \texttt{::end parameters}, na qual podemos definir os parâmetros que se tornam constantes dessa BRDF. O nosso compilador gera shaders nesse formato.



\begin{figure}[htb]
        \caption{\label{fig-disney-tool} \small Ferramenta de visualização de BRDFs da Disney.}
        \begin{center}
            \includegraphics[scale=0.65]{./Imagens/disney-brdf-tool-original.png}
        \end{center}
  \legend{ \small Fonte: autor.}
\end{figure}


\begin{figure}[h]
        \caption{\label{fig-disney-code} \small O código GLSL com sintaxe extra para definir parâmetros.}
        \begin{center}
            \includegraphics[scale=0.7]{./Imagens/disney-brdf-code.png}
        \end{center}
\end{figure}




%%%%%%%%%%%%%%%%%%%%%
%% Desenvolvimento %%
%%%%%%%%%%%%%%%%%%%%%

\chapter{Desenvolvimento}

Este capítulo aborda o processo de desenvolvimento do compilador proposto como um todo na linguagem Odin.
Cada etapa é encapsulado em um pacote, representado em \autoref{estrutura-de-pacotes} diferente \texttt{lexer} corresponde à tokenização da linguagem, \texttt{parser} corresponde à análise sintatica, \texttt{walker} contém funções que auxliam tanto a visualizar o resultado da analise sintatica, a AST, quando na checacgem de tipos da analise sintatixe, pois ambas dependem de fazer a transver@@@ da arvore em ordem,
\texttt{}. A arquitetura da pipeline para o compilador é delineado na \autoref{fig-estrutura-geral-compilador}.
O repositório pode ser encontrado em \url{https://github.com/evertonse/@@@}

\begin{figure}[H]
  \caption{\label{estrutura-de-pacotes} \small Estrutura de Pacotes do Compilador.}
  \begin{center}
    \includegraphics[scale=0.5]{./Imagens/package-structure.png}
  \end{center}
\end{figure}

\begin{figure}[H]
  \caption{\label{fig-estrutura-geral-compilador} \small Estrutura de geral da arquitetura da pipeline do Compilador.}
  \begin{center}
    \includegraphics[scale=0.62]{./Imagens/estutura-geral-do-projeto.png}
  \end{center}
\end{figure}

Os resultados do desenvolvimento desse compilador pode ser encontrado em \autoref{resultados}.
A especificação da linguagem pode ser encontrada no \autoref{@@@}. Nesse apendice temos a gramática @@@ para tokens e gramatica que gera AST, a tabela de precedencia que é necessário para desambiguar a linguamge encontra-se em \autoref{@@@}.
Os exemplos de BRDFs mostrados no \autoref{resultados} foram usados como base para verificação da corretude da gramática durante seu desenvolvimento.

Nesta construção do compilador, foi feita análises léxica manualmente através de loops mudando o estado atual para separada a entrada, que seria um string do arquivo inteiro, para uma lista de tokens. Já a análise sintática usamos a gramática livre de contexto \autoref{@@@} para nos guiar, somado a tabela de precedencia para aplicamo o Pratt Parsing que resulta em uma AST.

% Why pratt is better:
% exemplo de como uma lingaugem um parser LALR(1) poderia fazer o encode na propria definição
%
% MultiplicativeExpr = MultiplicativeExpr * AddExpr
% AddExpr = AddExpr * Expr
%
% no prat parsing a regra de derivação é a mesma , adicionado de uma tabela
% Expr = Expr (*|+) Expr

@{Add develpment preview of wahts to come}
\subsection{Desenvolvimento}

Primeiro foi criado o analisdor lexico, um pacote inteiro para esse analisador na linguagem odin.
O trabalho desse analisdor é transform um array de caracteres que é a entrada e retonar uma sequencia de tokens.
Cada token tem um tipo ( chamado de kind em código), um valor, reservado para numeros, texto, e posição, que é usado para reportar erros.

Cada tipo (\textit{kind}) é cado pela enumeração \textbf{Token\_Kind}, essa encoda todos os possiveis tipos comomo dito @{cite previous chapter talking about the entry language}.
Esses token podem ser: comentarios gerados por uma linha que comece com \%, números, identificadores que são qualquer sequencia de caractheres que não seja palavras especiais, simbolo de igual ('='), simbolos de operadores ('\^', '*') .. bla, funções espciais ($\max$, $\sin$, $\arccos$, etc ...)


\begin{codigo}[H]
  \caption{\small } \label{}
\begin{lstlisting}
Token :: struct {
    kind: Token\_Kind,
    val: union{i64,f64},
    text: string,
    pos:  Position,
}

\end{lstlisting}
\end{codigo}

O processo de lexing feito com um loop, simulado a uma maquina de estados, que decide qual token deve ser criado em sequencia ao olhar o caractere atual e o estado.

Estados estão relacionados ao processo de identificar estados pode estar relacionados a identificar palavras.

É  adiante, por exemplo se encontrar um um '1' sabemos que é um numero, podendo ter um '.' para indicar decimal, então utilizamos 
uma subrotina para identificar esse continuar processando o "input" até o token de numeros ter sido totalmente coletado, se no meio de processar um número um caractere não esperado for encontrado, reportamos um error léxico, exemplos pode ser visto na imagem @{Mostre Imagem com Erro}
O mais simples são tokens de um caractere '\^', '*', '/', '+', '-', '?', '=', '~', '(', ')', ',', ':', '{', '}', '\_', cara um tem um proposito especifico na analise lexica. Na etapa lexica nos preopados apenas em separar nos tokens de maneira cega ao seu significado.


Todo identificador, especial ou não é processado da mesma maneira, é verificado se o caractere atual é um letra ou um '\\', isso indica o começo 
de um identificador. Depois de de

A gramatica dos tokens é regular e será representada abaixio:


Vale ressaltar que nesse moment é criado uma tabela que mapeia cada numero de linha à um string dessa mesma linha, para reportar error, printando a linha do problema mais a linha anterior e posterior para.
Tem um token que é especial que indica o começo de um ambiente `\\begin{equation}`, qualquer comentario antes de apaerecer esse token é ignorado, isso é para poder dar como entrada ao compilador um documento inteir ocontendo begin document e ainda funciojnar



\subsection{Analise Semantica}

\subsubsection{Tabela de Symbolos}
Symbolos podem ser declarados fora de ordem, ciramos um grafo de dependencias e fazemos um orednação topologica de dependencia.
Isso é póis, ao detectar analisa um certo symbolo queremos dizer se está usando simbolos não definidos, para isso precisamos definifir todos os simbolos glocais que estão no escopo visivel à todos, isso incluisimbolos pre-definidos pela linguagem, (ver tabela @{tabela de simbolos predefinidos}, para isso precisamos primeiro primeiro coletar todos esses e analisar priomeiros oq que dependen de ninguem, e medida que tão
. Também pode ocorrer dependecia circular sem reoslução e nesse caso reportamos um erro, nesse caso precisamos. @{true? ciruclar dependency?}

\subsubsection{Inferencia de Tipos}

\subsection{SVG da arvore abstrata com inferencia de tipos}
Para identificar possiveis erros de ordenação algumas medidas foram feitas para auxiliar, como a geração de uma imagem da 
em SVG da arvore sintatica, já com inferencia de tipos




%%%%%%%%%%%%%%%%
%% Resultados %%
%%%%%%%%%%%%%%%%
\chapter{Resultados}
\label{chapter.resultados}


Este capítulo apresenta os resultados dos experimentos com diferentes BRDFs, servindo como validação e visualização da capacidade do compilador desenvolvido. A escolha de cada BRDF foi direcionada para explorar expressões matemáticas com diversos níveis de complexidade, aspectos importante para testar a capacidade do compilador desenvolvido neste projeto.

Os experimentos seguem uma metodologia padronizada. Inicialmente, apresenta-se a BRDF do experimento, incluindo sua referência, quando relevante e uma breve explicação conceitual. Em seguida, demonstra-se o código-fonte que descreve a BRDF em \texttt{EquationLang}, acompanhado de sua representação em PDF \LaTeX{}. Utilizando o compilador desenvolvido, o código-fonte é traduzido para linguagem de \textit{shading} GLSL e carregado na ferramenta Disney BRDF Explorer.

A análise inclui gráficos 3D e 2D da distribuição de reflexão especular e difusa da BRDF. Para demonstrar a eficácia do código GLSL gerado, são renderizados três objetos tridimensionais utilizando técnicas de \textit{ray tracing} fornecidas pela ferramenta Disney. Os objetos possuem ângulos em coordenadas polares fixas, com condições padronizadas de iluminação: ângulos de elevação ($\theta_i$) e azimutal ($\phi_i$) da luz incidente predefinidos em $33,8941$ e $145,826$ respectivamente; gamma fixado em $2,112$ e exposição em $-1,248$.
% Adicionalmente, apresenta-se o efeito da BRDF em uma esfera com renderização projetiva padrão.

O \textit{plot}\footnote{O termo \textit{plot} é comumente usado em ferramentas de visualização e análise para se referir a gráficos ou representações visuais de dados. Aqui, refere-se a representações 2D (polar) ou 3D usadas para ilustrar os componentes especular e difuso de cada canal de cor da BRDF.} 3D de BRDFs na ferramenta Disney Explorer oferece uma visualização que fixa uma direção de luz incidente ($\omega_i$) e coleta valores da direções de visualização ($\omega_o$) em um hemisfério como entrada para BRDF. Para cada direção de visualização, renderiza-se um primitivo proporcional ao valor da função BRDF.

O \textit{plot} polar, por sua vez, representa um corte bidimensional, fixando a direção de luz incidente ($\omega_i$) e o ângulo azimutal de saída ($\phi_o$), variando apenas o ângulo polar de saída ($\theta_o$), similar ao mostrado nas figuras da \autoref{brdfmodels}. Cada ponto representa o valor médio das componentes da BRDF, visualizando o comportamento da refletância em diferentes ângulos de observação. Em alguns casos, fatores logarítmicos são utilizados para melhor visualização do gráfico.

É importante observar que os gráficos polares e 3D representam simultaneamente os três canais de cores, como na \autoref{fig-ashikhmin-shirley-alternative-plots}, podendo haver sobreposição entre vermelho, azul e verde na visualização, já que a distribuição de cada canal pode ser idêntica em um dado experimento.

% Embora os experimentos contenha uma explicação sobre a BRDF, o foco principal permanece na representação fidedigna das equações em GLSL provida pelo compilador. 
% E Ainda, toda explicação que esteja fora do ambiente equatio é ignorada pelo compilaodr, então realmente é apensar para ilustrar a compilação de um arquivo por completo, por isso incluimos as explicações rudimentares ( muitos vezes em inles) para demonstrar isso. Recomenda-se observar rapidamente o código gerado para compreender sua estrutura, reconhecendo que o código GLSL gerado por computador não é tão legível quanto código \textit{shading} escrito manualmente.

Embora os experimentos contenham uma explicação sobre a BRDF, o foco principal permanece na representação fidedigna das equações em GLSL fornecida pelo compilador. Além disso, qualquer explicação presente no arquivo de entrada (\verb".tex") que esteja fora do ambiente \texttt{equation} é ignorada pelo compilador, sendo incluída apenas para ilustrar a compilação de um arquivo completo. Por essa razão, inserimos explicações rudimentares (muitas vezes em inglês) para demonstrar esse aspecto. Recomenda-se observar rapidamente o código gerado para compreender sua estrutura, reconhecendo que o código GLSL gerado pelo computador não é tão legível quanto o código \textit{shading} escrito manualmente.

Alguns experimentos exploram múltiplas formas de expressar a mesma BRDF, não apenas com parâmetros distintos, mas também com expressões matemáticas alternativas. Para facilitar a navegação, a \autoref{table-experiments} é disponibilizada para acesso rápido às imagens e códigos dos experimentos.



\begin{table}[H]
\centering
\begin{tabular}{|l|c|c|c|c|c|}
\hline
    \textbf{Experimento} & \textbf{Seção}                                            & \textbf{Equações}                                                & \textbf{Objetos 3D}                                       & \textbf{\textit{Plots}}                                  & \textbf{GLSL}  \\ \hline
    Blinn-Phong          &\autoref{section-experiment-blinn-phong}                   & \autoref{fig-blinn-phong-eqlang-latex}                           & \autoref{fig-blinn-phong-eqlang}                          & \autoref{fig-blinn-phong-plots}                          &                  \autoref{cod-blinn-phong-glsl-pt-1}              \\ \hline
    Cook-Torrance        &\autoref{sec:cook-torrance}                 & \autoref{fig-cook-torrance-eqlang-latex}                         & \autoref{fig-cook-torrance-eqlang}                        & \autoref{fig-cook-torrance-plots}                        &             \autoref{cod-cook-torrance-glsl-pt-1}              \\ \hline
    Ward                 &\autoref{section-experiment-ward}                          & \autoref{fig-ward-eqlang-latex}                                  & \autoref{fig-ward-objetcs}                                & \autoref{fig-ward-plots}                                 &                     \autoref{cod-ward-glsl-pt-1}               \\ \hline
    Ashikhmin-Shirley    &\autoref{sec:ashikhmin-shirley}             & \autoref{fig-ashikhmin-shirley-close-to-original-eqlang-latex}   & \autoref{fig-ashikhmin-shirley-close-to-original-eqlang}  & \autoref{fig-ashikhmin-shirley-close-to-original-plots}  &                  \autoref{cod-ashikhmin-shirley-close-to-original-glsl-pt-1}              \\ \hline
    Oren-Nayar           &\autoref{section-experiment-oren-nayar}                    & \autoref{fig-oren-nayar-eqlang-latex}                            & \autoref{fig-oren-nayar-eqlang}                           & \autoref{fig-oren-nayar-plots}                           &                \autoref{cod-oren-nayar-glsl-pt-1}              \\ \hline
    Ashikhmin-Shirley$_2$&\autoref{section-experiment-ashikhmin-shirley-alternative} & \autoref{fig-ashikhmin-shirley-alternative-eqlang-latex}         & \autoref{fig-ashikhmin-shirley-alternative-eqlang}        & \autoref{fig-ashikhmin-shirley-alternative-plots}        &                  \autoref{cod-ashikhmin-shirley-alternative-glsl-pt-1}              \\ \hline
    Cook-torrance$_2$    &\autoref{section-experiment-cook-torrance-alternative}     & \autoref{fig-cook-torrance-alternative-eqlang-latex}             & \autoref{fig-cook-torrance-alternative-eqlang}            & \autoref{fig-cook-torrance-alternative-plots}            & \autoref{cod-cook-torrance-alternative-glsl-pt-1}              \\ \hline
    Dür                  &\autoref{section-experiment-duer}                          & \autoref{fig-duer-eqlang-latex}                                  & \autoref{fig-duer-eqlang}                                 & \autoref{fig-duer-plots}                                 &                      \autoref{cod-duer-glsl-pt-1}              \\ \hline
    Edwards-2006         &\autoref{section-experiment-edwards-2006}                  & \autoref{fig-edwards-2006-eqlang-latex}                          & \autoref{fig-edwards-2006-eqlang}                         & \autoref{fig-edwards-2006-plots}                         &              \autoref{cod-edwards-2006-glsl-pt-1}              \\ \hline
    Kajiya-Kay-1989$_*$  &\autoref{section-experiment-kajiya}                        & \autoref{fig-kajiya-eqlang-latex}                                & \autoref{fig-kajiya-objects}                              & \autoref{fig-kajiya-plots}                               &                  \autoref{cod-kajiya-glsl-pt-1}              \\ \hline
    Minnaert             &\autoref{section-experiment-minnaert}                      & \autoref{fig-minnaert-eqlang-latex}                              & \autoref{fig-minnaert-eqlang}                             & \autoref{fig-minnaert-plots}                             &                  \autoref{cod-minnaert-glsl-pt-1}              \\ \hline
\end{tabular}
\caption{Tabela dos Experimentos.}
\label{table-experiments}
\end{table}


%%%
Concluímos que os experimentos realizados apresentaram resultados satisfatórios. O compilador desenvolvido demonstra flexibilidade ao capturar as nuances das diferentes BRDFs, inclusive em materiais com estruturas complexas. O sistema permite diversas parametrizações e equações alternativas para representar os comportamentos da superfície.

Os resultados obtidos não apenas validam a metodologia adotada, mas também abrem perspectivas para futuras extensões e refinamentos da ferramenta. Após o último experimento, seguimos diretamente para o capítulo de conclusão (\autoref{chapter-conclusion}), onde são discutidas as possíveis direções para a continuidade deste trabalho.


\section{Experimento BRDF Blinn-Phong}

Neste experimento usamos uma BRDF com um método simplificado de cálculo de reflexão especular, presente na \autoref{fig-blinn-phong-eqlang-latex}, introduzido por Blinn-Phong \cite{blinn1977models}. O \autoref{cod-blinn-phong-eqlang}, escrito \texttt{EquationLang}, é a entrada para o compilador. O \autoref{cod-blinn-phong-glsl-pt-1} e \autoref{cod-blinn-phong-glsl-pt-2} são a sáida em GLSL. A renderização dos objetos 3D estão em \autoref{fig-blinn-phong-eqlang}. Por fim, os plots estão na \autoref{fig-blinn-phong-plots}.

%%%%%%%%%%%%%%%%%%%%%%%%%%%%%%%%%%%%%%%%%%%%%%%%%
\subsection{Representação em documento \LaTeX{}}
%%%%%%%%%%%%%%%%%%%%%%%%%%%%%%%%%%%%%%%%%%%%%%%%%
\begin{figure}[H]
    \caption{\label{fig-blinn-phong-eqlang-latex} \small Equações da BRDF do experimento blinn-phong-Kay em documento \LaTeX{}.}
    \begin{center}
        % \includegraphics[scale=1.1,width=\textwidth]{./Imagens/brdfs/aniso.pdf}
        \includegraphics[scale=0.92]{./Imagens/brdfs/blinn-phong.pdf}
    \end{center}
\end{figure}

%%%%%%%%%%%%%%%%%%%%%%%%%%%%%%%%%%%%%%%%%%%%%%%%%
\subsection{Visualização do Resultado}
%%%%%%%%%%%%%%%%%%%%%%%%%%%%%%%%%%%%%%%%%%%%%%%%%

\begin{figure}[H]
    \caption{\small{Distribuição de Reflexão Especular e Difusa da BRDF}}\label{fig-blinn-phong-plots}
\minipage{0.48\textwidth}
    \vspace{42px}
  \includegraphics[width=\linewidth]{./Imagens/brdfs/blinn-phong-3D-plot}
    % \caption{\small{(a)}}\label{fig:awesome_image1}
    % \vspace{0.1px}
    \legend{ \small (a) 3D \textit{plot}}
\endminipage\hfill
\minipage{0.48\textwidth}
  \includegraphics[width=\linewidth]{./Imagens/brdfs/blinn-phong-polar-plot-log.png}
    \legend{ \small (b) \textit{Polar plot}}
    % \caption{\small{(b)}}\label{fig:awesome_image1}
\endminipage\hfill
\end{figure}

\begin{figure}[H]
    \caption{\small{Objetos 3D renderizados por este experimento}}\label{fig-blinn-phong-eqlang}
\minipage{0.32\textwidth}
  \includegraphics[width=\linewidth]{./Imagens/brdfs/blinn-phong-teapot.png}
    \legend{ \small (a) \textit{Teapot}}
\endminipage\hfill
\minipage{0.32\textwidth}
  \includegraphics[width=\linewidth]{./Imagens/brdfs/blinn-phong-dragon.png}
    \legend{ \small (b) Dragão de Stanford}
\endminipage\hfill
\minipage{0.32\textwidth}%
  \includegraphics[width=\linewidth]{./Imagens/brdfs/blinn-phong-goblin.png}
    \legend{ \small (c) Goblin}
\endminipage
\end{figure}

%%%%%%%%%%%%%%%%%%%%%%%%%%%%%%%%%%%%%%%%%%%%%%%%%
\subsection{Código GLSL Gerado}
%%%%%%%%%%%%%%%%%%%%%%%%%%%%%%%%%%%%%%%%%%%%%%%%%
\begin{codigo}[H]
    \caption{\small Saida do compilador, código GLSL da BRDF deste experimento (parte 1). }
    \label{cod-blinn-phong-glsl-pt-1}
\begin{lstlisting}[language=C, inputencoding=utf8]
analytic ::begin parameters
#[type][name][min val][max val][default val]
::end parameters
::begin shader
//////////// START OF BUILTINS DECLARTION ////////////
vec3 var_0_vec_h;
vec3 var_3_vec_n;
float var_10_theta_h;
float var_11_theta_d;
float var_1_pi;
float var_2_epsilon;
vec3 var_4_vec_omega_i;
float var_5_theta_i;
float var_6_phi_i;
vec3 var_7_vec_omega_o;
float var_8_theta_o;
float var_9_phi_o;
//////////// END OF BUILTINS DECLARTION ////////////

//////////// START OF USER DECLARED ////////////
vec3 var_12_rho_s;
float var_13_n;
vec3 var_14_rho_d;
vec3 var_15_f;
//////////// END OF USER DECLARED ////////////
\end{lstlisting}
\end{codigo}

\begin{codigo}[H]
    \caption{\small Saida do compilador, código GLSL da BRDF deste experimento  (parte 2). }
    \label{cod-blinn-phong-glsl-pt-2}
\begin{lstlisting}[language=C, inputencoding=utf8]
vec3 BRDF(vec3 L, vec3 V, vec3 N, vec3 X, vec3 Y) {

  //////////// START OF BUILTINS INITIALIZATION ////////////
  var_0_vec_h = normalize(L + V);
  var_3_vec_n = normalize(N);
  var_1_pi = 3.141592653589793;
  var_2_epsilon = 1.192092896e-07;
  var_4_vec_omega_i = L;
  var_5_theta_i = atan(var_4_vec_omega_i.y, var_4_vec_omega_i.x);
  var_6_phi_i = atan(sqrt(var_4_vec_omega_i.y * var_4_vec_omega_i.y +
                          var_4_vec_omega_i.x * var_4_vec_omega_i.x),
                     var_4_vec_omega_i.z);
  var_7_vec_omega_o = V;
  var_8_theta_o = atan(var_7_vec_omega_o.y, var_7_vec_omega_o.x);
  var_9_phi_o = atan(sqrt(var_7_vec_omega_o.y * var_7_vec_omega_o.y +
                          var_7_vec_omega_o.x * var_7_vec_omega_o.x),
                     var_7_vec_omega_o.z);
  var_10_theta_h = acos(dot(var_0_vec_h, N));
  var_11_theta_d = acos(dot(var_0_vec_h, var_4_vec_omega_i));
  //////////// END OF BUILTINS INITIALIZATION ////////////
  var_12_rho_s = vec3(1.0, 0.0, 1.0);
  var_13_n = pow(2.0, 8.0);
  var_14_rho_d = vec3(0.0, 1.0, 1.0);
  var_15_f = ((var_14_rho_d / var_1_pi) +
              ((var_12_rho_s * ((var_13_n + 2.0) / (2.0 * var_1_pi))) *
               pow(cos(var_10_theta_h), var_13_n)));

  return vec3(var_15_f);
}
\end{lstlisting}
\end{codigo}

%%%%%%%%%%%%%%%%%%%%%%%%%%%%%%%%%%%%%%%%%%%%%%%%%
\subsection{Código Fonte em \texttt{EquationLang}}
%%%%%%%%%%%%%%%%%%%%%%%%%%%%%%%%%%%%%%%%%%%%%%%%%
\begin{codigo}[H]
    \caption{\small Código fonte da BRDF deste experimento (parte 1).}
    \label{cod-blinn-phong-eqlang}
\begin{lstlisting}[language=tex, frame=none, inputencoding=utf8]
\begin{document}

\begin{equation}
    \rho_{d} = \vec{0,1,1}
\end{equation}

\begin{equation}
    \rho_{s} = \vec{1,0,1}
\end{equation}

\begin{equation}
    n = +2^8
\end{equation}

\begin{equation}
f = \frac{\rho_{d}}{\pi} + \rho_{s} * \frac{n+2}{2*\pi} *
\cos{\theta_{h}}^{n}
\end{equation}
\end{lstlisting}
\end{codigo}

\section{Experimento BRDF cook-torrance}

As equações que descrevem esse experimento se encontram em \autoref{fig-cook-torrance-eqlang-latex}. O código fonte de entrada para o compilador está dividido em duas partes, parte 1 está no \autoref{cod-cook-torrance-eqlang} e a segunda parte está em \autoref{cod-cook-torrance-eqlang-pt2}. A renderização dos objetos 3D usando essa BRDF se encontra em \autoref{fig-cook-torrance-eqlang}. Usamos plot logaritmo

%%%%%%%%%%%%%%%%%%%%%%%%%%%%%%%%%%%%%%%%%%%%%%%%%
\subsection{Representação em documento \LaTeX{}}
%%%%%%%%%%%%%%%%%%%%%%%%%%%%%%%%%%%%%%%%%%%%%%%%%
\begin{figure}[H]
    \caption{\label{fig-cook-torrance-eqlang-latex} \small Equações da BRDF do experimento cook-torrance em documento \LaTeX{}.}
    \begin{center}
        \includegraphics[scale=0.92]{./Imagens/brdfs/cook-torrance.pdf}
    \end{center}
\end{figure}

%%%%%%%%%%%%%%%%%%%%%%%%%%%%%%%%%%%%%%%%%%%%%%%%%
\subsection{Visualização do Resultado}
%%%%%%%%%%%%%%%%%%%%%%%%%%%%%%%%%%%%%%%%%%%%%%%%%
\begin{figure}[H]
    \caption{\small{Distribuição de Reflexão Especular e Difusa da BRDF}}\label{fig-cook-torrance-eqlang}
\minipage{0.48\textwidth}
    \vspace{42px}
  \includegraphics[width=\linewidth]{./Imagens/brdfs/cook-torrance-3D-plot}
    % \vspace{0.1px}
    \legend{ \small (a) 3D \textit{plot}}
\endminipage\hfill
\minipage{0.48\textwidth}
  \includegraphics[width=\linewidth]{./Imagens/brdfs/cook-torrance-polar-plot-log.png}
    \legend{ \small (b) \textit{Polar plot}}
\endminipage\hfill
\end{figure}

\begin{figure}[H]
    \caption{\small{Objetos 3D renderizados por este experimento}}\label{fig-cook-torrance-eqlang}
\minipage{0.32\textwidth}
  \includegraphics[width=\linewidth]{./Imagens/brdfs/cook-torrance-teapot.png}
    \legend{ \small (a) \textit{Teapot}}
\endminipage\hfill
\minipage{0.32\textwidth}
  \includegraphics[width=\linewidth]{./Imagens/brdfs/cook-torrance-dragon.png}
    \legend{ \small (b) Dragão de Stanford}
\endminipage\hfill
\minipage{0.32\textwidth}%
  \includegraphics[width=\linewidth]{./Imagens/brdfs/cook-torrance-goblin.png}
    \legend{ \small (c) Goblin}
\endminipage
\end{figure}

%%%%%%%%%%%%%%%%%%%%%%%%%%%%%%%%%%%%%%%%%%%%%%%%%
\subsection{Código GLSL Gerado}
%%%%%%%%%%%%%%%%%%%%%%%%%%%%%%%%%%%%%%%%%%%%%%%%%
\begin{codigo}[H]
    \caption{\small Saida do compilador, código GLSL da BRDF deste experimento (parte 1). }
    \label{cod-cook-torrance-eqlang-declarations}
\begin{lstlisting}[language=C, inputencoding=utf8]
\end{lstlisting}
\end{codigo}

\begin{codigo}[H]
    \caption{\small Saida do compilador, código GLSL da BRDF deste experimento  (parte 2). }
    \label{cod-cook-torrance-eqlang}
\begin{lstlisting}[language=C, inputencoding=utf8]
\end{lstlisting}
\end{codigo}

%%%%%%%%%%%%%%%%%%%%%%%%%%%%%%%%%%%%%%%%%%%%%%%%%
\subsection{Código Fonte em \texttt{EquationLang}}
%%%%%%%%%%%%%%%%%%%%%%%%%%%%%%%%%%%%%%%%%%%%%%%%%
\begin{codigo}[H]
    \caption{\small Código fonte da BRDF deste experimento (parte 1).}
    \label{cod-cook-torrance-eqlang}
\begin{lstlisting}[language=tex, frame=none, inputencoding=utf8]
\end{lstlisting}
\end{codigo}

\section{Experimento BRDF Ward}

Este experimento é baseado na BRDF revisada segundo as notas de Walter \cite{walter2005notes}, que detalham o modelo de Ward. Suas equações podem ser vistas na \autoref{fig-ward-eqlang-latex}, enquanto o código em \texttt{EquationLang} está disponível no \autoref{cod-ward-eqlang-pt-1} (parte 1) e no \autoref{cod-ward-eqlang-pt-2} (parte 2). O código gerado pelo compilador é formado pelo \autoref{cod-ward-glsl-pt-1} e pelo \autoref{cod-ward-glsl-pt-2}. A renderização de objetos usando este modelo é ilustrada na \autoref{fig-ward-objetcs}, e os \textit{plots} da sua refletância estão na \autoref{fig-ward-plots}.
%%


%%%%%%%%%%%%%%%%%%%%%%%%%%%%%%%%%%%%%%%%%%%%%%%%%
\subsection{Representação em documento \LaTeX{}}
%%%%%%%%%%%%%%%%%%%%%%%%%%%%%%%%%%%%%%%%%%%%%%%%%
\begin{figure}[H]
    \caption{\label{fig-ward-eqlang-latex} \small Equações da BRDF do experimento Ward em documento \LaTeX{}.}
    \begin{center}
        \includegraphics[scale=0.92]{./Imagens/brdfs/ward.pdf}
    \end{center}
\end{figure}

%%%%%%%%%%%%%%%%%%%%%%%%%%%%%%%%%%%%%%%%%%%%%%%%%
\subsection{Código Fonte em \texttt{EquationLang}}
%%%%%%%%%%%%%%%%%%%%%%%%%%%%%%%%%%%%%%%%%%%%%%%%%
\begin{codigo}[H]
    \caption{\small Código fonte da BRDF deste experimento (parte 1 de 2).}
    \label{cod-ward-eqlang-pt-1}
\begin{lstlisting}[language=tex, frame=none, inputencoding=utf8]
Equations representing the Ward BRDF:
   \begin{equation}
      \text{normalize}(\vec{u}) = \frac{\vec{u}}{\sqrt{\vec{u} \cdot \vec{u}}}
   \end{equation}
1. Half vector:
   \begin{equation}
   \vec{H} = \text{normalize}(\vec{\omega_i} + \vec{\omega_o})
   \end{equation}
2. Tangent vector:
   \begin{equation}
   \vec{X} = \text{normalize}(\vec{0,1,0} \times \vec{n})
   \end{equation}
3. Bitangent vector:
   \begin{equation}
   \vec{Y} = \text{normalize}(\vec{n} \times \vec{X})
   \end{equation}
4. Roughness parameters:
   \begin{equation}
   \alpha_x = 0.4
   \end{equation}
   \begin{equation}
   \alpha_y = 0.2
   \end{equation}
\end{lstlisting}
\end{codigo}

\begin{codigo}[H]
    \caption{\small Código fonte da BRDF deste experimento (parte 2 de 2).}
    \label{cod-ward-eqlang-pt-2}
\begin{lstlisting}[language=tex, frame=none, inputencoding=utf8]
5. Exponent calculation:
   \begin{equation}
   \text{exponent} = -\frac{
       \frac{\vec{H} \cdot \vec{X}}{\alpha_x}^2 +
       \frac{\vec{H} \cdot \vec{Y}}{\alpha_y}^2
   }{(\vec{H} \cdot \vec{n})^2}
   \end{equation}
6. Specular term:
7. And Exponent:
   \begin{equation}
   \text{spec} = \frac{1}{4*\pi * \alpha_x *\alpha_y *\sqrt{(\vec{\omega_i} \cdot \vec{n}) * (\vec{\omega_o} \cdot \vec{n})}}
      \cdot \exp{( \text{exponent} )}
   \end{equation}
8. Color parameters
   \begin{equation}
   \vec{C_s} = \vec{1, 1, 1}
   \end{equation}
   \begin{equation}
   \vec{C_d} = \vec{ 1, 1, 1 }
   \end{equation}
9. Final BRDF:
   \begin{equation}
   f = \frac{\vec{C_d}}{\pi} + \vec{C_s} \cdot \text{spec}
   \end{equation}
\end{lstlisting}
\end{codigo}

%%%%%%%%%%%%%%%%%%%%%%%%%%%%%%%%%%%%%%%%%%%%%%%%%
\subsection{Código GLSL Gerado}
%%%%%%%%%%%%%%%%%%%%%%%%%%%%%%%%%%%%%%%%%%%%%%%%%
\begin{codigo}[H]
    \caption{\small Saída do compilador: código GLSL da BRDF deste experimento (parte 1 de 2).}
    \label{cod-ward-glsl-pt-1}
\begin{lstlisting}[language=C, inputencoding=utf8]
analytic
::begin parameters
#[type][name][min val][max val][default val]
::end parameters
::begin shader
//////////// START OF BUILTINS DECLARTION ////////////
vec3 var_0_vec_h;
vec3 var_3_vec_n;
float var_10_theta_h;
float var_11_theta_d;
float var_1_pi;
float var_2_epsilon;
vec3 var_4_vec_omega_i;
float var_5_theta_i;
float var_6_phi_i;
vec3 var_7_vec_omega_o;
float var_8_theta_o;
float var_9_phi_o;
//////////// END OF BUILTINS DECLARTION ////////////
//////////// START OF USER DECLARED ////////////
vec3 var_14_vec_X;
vec3 var_15_vec_Y;
float var_16_alpha_x;
float var_17_alpha_y;
vec3 var_18_vec_C_d;
vec3 var_19_vec_C_s;
vec3 var_20_vec_H;
float var_21_text_exponent;
float var_22_text_spec;
vec3 var_23_f;
//////////// END OF USER DECLARED ////////////
//////////// START FUNCTIONS DECLARATIONS ////////////
vec3 var_12_text_normalize(vec3 var_13_vec_u) {
  return (var_13_vec_u / sqrt(dot(var_13_vec_u, var_13_vec_u)));
}
//////////// END FUNCTIONS DECLARATIONS ////////////
\end{lstlisting}
\end{codigo}

\begin{codigo}[H]
    \caption{\small Saída do compilador: código GLSL da BRDF deste experimento (parte 2 de 2).}
\label{cod-ward-glsl-pt-2}
\begin{lstlisting}[language=C, inputencoding=utf8]
vec3 BRDF(vec3 L, vec3 V, vec3 N, vec3 X, vec3 Y) {
  //////////// START OF BUILTINS INITIALIZATION ////////////
  var_0_vec_h = normalize(L + V);
  var_3_vec_n = normalize(N);
  var_1_pi = 3.141592653589793;
  var_2_epsilon = 1.192092896e-07;
  var_4_vec_omega_i = L;
  var_5_theta_i = atan(var_4_vec_omega_i.y, var_4_vec_omega_i.x);
  var_6_phi_i = atan(sqrt(var_4_vec_omega_i.y * var_4_vec_omega_i.y +
                          var_4_vec_omega_i.x * var_4_vec_omega_i.x),
                     var_4_vec_omega_i.z);
  var_7_vec_omega_o = V;
  var_8_theta_o = atan(var_7_vec_omega_o.y, var_7_vec_omega_o.x);
  var_9_phi_o = atan(sqrt(var_7_vec_omega_o.y * var_7_vec_omega_o.y +
                          var_7_vec_omega_o.x * var_7_vec_omega_o.x),
                     var_7_vec_omega_o.z);
  var_10_theta_h = acos(dot(var_0_vec_h, N));
  var_11_theta_d = acos(dot(var_0_vec_h, var_4_vec_omega_i));
  //////////// END OF BUILTINS INITIALIZATION ////////////
  var_14_vec_X = var_12_text_normalize(cross(vec3(0.0, 1.0, 0.0), var_3_vec_n));
  var_15_vec_Y = var_12_text_normalize(cross(var_3_vec_n, var_14_vec_X));
  var_16_alpha_x = 0.4;
  var_17_alpha_y = 0.2;
  var_18_vec_C_d = vec3(1.0, 1.0, 1.0);
  var_19_vec_C_s = vec3(1.0, 1.0, 1.0);
  var_20_vec_H = var_12_text_normalize((var_4_vec_omega_i + var_7_vec_omega_o));
  var_21_text_exponent = (-((pow((dot(var_20_vec_H, var_14_vec_X) / var_16_alpha_x), 2.0) +
          pow((dot(var_20_vec_H, var_15_vec_Y) / var_17_alpha_y), 2.0)) /
         pow((dot(var_20_vec_H, var_3_vec_n)), 2.0)));
  var_22_text_spec = ((1.0 / ((((4.0 * var_1_pi) * var_16_alpha_x) * var_17_alpha_y) *
               sqrt(((dot(var_4_vec_omega_i, var_3_vec_n)) *
                     (dot(var_7_vec_omega_o, var_3_vec_n)))))) *
       exp((var_21_text_exponent)));
  var_23_f = ((var_18_vec_C_d / var_1_pi) + (var_19_vec_C_s * var_22_text_spec));

  return vec3(var_23_f);
}
\end{lstlisting}
\end{codigo}

%%%%%%%%%%%%%%%%%%%%%%%%%%%%%%%%%%%%%%%%%%%%%%%%%
\subsection{Visualização do Resultado}
%%%%%%%%%%%%%%%%%%%%%%%%%%%%%%%%%%%%%%%%%%%%%%%%%
\begin{figure}[H]
    \caption{\small{\textit{Plots} da distribuição de reflexão especular e difusa deste experimento.}}
    \label{fig-ward-plots}

\minipage{0.48\textwidth}
    \vspace{42px}
  \includegraphics[width=\linewidth]{./Imagens/brdfs/ward-3D-plot}
    % \vspace{0.1px}
    \legend{ \small (a) 3D \textit{plot}}
\endminipage\hfill
\minipage{0.48\textwidth}
  \includegraphics[width=\linewidth]{./Imagens/brdfs/ward-polar-plot.png}
    \legend{ \small (b) \textit{Polar plot}}
\endminipage\hfill
\end{figure}

\begin{figure}[H]
    \caption{\small{Objetos 3D renderizados por este experimento.}}\label{fig-ward-objetcs}
\minipage{0.32\textwidth}
  \includegraphics[width=\linewidth]{./Imagens/brdfs/ward-teapot.png}
    \legend{ \small (a) \textit{Teapot}}
\endminipage\hfill
\minipage{0.32\textwidth}
  \includegraphics[width=\linewidth]{./Imagens/brdfs/ward-dragon.png}
    \legend{ \small (b) Dragão de Stanford}
\endminipage\hfill
\minipage{0.32\textwidth}%
  \includegraphics[width=\linewidth]{./Imagens/brdfs/ward-goblin.png}
    \legend{ \small (c) Goblin}
\endminipage
\end{figure}


\input{Content/Resultados/Experimentos/Ashikhmin-Shirley}
\section{Experimento BRDF oren-nayar}

As equações que descrevem esse experimento se encontram em \autoref{fig-oren-nayar-eqlang-latex}. O código fonte de entrada para o compilador está dividido em duas partes, parte 1 está no \autoref{cod-oren-nayar-eqlang} e a segunda parte está em \autoref{cod-oren-nayar-eqlang-pt2}. A renderização dos objetos 3D usando essa BRDF se encontra em \autoref{fig-oren-nayar-eqlang}.

%%%%%%%%%%%%%%%%%%%%%%%%%%%%%%%%%%%%%%%%%%%%%%%%%
\subsection{Representação em documento \LaTeX{}}
%%%%%%%%%%%%%%%%%%%%%%%%%%%%%%%%%%%%%%%%%%%%%%%%%
\begin{figure}[H]
    \caption{\label{fig-oren-nayar-eqlang-latex} \small Equações da BRDF do experimento oren-nayar em documento \LaTeX{}.}
    \begin{center}
        \includegraphics[scale=0.82]{./Imagens/brdfs/oren-nayar.pdf}
    \end{center}
\end{figure}

%%%%%%%%%%%%%%%%%%%%%%%%%%%%%%%%%%%%%%%%%%%%%%%%%
\subsection{Visualização do Resultado}
%%%%%%%%%%%%%%%%%%%%%%%%%%%%%%%%%%%%%%%%%%%%%%%%%
\begin{figure}[H]
    \caption{\small{Distribuição de Reflexão Especular e Difusa da BRDF}}\label{fig-oren-nayar-eqlang}
\minipage{0.48\textwidth}
    \vspace{42px}
  \includegraphics[width=\linewidth]{./Imagens/brdfs/oren-nayar-3D-plot}
    % \vspace{0.1px}
    \legend{ \small (a) 3D \textit{plot}}
\endminipage\hfill
\minipage{0.48\textwidth}
  \includegraphics[width=\linewidth]{./Imagens/brdfs/oren-nayar-polar-plot.png}
    \legend{ \small (b) \textit{Polar plot}}
\endminipage\hfill
\end{figure}

\begin{figure}[H]
    \caption{\small{Objetos 3D renderizados por este experimento}}\label{fig-oren-nayar-eqlang}
\minipage{0.32\textwidth}
  \includegraphics[width=\linewidth]{./Imagens/brdfs/oren-nayar-teapot.png}
    \legend{ \small (a) \textit{Teapot}}
\endminipage\hfill
\minipage{0.32\textwidth}
  \includegraphics[width=\linewidth]{./Imagens/brdfs/oren-nayar-dragon.png}
    \legend{ \small (b) Dragão de Stanford}
\endminipage\hfill
\minipage{0.32\textwidth}%
  \includegraphics[width=\linewidth]{./Imagens/brdfs/oren-nayar-goblin.png}
    \legend{ \small (c) Goblin}
\endminipage
\end{figure}

%%%%%%%%%%%%%%%%%%%%%%%%%%%%%%%%%%%%%%%%%%%%%%%%%
\subsection{Código GLSL Gerado}
%%%%%%%%%%%%%%%%%%%%%%%%%%%%%%%%%%%%%%%%%%%%%%%%%
\begin{codigo}[H]
    \caption{\small Saida do compilador, código GLSL da BRDF deste experimento (parte 1). }
    \label{cod-oren-nayar-eqlang-declarations}
\begin{lstlisting}[language=C, inputencoding=utf8]
\end{lstlisting}
\end{codigo}

\begin{codigo}[H]
    \caption{\small Saida do compilador, código GLSL da BRDF deste experimento  (parte 2). }
    \label{cod-oren-nayar-eqlang}
\begin{lstlisting}[language=C, inputencoding=utf8]
\end{lstlisting}
\end{codigo}

%%%%%%%%%%%%%%%%%%%%%%%%%%%%%%%%%%%%%%%%%%%%%%%%%
\subsection{Código Fonte em \texttt{EquationLang}}
%%%%%%%%%%%%%%%%%%%%%%%%%%%%%%%%%%%%%%%%%%%%%%%%%
\begin{codigo}[H]
    \caption{\small Código fonte da BRDF deste experimento (parte 1).}
    \label{cod-oren-nayar-eqlang}
\begin{lstlisting}[language=tex, frame=none, inputencoding=utf8]
\end{lstlisting}
\end{codigo}

\section{Experimento BRDF ashikhmin-shirley-alternative}

As equações que descrevem esse experimento se encontram em \autoref{fig-ashikhmin-shirley-alternative-eqlang-latex}. O código fonte de entrada para o compilador está dividido em duas partes, parte 1 está no \autoref{cod-ashikhmin-shirley-alternative-eqlang} e a segunda parte está em \autoref{cod-ashikhmin-shirley-alternative-eqlang-pt2}. A renderização dos objetos 3D usando essa BRDF se encontra em \autoref{fig-ashikhmin-shirley-alternative-eqlang}.

%%%%%%%%%%%%%%%%%%%%%%%%%%%%%%%%%%%%%%%%%%%%%%%%%
\subsection{Representação em documento \LaTeX{}}
%%%%%%%%%%%%%%%%%%%%%%%%%%%%%%%%%%%%%%%%%%%%%%%%%
\begin{figure}[H]
    \caption{\label{fig-ashikhmin-shirley-alternative-eqlang-latex} \small Equações da BRDF do experimento ashikhmin-shirley-alternative em documento \LaTeX{}.}
    \begin{center}
        \includegraphics[scale=0.92]{./Imagens/brdfs/ashikhmin-shirley-alternative.pdf}
    \end{center}
\end{figure}

%%%%%%%%%%%%%%%%%%%%%%%%%%%%%%%%%%%%%%%%%%%%%%%%%
\subsection{Visualização do Resultado}
%%%%%%%%%%%%%%%%%%%%%%%%%%%%%%%%%%%%%%%%%%%%%%%%%
\begin{figure}[H]
    \caption{\small{Distribuição de Reflexão Especular e Difusa da BRDF}}\label{fig-ashikhmin-shirley-alternative-eqlang}
\minipage{0.48\textwidth}
    \vspace{42px}
  \includegraphics[width=\linewidth]{./Imagens/brdfs/ashikhmin-shirley-alternative-3D-plot}
    % \vspace{0.1px}
    \legend{ \small (a) 3D \textit{plot}}
\endminipage\hfill
\minipage{0.48\textwidth}
  \includegraphics[width=\linewidth]{./Imagens/brdfs/ashikhmin-shirley-alternative-polar-plot.png}
    \legend{ \small (b) \textit{Polar plot}}
\endminipage\hfill
\end{figure}

\begin{figure}[H]
    \caption{\small{Objetos 3D renderizados por este experimento}}\label{fig-ashikhmin-shirley-alternative-eqlang}
\minipage{0.32\textwidth}
  \includegraphics[width=\linewidth]{./Imagens/brdfs/ashikhmin-shirley-alternative-teapot.png}
    \legend{ \small (a) \textit{Teapot}}
\endminipage\hfill
\minipage{0.32\textwidth}
  \includegraphics[width=\linewidth]{./Imagens/brdfs/ashikhmin-shirley-alternative-dragon.png}
    \legend{ \small (b) Dragão de Stanford}
\endminipage\hfill
\minipage{0.32\textwidth}%
  \includegraphics[width=\linewidth]{./Imagens/brdfs/ashikhmin-shirley-alternative-goblin.png}
    \legend{ \small (c) Goblin}
\endminipage
\end{figure}

%%%%%%%%%%%%%%%%%%%%%%%%%%%%%%%%%%%%%%%%%%%%%%%%%
\subsection{Código GLSL Gerado}
%%%%%%%%%%%%%%%%%%%%%%%%%%%%%%%%%%%%%%%%%%%%%%%%%
\begin{codigo}[H]
    \caption{\small Saida do compilador, código GLSL da BRDF deste experimento (parte 1). }
    \label{cod-ashikhmin-shirley-alternative-eqlang-declarations}
\begin{lstlisting}[language=C, inputencoding=utf8]
\end{lstlisting}
\end{codigo}

\begin{codigo}[H]
    \caption{\small Saida do compilador, código GLSL da BRDF deste experimento  (parte 2). }
    \label{cod-ashikhmin-shirley-alternative-eqlang}
\begin{lstlisting}[language=C, inputencoding=utf8]
\end{lstlisting}
\end{codigo}

%%%%%%%%%%%%%%%%%%%%%%%%%%%%%%%%%%%%%%%%%%%%%%%%%
\subsection{Código Fonte em \texttt{EquationLang}}
%%%%%%%%%%%%%%%%%%%%%%%%%%%%%%%%%%%%%%%%%%%%%%%%%
\begin{codigo}[H]
    \caption{\small Código fonte da BRDF deste experimento (parte 1).}
    \label{cod-ashikhmin-shirley-alternative-eqlang}
\begin{lstlisting}[language=tex, frame=none, inputencoding=utf8]
\end{lstlisting}
\end{codigo}

\section{Experimento BRDF cook-torrance-alternative}

As equações que descrevem esse experimento se encontram em \autoref{fig-cook-torrance-alternative-eqlang-latex}. O código fonte de entrada para o compilador está dividido em duas partes, parte 1 está no \autoref{cod-cook-torrance-alternative-eqlang} e a segunda parte está em \autoref{cod-cook-torrance-alternative-eqlang-pt2}. A renderização dos objetos 3D usando essa BRDF se encontra em \autoref{fig-cook-torrance-alternative-eqlang}.

%%%%%%%%%%%%%%%%%%%%%%%%%%%%%%%%%%%%%%%%%%%%%%%%%
\subsection{Representação em documento \LaTeX{}}
%%%%%%%%%%%%%%%%%%%%%%%%%%%%%%%%%%%%%%%%%%%%%%%%%
\begin{figure}[H]
    \caption{\label{fig-cook-torrance-alternative-eqlang-latex} \small Equações da BRDF do experimento cook-torrance-alternative em documento \LaTeX{}.}
    \begin{center}
        \includegraphics[scale=0.92]{./Imagens/brdfs/cook-torrance-alternative.pdf}
    \end{center}
\end{figure}

%%%%%%%%%%%%%%%%%%%%%%%%%%%%%%%%%%%%%%%%%%%%%%%%%
\subsection{Visualização do Resultado}
%%%%%%%%%%%%%%%%%%%%%%%%%%%%%%%%%%%%%%%%%%%%%%%%%
\begin{figure}[H]
    \caption{\small{Distribuição de Reflexão Especular e Difusa da BRDF}}\label{fig-cook-torrance-alternative-eqlang}
\minipage{0.48\textwidth}
    \vspace{42px}
  \includegraphics[width=\linewidth]{./Imagens/brdfs/cook-torrance-alternative-3D-plot}
    % \vspace{0.1px}
    \legend{ \small (a) 3D \textit{plot}}
\endminipage\hfill
\minipage{0.48\textwidth}
  \includegraphics[width=\linewidth]{./Imagens/brdfs/cook-torrance-alternative-polar-plot-log.png}
    \legend{ \small (b) \textit{Polar plot}}
\endminipage\hfill
\end{figure}

\begin{figure}[H]
    \caption{\small{Objetos 3D renderizados por este experimento}}\label{fig-cook-torrance-alternative-eqlang}
\minipage{0.32\textwidth}
  \includegraphics[width=\linewidth]{./Imagens/brdfs/cook-torrance-alternative-teapot.png}
    \legend{ \small (a) \textit{Teapot}}
\endminipage\hfill
\minipage{0.32\textwidth}
  \includegraphics[width=\linewidth]{./Imagens/brdfs/cook-torrance-alternative-dragon.png}
    \legend{ \small (b) Dragão de Stanford}
\endminipage\hfill
\minipage{0.32\textwidth}%
  \includegraphics[width=\linewidth]{./Imagens/brdfs/cook-torrance-alternative-goblin.png}
    \legend{ \small (c) Goblin}
\endminipage
\end{figure}

%%%%%%%%%%%%%%%%%%%%%%%%%%%%%%%%%%%%%%%%%%%%%%%%%
\subsection{Código GLSL Gerado}
%%%%%%%%%%%%%%%%%%%%%%%%%%%%%%%%%%%%%%%%%%%%%%%%%
\begin{codigo}[H]
    \caption{\small Saida do compilador, código GLSL da BRDF deste experimento (parte 1). }
    \label{cod-cook-torrance-alternative-eqlang-declarations}
\begin{lstlisting}[language=C, inputencoding=utf8]
\end{lstlisting}
\end{codigo}

\begin{codigo}[H]
    \caption{\small Saida do compilador, código GLSL da BRDF deste experimento  (parte 2). }
    \label{cod-cook-torrance-alternative-eqlang}
\begin{lstlisting}[language=C, inputencoding=utf8]
\end{lstlisting}
\end{codigo}

%%%%%%%%%%%%%%%%%%%%%%%%%%%%%%%%%%%%%%%%%%%%%%%%%
\subsection{Código Fonte em \texttt{EquationLang}}
%%%%%%%%%%%%%%%%%%%%%%%%%%%%%%%%%%%%%%%%%%%%%%%%%
\begin{codigo}[H]
    \caption{\small Código fonte da BRDF deste experimento (parte 1).}
    \label{cod-cook-torrance-alternative-eqlang}
\begin{lstlisting}[language=tex, frame=none, inputencoding=utf8]
\end{lstlisting}
\end{codigo}

\section{Experimento BRDF Dür}
\label{section-experiment-duer}

No artigo de Geisler-Moroder e Dür \cite{duer2010bounding}, é discutido sobre as limitações do modelo de reflexão de Ward, no qual é proposto uma forma de restringir o albedo e garantir a conservação de energia. Este experimento é baseado nessa BRDF com albedo restringido. As equações são apresentadas na \autoref{fig-duer-eqlang-latex}, com o código fonte em \texttt{EquationLang} disponível no \autoref{cod-duer-eqlang}. Os códigos gerados em GLSL podem ser vistos no \autoref{cod-duer-glsl-pt-1} e no \autoref{cod-duer-glsl-pt-2}. A renderização dos objetos 3D pode ser observada na \autoref{fig-duer-eqlang} e os \textit{plots} na \autoref{fig-duer-plots}.

%%%%%%%%%%%%%%%%%%%%%%%%%%%%%%%%%%%%%%%%%%%%%%%%%
\subsection{Representação em documento \LaTeX{}}
%%%%%%%%%%%%%%%%%%%%%%%%%%%%%%%%%%%%%%%%%%%%%%%%%
\begin{figure}[H]
    \caption{\label{fig-duer-eqlang-latex} \small Equações da BRDF do experimento Dür em documento \LaTeX{}.}
    \begin{center}
        \includegraphics[scale=0.92]{./Imagens/brdfs/duer.pdf}
    \end{center}
\end{figure}

%%%%%%%%%%%%%%%%%%%%%%%%%%%%%%%%%%%%%%%%%%%%%%%%%
\subsection{Código Fonte em \texttt{EquationLang}}
%%%%%%%%%%%%%%%%%%%%%%%%%%%%%%%%%%%%%%%%%%%%%%%%%
\begin{codigo}[H]
    \caption{\small Código fonte da BRDF do experimento Dür.}
    \label{cod-duer-eqlang}
\begin{lstlisting}[language=tex, frame=none, inputencoding=utf8]
Duer 2010 Bounding the Albedo of the Ward Reflectance Model

\begin{equation}
G = ((\vec{\omega_i}+\vec{\omega_o}) \cdot(\vec{\omega_i}+\vec{\omega_o})) * ((\vec{\omega_i}+\vec{\omega_o}) \cdot \vec{n})^-4
    * (\vec{n} \cdot \vec{\omega_i})*(\vec{n} \cdot \vec{\omega_o})
\end{equation}

\begin{equation}
f =  G
\end{equation}
\end{lstlisting}
\end{codigo}

%%%%%%%%%%%%%%%%%%%%%%%%%%%%%%%%%%%%%%%%%%%%%%%%%
\subsection{Código GLSL Gerado}
%%%%%%%%%%%%%%%%%%%%%%%%%%%%%%%%%%%%%%%%%%%%%%%%%
\begin{codigo}[H]
    \caption{\small Saída do compilador: código GLSL da BRDF do experimento Dür (parte 1 de 2).}
    \label{cod-duer-glsl-pt-1}
\begin{lstlisting}[language=C, inputencoding=utf8]
analytic ::begin parameters
#[type][name][min val][max val][default val]
::end parameters
::begin shader
//////////// START OF BUILTINS DECLARTION ////////////
vec3 var_0_vec_h;
vec3 var_3_vec_n;
float var_10_theta_h;
float var_11_theta_d;
float var_1_pi;
float var_2_epsilon;
vec3 var_4_vec_omega_i;
float var_5_theta_i;
float var_6_phi_i;
vec3 var_7_vec_omega_o;
float var_8_theta_o;
float var_9_phi_o;
//////////// END OF BUILTINS DECLARTION ////////////
//////////// START OF USER DECLARED ////////////
float var_12_G;
float var_13_f;
//////////// END OF USER DECLARED ////////////
//////////// START FUNCTIONS DECLARATIONS ////////////
//////////// END FUNCTIONS DECLARATIONS ////////////

\end{lstlisting}
\end{codigo}

\begin{codigo}[H]
    \caption{\small Saída do compilador: código GLSL da BRDF do experimento Dür (parte 2 de 2).}
    \label{cod-duer-glsl-pt-2}
\begin{lstlisting}[language=C, inputencoding=utf8]
vec3 BRDF(vec3 L, vec3 V, vec3 N, vec3 X, vec3 Y) {

  //////////// START OF BUILTINS INITIALIZATION ////////////
  var_0_vec_h = normalize(L + V);
  var_3_vec_n = normalize(N);
  var_1_pi = 3.141592653589793;
  var_2_epsilon = 1.192092896e-07;
  var_4_vec_omega_i = L;
  var_5_theta_i = atan(var_4_vec_omega_i.y, var_4_vec_omega_i.x);
  var_6_phi_i = atan(sqrt(var_4_vec_omega_i.y * var_4_vec_omega_i.y +
                          var_4_vec_omega_i.x * var_4_vec_omega_i.x),
                     var_4_vec_omega_i.z);
  var_7_vec_omega_o = V;
  var_8_theta_o = atan(var_7_vec_omega_o.y, var_7_vec_omega_o.x);
  var_9_phi_o = atan(sqrt(var_7_vec_omega_o.y * var_7_vec_omega_o.y +
                          var_7_vec_omega_o.x * var_7_vec_omega_o.x),
                     var_7_vec_omega_o.z);
  var_10_theta_h = acos(dot(var_0_vec_h, N));
  var_11_theta_d = acos(dot(var_0_vec_h, var_4_vec_omega_i));
  //////////// END OF BUILTINS INITIALIZATION ////////////

  var_12_G =
      ((((dot(((var_4_vec_omega_i + var_7_vec_omega_o)),
              ((var_4_vec_omega_i + var_7_vec_omega_o)))) *
         pow((dot(((var_4_vec_omega_i + var_7_vec_omega_o)), var_3_vec_n)),
             (-4.0))) *
        (dot(var_3_vec_n, var_4_vec_omega_i))) *
       (dot(var_3_vec_n, var_7_vec_omega_o)));
  var_13_f = var_12_G;

  return vec3(var_13_f);
}
\end{lstlisting}
\end{codigo}


%%%%%%%%%%%%%%%%%%%%%%%%%%%%%%%%%%%%%%%%%%%%%%%%%
\subsection{Visualização do Resultado}
%%%%%%%%%%%%%%%%%%%%%%%%%%%%%%%%%%%%%%%%%%%%%%%%%
\begin{figure}[H]
    \caption{\small{\textit{Plots} da distribuição de reflexão especular e difusa do experimento Dür.}}
    \label{fig-duer-plots}
\minipage{0.48\textwidth}
    \vspace{42px}
  \includegraphics[width=\linewidth]{./Imagens/brdfs/duer-3D-plot}
    % \vspace{0.1px}
    \legend{ \small (a) 3D \textit{plot}}
\endminipage\hfill
\minipage{0.48\textwidth}
  \includegraphics[width=\linewidth]{./Imagens/brdfs/duer-polar-plot.png}
    \legend{ \small (b) \textit{Polar plot}}
\endminipage\hfill
\end{figure}

\begin{figure}[H]
    \caption{\small{Objetos 3D renderizados pelo experimento Dür.}}
    \label{fig-duer-eqlang}
\minipage{0.32\textwidth}
  \includegraphics[width=\linewidth]{./Imagens/brdfs/duer-teapot.png}
    \legend{ \small (a) \textit{Teapot}}
\endminipage\hfill
\minipage{0.32\textwidth}
  \includegraphics[width=\linewidth]{./Imagens/brdfs/duer-dragon.png}
    \legend{ \small (b) Dragão de Stanford}
\endminipage\hfill
\minipage{0.32\textwidth}%
  \includegraphics[width=\linewidth]{./Imagens/brdfs/duer-goblin.png}
    \legend{ \small (c) Goblin}
\endminipage
\end{figure}



\section{Experimento BRDF Edwards 2006}
\label{section-experiment-edwards-2006}

No artigo de Edwards et al. \cite{edwards2006halfway}, é apresentado o conceito do \textit{Halfway Vector Disk} como uma extensão para modelagem de BRDFs. Este método, usado neste experimento, propõe uma ideia geométrica que melhora a eficiência computacional. As equações principais são descritas na \autoref{fig-edwards-2006-eqlang-latex}, com o código fonte em \texttt{EquationLang} disponível no \autoref{cod-edwards-2006-eqlang}. Os códigos gerados em GLSL podem ser vistos no \autoref{cod-edwards-2006-glsl-pt-1} e no \autoref{cod-edwards-2006-glsl-pt-2}. A renderização de objetos 3D utilizando o método pode ser observada na \autoref{fig-edwards-2006-eqlang}, enquanto os \textit{plots} estão ilustrados na \autoref{fig-edwards-2006-plots}.

%%%%%%%%%%%%%%%%%%%%%%%%%%%%%%%%%%%%%%%%%%%%%%%%%
\subsection{Representação em documento \LaTeX{}}
%%%%%%%%%%%%%%%%%%%%%%%%%%%%%%%%%%%%%%%%%%%%%%%%%
\begin{figure}[H]
    \caption{\label{fig-edwards-2006-eqlang-latex} \small Equações da BRDF do experimento Edwards em documento \LaTeX{}.}
    \begin{center}
        \includegraphics[scale=0.92]{./Imagens/brdfs/edwards-2006.pdf}
    \end{center}
\end{figure}

%%%%%%%%%%%%%%%%%%%%%%%%%%%%%%%%%%%%%%%%%%%%%%%%%
\subsection{Código GLSL Gerado}
%%%%%%%%%%%%%%%%%%%%%%%%%%%%%%%%%%%%%%%%%%%%%%%%%
\begin{codigo}[H]
    \caption{\small Saída do compilador: código GLSL da BRDF do experimento Edwards (parte 1 de 2).}
    \label{cod-edwards-2006-glsl-pt-1}
\begin{lstlisting}[language=C, inputencoding=utf8]
analytic ::begin parameters
#[type][name][min val][max val][default val]
::end parameters
::begin shader
//////////// START OF BUILTINS DECLARTION ////////////
vec3 var_0_vec_h;
vec3 var_3_vec_n;
float var_10_theta_h;
float var_11_theta_d;
float var_1_pi;
float var_2_epsilon;
vec3 var_4_vec_omega_i;
float var_5_theta_i;
float var_6_phi_i;
vec3 var_7_vec_omega_o;
float var_8_theta_o;
float var_9_phi_o;
//////////// END OF BUILTINS DECLARTION ////////////

//////////// START OF USER DECLARED ////////////
vec3 var_15_uH;
vec3 var_16_h;
float var_14_n;
vec3 var_17_huv;
float var_13_R;
float var_18_p;
float var_19_f;
\end{lstlisting}
\end{codigo}

\begin{codigo}[H]
    \caption{\small Saída do compilador: código GLSL da BRDF do experimento Edwards (parte 2 de 2).}
    \label{cod-edwards-2006-glsl-pt-2}
\begin{lstlisting}[language=C, inputencoding=utf8]
//////////// END OF USER DECLARED ////////////

//////////// START FUNCTIONS DECLARATIONS ////////////
float var_12_text_lump(vec3 var_0_vec_h, float var_13_R, float var_14_n) {
  return ((((var_14_n + 1.0)) / (((var_1_pi * var_13_R) * var_13_R))) *
          ((1.0 - ((dot(var_0_vec_h, var_0_vec_h)) /
                   pow(((var_13_R * var_13_R)), var_14_n)))));
}
//////////// END FUNCTIONS DECLARATIONS ////////////
vec3 BRDF(vec3 L, vec3 V, vec3 N, vec3 X, vec3 Y) {

  //////////// START OF BUILTINS INITIALIZATION ////////////
  var_0_vec_h = normalize(L + V);
  var_3_vec_n = normalize(N);
  var_1_pi = 3.141592653589793;
  var_2_epsilon = 1.192092896e-07;
  var_4_vec_omega_i = L;
  var_5_theta_i = atan(var_4_vec_omega_i.y, var_4_vec_omega_i.x);
  var_6_phi_i = atan(sqrt(var_4_vec_omega_i.y * var_4_vec_omega_i.y +
                          var_4_vec_omega_i.x * var_4_vec_omega_i.x),
                     var_4_vec_omega_i.z);
  var_7_vec_omega_o = V;
  var_8_theta_o = atan(var_7_vec_omega_o.y, var_7_vec_omega_o.x);
  var_9_phi_o = atan(sqrt(var_7_vec_omega_o.y * var_7_vec_omega_o.y +
                          var_7_vec_omega_o.x * var_7_vec_omega_o.x),
                     var_7_vec_omega_o.z);
  var_10_theta_h = acos(dot(var_0_vec_h, N));
  var_11_theta_d = acos(dot(var_0_vec_h, var_4_vec_omega_i));
  //////////// END OF BUILTINS INITIALIZATION ////////////

  var_15_uH = (var_4_vec_omega_i + var_7_vec_omega_o);
  var_16_h =
      (((dot(var_3_vec_n, var_7_vec_omega_o)) / (dot(var_3_vec_n, var_15_uH))) *
       var_15_uH);
  var_14_n = 10.0;
  var_17_huv =
      (var_16_h - ((dot(var_3_vec_n, var_7_vec_omega_o)) * var_3_vec_n));
  var_13_R = 1.0;
  var_18_p = var_12_text_lump(var_17_huv, var_13_R, var_14_n);
  var_19_f = ((var_18_p * (pow((dot(var_3_vec_n, var_7_vec_omega_o)), 2.0))) /
              ((((4.0 * (dot(var_3_vec_n, var_4_vec_omega_i))) *
                 (dot(var_4_vec_omega_i, var_0_vec_h))) *
                (pow((dot(var_3_vec_n, var_0_vec_h)), 3.0)))));

  return vec3(var_19_f);
}
\end{lstlisting}
\end{codigo}

%%%%%%%%%%%%%%%%%%%%%%%%%%%%%%%%%%%%%%%%%%%%%%%%%
\subsection{Código Fonte em \texttt{EquationLang}}
%%%%%%%%%%%%%%%%%%%%%%%%%%%%%%%%%%%%%%%%%%%%%%%%%
\begin{codigo}[H]
    \caption{\small Código fonte da BRDF do experimento Edwards.}
    \label{cod-edwards-2006-eqlang}
\begin{lstlisting}[language=tex, frame=none, inputencoding=utf8]
\begin{equation}
n = 10
\end{equation}

\begin{equation}
R = 1
\end{equation}

\begin{equation}
\text{lump}(\vec{h}, R, n) = (n+1)/(\pi*R*R) * (1-(\vec{h} \cdot \vec{h})/(R*R)^ n)
\end{equation}

Scaling projection
\begin{equation}
    uH = \vec{\omega_i}+\vec{\omega_o} % // unnormalized H
\end{equation}

\begin{equation}
    h = (\vec{n} \cdot \vec{\omega_o}) / (\vec{n} \cdot uH) * uH
\end{equation}

\begin{equation}
    huv = h - (\vec{n} \cdot \vec{\omega_o}) * \vec{n}
\end{equation}

Specular term (D and G)

\begin{equation}
    p = \text{lump}(huv, R, n)
\end{equation}

\begin{equation}
    f = p * ((\vec{n} \cdot \vec{\omega_o})^ 2)
        / (4 * (\vec{n} \cdot \vec{\omega_i}) * (\vec{\omega_i} \cdot \vec{h})
        * ((\vec{n} \cdot \vec{h})^ 3))
\end{equation}
\end{lstlisting}
\end{codigo}

%%%%%%%%%%%%%%%%%%%%%%%%%%%%%%%%%%%%%%%%%%%%%%%%%
\subsection{Visualização do Resultado}
%%%%%%%%%%%%%%%%%%%%%%%%%%%%%%%%%%%%%%%%%%%%%%%%%
\begin{figure}[H]
    \caption{\small{\textit{Plots} da distribuição de reflexão especular e difusa do experimento Edwards.}}
    \label{fig-edwards-2006-plots}
\minipage{0.48\textwidth}
    \vspace{42px}
  \includegraphics[width=\linewidth]{./Imagens/brdfs/edwards-2006-3D-plot}
    % \vspace{0.1px}
    \legend{ \small (a) 3D \textit{plot}}
\endminipage\hfill
\minipage{0.48\textwidth}
  \includegraphics[width=\linewidth]{./Imagens/brdfs/edwards-2006-polar-plot.png}
    \legend{ \small (b) \textit{Polar plot}}
\endminipage\hfill
\end{figure}

\begin{figure}[H]
    \caption{\small{Objetos 3D renderizados pelo experimento Edwards.}}\label{fig-edwards-2006-eqlang}
\minipage{0.32\textwidth}
  \includegraphics[width=\linewidth]{./Imagens/brdfs/edwards-2006-teapot.png}
    \legend{ \small (a) \textit{Teapot}}
\endminipage\hfill
\minipage{0.32\textwidth}
  \includegraphics[width=\linewidth]{./Imagens/brdfs/edwards-2006-dragon.png}
    \legend{ \small (b) Dragão de Stanford}
\endminipage\hfill
\minipage{0.32\textwidth}%
  \includegraphics[width=\linewidth]{./Imagens/brdfs/edwards-2006-goblin.png}
    \legend{ \small (c) Goblin}
\endminipage
\end{figure}



\section{Experimento BRDF Anisotrópica baseado em Kajiya-Kay (1989)}

Este experimento é baseado no modelo anisotrópico que descreve o comportamento de reflexão de superfícies rugosas simplificadas proposto no trabalho de Kajiya \cite{kajiya1985anisotropic}. As equações e parâmetros escolhidos que descrevem esse modelo estão em \autoref{fig-kajiya-eqlang-latex}. O código fonte em \texttt{EquationLang} para o compilador está em \autoref{cod-kajiya-eqlang}. A saída gerada pelo compilador está dividido em duas partes: a parte 1 está no \autoref{cod-kajiya-glsl-pt-1}, enquanto a parte 2 está em \autoref{cod-kajiya-glsl-pt-2}. A renderização dos objetos 3D usando essa BRDF se encontra em \autoref{fig-kajiya-objects}. Utilizamos interpolação linear para aproximar valores calculados previamente na tabela de refletância, com plots logarítmicos e polares presentes na \autoref{fig-kajiya-plots}.

%%%%%%%%%%%%%%%%%%%%%%%%%%%%%%%%%%%%%%%%%%%%%%%%%
\subsection{Representação em documento \LaTeX{}}
%%%%%%%%%%%%%%%%%%%%%%%%%%%%%%%%%%%%%%%%%%%%%%%%%
\begin{figure}[H]
    \caption{\label{fig-kajiya-eqlang-latex} \small Equações da BRDF do experimento Kajiya-Kay em documento \LaTeX{}.}
    \begin{center}
        % \includegraphics[scale=1.1,width=\textwidth]{./Imagens/brdfs/aniso.pdf}
        \includegraphics[scale=0.92]{./Imagens/brdfs/aniso.pdf}
    \end{center}
\end{figure}

%%%%%%%%%%%%%%%%%%%%%%%%%%%%%%%%%%%%%%%%%%%%%%%%%
\subsection{Visualização do Resultado}
%%%%%%%%%%%%%%%%%%%%%%%%%%%%%%%%%%%%%%%%%%%%%%%%%

\begin{figure}[H]
    \caption{\small{\textit{Plots} da BRDF deste experimento.}}\label{fig-kajiya-plots}
\minipage{0.48\textwidth}
    \vspace{42px}
  \includegraphics[width=\linewidth]{./Imagens/brdfs/aniso-3D-plot}
    % \caption{\small{(a)}}\label{fig:awesome_image1}
    % \vspace{0.1px}
    \legend{ \small (a) 3D \textit{plot}}
\endminipage\hfill
\minipage{0.48\textwidth}
  \includegraphics[width=\linewidth]{./Imagens/brdfs/aniso-polar-plot.png}
    \legend{ \small (b) \textit{Polar plot}}
    % \caption{\small{(b)}}\label{fig:awesome_image1}
\endminipage\hfill
\end{figure}

\begin{figure}[H]
    % \caption{\small{Objetos 3D renderizado pelo código GLSL gerado o experimento BRDF Anisotrópica: Kajiya-Kay (1989)}}\label{fig-kajiya-eqlang}
    \caption{\small{Objetos 3D renderizado no experimento BRDF Anisotrópica: Kajiya-Kay (1989)}}\label{fig-kajiya-objects}
\minipage{0.32\textwidth}
  \includegraphics[width=\linewidth]{./Imagens/brdfs/aniso-teapot.png}
    % \caption{\small{(a)}}\label{fig:awesome_image1}
    \legend{ \small (a) \textit{Teapot}}
\endminipage\hfill
\minipage{0.32\textwidth}
  \includegraphics[width=\linewidth]{./Imagens/brdfs/aniso-dragon.png}
    \legend{ \small (b) Dragão de Stanford}
    % \caption{\small{(b)}}\label{fig:awesome_image1}
\endminipage\hfill
\minipage{0.32\textwidth}%
  \includegraphics[width=\linewidth]{./Imagens/brdfs/aniso-goblin.png}
    \legend{ \small (c) Goblin}
    % \caption{\small{(c)}}\label{fig:awesome_image1}
\endminipage
\end{figure}

%%%%%%%%%%%%%%%%%%%%%%%%%%%%%%%%%%%%%%%%%%%%%%%%%
\subsection{Código GLSL Gerado}
%%%%%%%%%%%%%%%%%%%%%%%%%%%%%%%%%%%%%%%%%%%%%%%%%
\begin{codigo}[H]
    \caption{\small Saida do compilador, código GLSL da BRDF do experimento baseado em Kajiya-Kay (parte 1). }
    \label{cod-kajiya-glsl-pt-1}
\begin{lstlisting}[language=C, inputencoding=utf8]
analytic ::begin parameters
#[type][name][min val][max val][default val]
::end parameters
::begin shader
//////////// START OF BUILTINS DECLARTION ////////////
vec3 var_0_vec_h;
vec3 var_3_vec_n;
float var_10_theta_h;
float var_11_theta_d;
float var_1_pi;
float var_2_epsilon;
vec3 var_4_vec_omega_i;
float var_5_theta_i;
float var_6_phi_i;
vec3 var_7_vec_omega_o;
float var_8_theta_o;
float var_9_phi_o;
//////////// END OF BUILTINS DECLARTION ////////////

//////////// START OF USER DECLARED ////////////
vec3 var_12_L;
vec3 var_15_X;
vec3 var_16_Y;
vec3 var_17_T;
float var_18_s_alpha;
float var_19_text_roughness;
float var_20_text_glossiness;
float var_21_text_spec;
float var_22_f;
//////////// END OF USER DECLARED ////////////
//////////// START FUNCTIONS DECLARATIONS ////////////
vec3 var_13_text_normalize(vec3 var_14_vec_u) {
  return (var_14_vec_u / sqrt(dot(var_14_vec_u, var_14_vec_u)));
}
//////////// END FUNCTIONS DECLARATIONS ////////////
\end{lstlisting}
\end{codigo}

\begin{codigo}[H]
    \caption{\small Saida do compilador, código GLSL da BRDF do experimento baseado em Kajiya-Kay (parte 2). }
    \label{cod-kajiya-glsl-pt-2}
\begin{lstlisting}[language=C, inputencoding=utf8]
vec3 BRDF(vec3 L, vec3 V, vec3 N, vec3 X, vec3 Y) {
  //////////// START OF BUILTINS INITIALIZATION ////////////
  var_0_vec_h = normalize(L + V);
  var_3_vec_n = normalize(N);
  var_1_pi = 3.141592653589793;
  var_2_epsilon = 1.192092896e-07;
  var_4_vec_omega_i = L;
  var_5_theta_i = atan(var_4_vec_omega_i.y, var_4_vec_omega_i.x);
  var_6_phi_i = atan(sqrt(var_4_vec_omega_i.y * var_4_vec_omega_i.y +
                          var_4_vec_omega_i.x * var_4_vec_omega_i.x),
                     var_4_vec_omega_i.z);
  var_7_vec_omega_o = V;
  var_8_theta_o = atan(var_7_vec_omega_o.y, var_7_vec_omega_o.x);
  var_9_phi_o = atan(sqrt(var_7_vec_omega_o.y * var_7_vec_omega_o.y +
                          var_7_vec_omega_o.x * var_7_vec_omega_o.x),
                     var_7_vec_omega_o.z);
  var_10_theta_h = acos(dot(var_0_vec_h, N));
  var_11_theta_d = acos(dot(var_0_vec_h, var_4_vec_omega_i));
  //////////// END OF BUILTINS INITIALIZATION ////////////
  var_12_L = var_4_vec_omega_i;
  var_15_X = var_13_text_normalize(cross(vec3(0.0, 1.0, 0.0), var_3_vec_n));
  var_16_Y = var_13_text_normalize(cross(var_3_vec_n, var_15_X));
  var_17_T = var_16_Y;
  var_18_s_alpha = sqrt(((1.0 - (((dot(var_4_vec_omega_i, var_17_T)) *
                                  (dot(var_4_vec_omega_i, var_17_T)))))));
  var_19_text_roughness = 0.1;
  var_20_text_glossiness = ((1.0 / var_19_text_roughness));
  var_21_text_spec =
      pow(((((var_18_s_alpha *
              sqrt(((1.0 - (((dot(var_7_vec_omega_o, var_17_T)) *
                             (dot(var_7_vec_omega_o, var_17_T))))))))) -
            (((dot(var_4_vec_omega_i, var_17_T)) *
              (dot(var_7_vec_omega_o, var_17_T)))))),
          var_20_text_glossiness);
  var_22_f = var_21_text_spec;

  return vec3(var_22_f);
}


\end{lstlisting}
\end{codigo}

%%%%%%%%%%%%%%%%%%%%%%%%%%%%%%%%%%%%%%%%%%%%%%%%%
\subsection{Código Fonte em \texttt{EquationLang}}
%%%%%%%%%%%%%%%%%%%%%%%%%%%%%%%%%%%%%%%%%%%%%%%%%
\begin{codigo}[H]
    \caption{\small Código fonte da BRDF do experimento Kajiya-Kay.}
    \label{cod-kajiya-eqlang}
\begin{lstlisting}[language=tex, frame=none, inputencoding=utf8]
Based on Kajiya-Kay 1989

\begin{equation}
  \text{normalize}(\vec{u}) = \frac{\vec{u}}{\sqrt{\vec{u} \cdot \vec{u}}}
\end{equation}

Tangent vector:
\begin{equation}
   X = \text{normalize}(\vec{0,1,0} \times \vec{n})
\end{equation}

Bitangent vector:
\begin{equation}
   Y = \text{normalize}(\vec{n} \times X)
\end{equation}

\begin{equation}
    T = Y
\end{equation}

\begin{equation}
    L = \vec{\omega_i}
\end{equation}

\begin{equation}
    \text{roughness} =  0.1
\end{equation}

\begin{equation}
    \text{glossiness} = (1/\text{roughness})
\end{equation}

\begin{equation}
    s_\alpha = \sqrt{(1 - ((\vec{\omega_i} \cdot T) * (\vec{\omega_i} \cdot T)))}
\end{equation}

\begin{equation}
\text{spec} = ((s_\alpha  \cdot \sqrt(1 - ((\vec \omega_o \cdot T) \cdot (\vec \omega_o \cdot T))))
                  - ((\vec{\omega_i} \cdot T) \cdot (\vec \omega_o \cdot T)))^ \text{glossiness}
\end{equation}
\begin{equation}
f = \text{spec}
\end{equation}
\end{lstlisting}
\end{codigo}

\section{Experimento BRDF Minnaert}
\label{section-experiment-minnaert}

Este experimento foi realizado seguindo os princípios do artigo de Minnaert \cite{minnaert1941reciprocity}. Nele, é apresentado um modelo de reflexão que introduz uma abordagem para descrever superfícies que exibem comportamentos encontrados em superfícies porosas, como a lua. As equações desse experimento estão na \autoref{fig-minnaert-eqlang-latex}. O código-fonte pode ser encontrado no \autoref{cod-minnaert-eqlang}. O GLSL gerado pode ser encontrado no \autoref{cod-minnaert-glsl-pt-1} e no \autoref{cod-minnaert-glsl-pt-2}, enquanto os resultados de renderização podem ser observados na \autoref{fig-minnaert-eqlang} e os \textit{plots} na \autoref{fig-minnaert-plots}.


%%%%%%%%%%%%%%%%%%%%%%%%%%%%%%%%%%%%%%%%%%%%%%%%%
\subsection{Representação em documento \LaTeX{}}
%%%%%%%%%%%%%%%%%%%%%%%%%%%%%%%%%%%%%%%%%%%%%%%%%
\begin{figure}[H]
    \caption{\label{fig-minnaert-eqlang-latex} \small Equações da BRDF do experimento Minnaert em documento \LaTeX{}.}
    \begin{center}
        \includegraphics[scale=0.92]{./Imagens/brdfs/minnaert.pdf}
    \end{center}
\end{figure}

%%%%%%%%%%%%%%%%%%%%%%%%%%%%%%%%%%%%%%%%%%%%%%%%%
\subsection{Código Fonte em \texttt{EquationLang}}
%%%%%%%%%%%%%%%%%%%%%%%%%%%%%%%%%%%%%%%%%%%%%%%%%
\begin{codigo}[H]
    \caption{\small Código fonte da BRDF do experimento Minnaert.}
    \label{cod-minnaert-eqlang}
\begin{lstlisting}[language=tex, frame=none, inputencoding=utf8]
[Min41] MINNAERT M.: The reciprocity principle in lunar photometry. Astrophysical Journal, 3 (1941), 403- 410. 10

$\omega_o$: This is the outgoing (view) direction vector (often normalized).
$\cos\omega_i$ and $\cos\omega_o$: These are actually shorthand notations.

They don't mean the cosine of the entire vector, but rather:
$\cos\omega_i$ actually means $\cos(\theta_i) = \dot(\omega_i, n)$
$\cos\omega_o$ actually means $\cos(\theta_o) = \dot(\omega_o, n)$

Where:

$\theta_i$ is the angle between $\omega_i$ and the surface normal $n$.

$\theta_o$ is the angle between $\omega_o$ and the surface normal $n$.

\begin{equation}
    \rho_{d} = \vec{0.3,0.05,0.05}
\end{equation}

\begin{equation}
k = 0.5
\end{equation}

\begin{equation}
f = \frac{\rho_{d}}{\pi} * ((\vec{n} \cdot \vec \omega_i)*(\vec{n} \cdot \vec \omega_o))^{(k-1)}
\end{equation}
\end{lstlisting}
\end{codigo}

%%%%%%%%%%%%%%%%%%%%%%%%%%%%%%%%%%%%%%%%%%%%%%%%%
\subsection{Código GLSL Gerado}
%%%%%%%%%%%%%%%%%%%%%%%%%%%%%%%%%%%%%%%%%%%%%%%%%
\begin{codigo}[H]
    \caption{\small Saída do compilador: código GLSL da BRDF do experimento Minnaert (parte 1 de 2).}
    \label{cod-minnaert-glsl-pt-1}
\begin{lstlisting}[language=C, inputencoding=utf8]
analytic ::begin parameters
#[type][name][min val][max val][default val]
::end parameters
::begin shader
//////////// START OF BUILTINS DECLARTION ////////////
vec3 var_0_vec_h;
vec3 var_3_vec_n;
float var_10_theta_h;
float var_11_theta_d;
float var_1_pi;
float var_2_epsilon;
vec3 var_4_vec_omega_i;
float var_5_theta_i;
float var_6_phi_i;
vec3 var_7_vec_omega_o;
float var_8_theta_o;
float var_9_phi_o;
//////////// END OF BUILTINS DECLARTION ////////////

//////////// START OF USER DECLARED ////////////
vec3 var_12_rho_d;
float var_13_k;
vec3 var_14_f;
//////////// END OF USER DECLARED ////////////

//////////// START FUNCTIONS DECLARATIONS ////////////
//////////// END FUNCTIONS DECLARATIONS ////////////
\end{lstlisting}
\end{codigo}

\begin{codigo}[H]
    \caption{\small Saída do compilador: código GLSL da BRDF do experimento Minnaert (parte 2 de 2).}
    \label{cod-minnaert-glsl-pt-2}
\begin{lstlisting}[language=C, inputencoding=utf8]
vec3 BRDF(vec3 L, vec3 V, vec3 N, vec3 X, vec3 Y) {

  //////////// START OF BUILTINS INITIALIZATION ////////////
  var_0_vec_h = normalize(L + V);
  var_3_vec_n = normalize(N);
  var_1_pi = 3.141592653589793;
  var_2_epsilon = 1.192092896e-07;
  var_4_vec_omega_i = L;
  var_5_theta_i = atan(var_4_vec_omega_i.y, var_4_vec_omega_i.x);
  var_6_phi_i = atan(sqrt(var_4_vec_omega_i.y * var_4_vec_omega_i.y +
                          var_4_vec_omega_i.x * var_4_vec_omega_i.x),
                     var_4_vec_omega_i.z);
  var_7_vec_omega_o = V;
  var_8_theta_o = atan(var_7_vec_omega_o.y, var_7_vec_omega_o.x);
  var_9_phi_o = atan(sqrt(var_7_vec_omega_o.y * var_7_vec_omega_o.y +
                          var_7_vec_omega_o.x * var_7_vec_omega_o.x),
                     var_7_vec_omega_o.z);
  var_10_theta_h = acos(dot(var_0_vec_h, N));
  var_11_theta_d = acos(dot(var_0_vec_h, var_4_vec_omega_i));
  //////////// END OF BUILTINS INITIALIZATION ////////////

  var_12_rho_d = vec3(0.3, 0.05, 0.05);
  var_13_k = 0.5;
  var_14_f = ((var_12_rho_d / var_1_pi) *
              pow((((dot(var_3_vec_n, var_4_vec_omega_i)) *
                    (dot(var_3_vec_n, var_7_vec_omega_o)))),
                  ((var_13_k - 1.0))));

  return vec3(var_14_f);
}
\end{lstlisting}
\end{codigo}
%%%%%%%%%%%%%%%%%%%%%%%%%%%%%%%%%%%%%%%%%%%%%%%%%
\subsection{Visualização do Resultado}
%%%%%%%%%%%%%%%%%%%%%%%%%%%%%%%%%%%%%%%%%%%%%%%%%
\begin{figure}[H]
    \caption{\small{\textit{Plots} da distribuição de reflexão especular e difusa do experimento Minnaert.}}
    \label{fig-minnaert-plots}
\minipage{0.48\textwidth}
    \vspace{42px}
  \includegraphics[width=\linewidth]{./Imagens/brdfs/minnaert-3D-plot}
    % \vspace{0.1px}
    \legend{ \small (a) 3D \textit{plot}}
\endminipage\hfill
\minipage{0.48\textwidth}
  \includegraphics[width=\linewidth]{./Imagens/brdfs/minnaert-polar-plot.png}
    \legend{ \small (b) \textit{Polar plot}}
\endminipage\hfill
\end{figure}

\begin{figure}[H]
    \caption{\small{Objetos 3D renderizados pelo experimento Minnaert.}}\label{fig-minnaert-eqlang}
\minipage{0.32\textwidth}
  \includegraphics[width=\linewidth]{./Imagens/brdfs/minnaert-teapot.png}
    \legend{ \small (a) \textit{Teapot}}
\endminipage\hfill
\minipage{0.32\textwidth}
  \includegraphics[width=\linewidth]{./Imagens/brdfs/minnaert-dragon.png}
    \legend{ \small (b) Dragão de Stanford}
\endminipage\hfill
\minipage{0.32\textwidth}%
  \includegraphics[width=\linewidth]{./Imagens/brdfs/minnaert-goblin.png}
    \legend{ \small (c) Goblin}
\endminipage
\end{figure}




%%%%%%%%%%%%%%%
%% Conclusao %%
%%%%%%%%%%%%%%%

\chapter{Conclusão}

Este trabalho atinge as tarefas que setamos para fazer, cada pedaço, temos uma serie de teste que incluem não só visualização das BRDFs com uma serie de erros muito bem formatados para informar o usuário, temos testes com varias BRDFs que podemos visualizar em Latex, a linaguem gerada, uma ferramenta disney para visualizar, Realmente iria ajudar mt pessoas na area que não tem conhecimento de compilador ou shjading, isso faciita demais a vida slk. Agora é só esperar.
Entretando poderiamos ter melhores erros com mais contexto ainda, poderiamos aumentar as capacidades do compilador ao permitir mais construções matematicas como somatório através da notação $\Sigma$, poderiamos permitir definição de derivadas e integrais e utilizar algortimos numericos para calcular o valor desses expressões na lingaugem shading.  Apesar de não encontrar essas outras expressões na BRDFs exploradas neste trabalho, podesmos ainda assim aumentar o poder do compilador. Poderiamos ter geração de código para outros tipos de shader, seria um back-end para unity que é uma ferramenta para ciração de gamers onde também é usado para visualizar e eles teem linguagem de shading propria. Poderiamos desenvolver um editor que automaticamente compila seu shader e mostra o resultado no mesmo aplicativo, entre outras melhors. 

@@ Look at other conclusions to be write better @@

Este sistema fornece uma base suficiente para implementação de BRDFs complexas, permitindo que o usuário se concentre na lógica específica do modelo de reflectância enquanto mantém consistência nas transformações de coordenadas e cálculos geométricos fundamentais.

Esta implementação é particularmente relevante para simulações de iluminação física em computação gráfica, onde a precisão nos cálculos de ângulos e vetores é crucial para 
a correta representação do comportamento da luz






% \include{Conteudo/02_Comandos}


% \include{Conteudo/03_ConteudoEspecifico}
% \include{Conteudo/04_Outros}
% \include{Conteudo/05_Customizacao}
% \include{Conteudo/06_Conclusao}


\phantompart
\bibliography{Bibliografia}


%%%%%%%%%%%%%%%%%%%%%%%%%%%%%%%%%%%%%%%%%%%%%%%%%%%%%%
% ELEMENTOS PÓS-TEXTUAIS
%%%%%%%%%%%%%%%%%%%%%%%%%%%%%%%%%%%%%%%%%%%%%%%%%%%%%%


\postextual



\renewcommand{\chapnumfont}{\chaptitlefont}
\renewcommand{\afterchapternum}{}
% \include{Pos_Textual/Apendices}
% \include{Pos_Textual/Anexos}


\end{document}
