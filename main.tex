%%%%%%%%%%%%%%%%%%%%%%%%%%%%%%%%%%%%%%%%%%%%%%%%%%%%%%%%%%%%%%%%%%%%%%%%%%

% abnTeX2: Modelo de Trabalho Acadêmico em conformidade com 
% as normas da ABNT

%%%%%%%%%%%%%%%%%%%%%%%%%%%%%%%%%%%%%%%%%%%%%%%%%%%%%%%%%%%%%%%%%%%%%%%%%%

\documentclass[english, 
               brazil, 
               bsc] %Opções bsc (TCC) e msc (Mestrado)
               {dcomp-abntex2}


%%%%%%%%%%%%%%%%%%%%%%%%%%%%%%%%%%%%%%%%%%%%%%%%%%%%%%%%%%%%%%%%%%%%%%%%%%
% Área para adição de pacotes extras
%%%%%%%%%%%%%%%%%%%%%%%%%%%%%%%%%%%%%%%%%%%%%%%%%%%%%%%%%%%%%%%%%%%%%%%%%%

\usepackage{lipsum} % Retirar para a versão final do documento

%Utilize aqui seu pacote preferido para algoritmos
\usepackage[linesnumbered]{algorithm2e}

%%%%%%%%%%%%%%%%%%%%%%%%%%%%%%%%%%%%%%%%%%%%%%%%%%%%%%%%%%%%%%%%%%%%%%%%%%

%Compila o indice
\makeindex

\begin{document}

% Seleciona o idioma do documento (conforme pacotes do babel)
\selectlanguage{brazil}

% Retira espaço extra obsoleto entre as frases.
\frenchspacing 

%%%%%%%%%%%%%%%%%%%%%%%%%%%%%%%%%%%%%%%%%%%%%%%%%%%%%%%%%%%%%%%%%%%%%%%%%%
% ELEMENTOS PRÉ-TEXTUAIS
%%%%%%%%%%%%%%%%%%%%%%%%%%%%%%%%%%%%%%%%%%%%%%%%%%%%%%%%%%%%%%%%%%%%%%%%%%

\pretextual


\titulo{Desenvolvimento de um Compilador de BRDFs em LaTeX para linguagem de shading GLSL, através da técnica Pratt Parsing } 
\autor{Everton Santos de Andrade Júnior}
\orientador{Beatriz Trinchão Andrade}
\coorientador{Gastao Florencio Miranda Junior}
\curso{Ciência da Computação}

\inserirInformacoesPDF

\imprimircapa
\imprimirfolhaderosto*

% \begin{dedicatoria}
   \vspace*{\fill}
   \centering
   \noindent
   \textit{Esta página foi deixada em branco, não-ironicamente, de propósito} \vspace*{\fill}
\end{dedicatoria}
% ---

% \include{Pre_Textual/Agradecimentos}
% \include{Pre_Textual/Epigrafe}
% % resumo em português
\setlength{\absparsep}{18pt} % ajusta o espaçamento dos parágrafos do resumo
\begin{resumo}
 
% Segundo a \citeonline[3.1-3.2]{NBR6028:2003}, o resumo deve ressaltar o objetivo, o método, os resultados e as conclusões do documento. A ordem e a extensão destes itens dependem do tipo de resumo (informativo ou indicativo) e do tratamento que cada item recebe no documento original. O resumo deve ser precedido da referência do documento, com exceção do resumo inserido no próprio documento. (\ldots) As palavras-chave devem figurar logo abaixo do resumo, antecedidas da expressão Palavras-chave:, separadas entre si por ponto e finalizadas também por ponto.

% O presente trabalho propõe o desenvolvimento de um compilador de funções de distribuição de reflexão bidirecional (BRDFs) expressas em LaTeX para a linguagem de shading GLSL, utilizando a técnica de parsing de Pratt. O objetivo é automatizar o processo de tradução de funções complexas de materiais, frequentemente descritas em LaTeX, para o código GLSL utilizado em programação de shaders para OpenGL. Para isso, será empregada a técnica de parsing de Pratt, uma abordagem eficiente e flexível para analisar e traduzir expressões matemáticas e lógicas. O trabalho incluirá a implementação do compilador, a análise de desempenho e precisão da tradução, e a comparação com métodos tradicionais de tradução manual. Ao final, espera-se oferecer uma ferramenta útil para desenvolvedores e pesquisadores na área de computação gráfica, facilitando a utilização e compreensão de modelos de materiais complexos em aplicações gráficas. Palavras-chave: Compilador, BRDFs, LaTeX, GLSL, Pratt Parsing.

O presente trabalho propõe o desenvolvimento de um compilador de funções de distribuição de reflexão bidirecional (BRDFs) expressas em LaTeX para a linguagem de shading GLSL, utilizando a técnica de parsing de Pratt. O objetivo é automatizar o processo de tradução de funções complexas de materiais, frequentemente descritas em LaTeX, para o código GLSL utilizado em programação de shaders para OpenGL. Ao fornecer essa ferramenta, pretende-se não apenas simplificar o trabalho dos desenvolvedores e pesquisadores na área de computação gráfica, mas também democratizar o acesso e compreensão de modelos de materiais complexos. Além disso, ao permitir que as BRDFs sejam expressas em uma forma mais familiar e acessível, como a notação matemática, o compilador reduz a barreira de entrada para aqueles que não estão familiarizados com linguagens programação. Isso pode facilitar a colaboração interdisciplinar entre profissionais de diferentes áreas, como artistas visuais, designers e cientistas de materiais, que desejam explorar e entender o comportamento visual de materiais em suas aplicações.

 \textbf{Palavras-chave}: Compilador, BRDFs, LaTeX, GLSL, Shading, Pratt Parsing.
\end{resumo}

% \include{Pre_Textual/Abstract}


\mostrarlistadeILUSTRACOES
\mostrarlistadeQUADROS
\mostrarlistadeTABELAS
\mostrarlistadeCODIGOS
\mostrarlistadeALGORITMOS
 
\include{Pre_Textual/Abreviaturas}
\include{Pre_Textual/Simbolos}
    
\mostrarSUMARIO

%%%%%%%%%%%%%%%%%%%%%%%%%%%%%%%%%%%%%%%%%%%%%%%%%%%%%%%%%%%%%%%%%%%%%%%%%%
% ELEMENTOS TEXTUAIS
%%%%%%%%%%%%%%%%%%%%%%%%%%%%%%%%%%%%%%%%%%%%%%%%%%%%%%%%%%%%%%%%%%%%%%%%%%

\textual

%%%%%%%%%%%%%%%%%%%%%%%%%%%%%%%%%%%%%%%%%%%%%%%%%%%%%%%%%%%%%%%%%%%%%%%%%%
% Introdução
%%%%%%%%%%%%%%%%%%%%%%%%%%%%%%%%%%%%%%%%%%%%%%%%%%%%%%%%%%%%%%%%%%%%%%%%%%
\chapter{Introdução}


\section{Contexto}

Na computação gráfica, a representação realista de cenas tridimensionais depende fortemente da modelagem da luz. A interação da luminosidade incidente no objeto, bem como os materiais que compõem esses objetos, são aspectos críticos a serem considerados na geração dessas cenas [referencia]. Na prática, essa interação é frequentemente modelada por meio de funções de distribuição de refletância bidirecional, conhecidas como BRDFs.


As BRDFs, essencialmente, calculam a proporção entre a energia luminosa que atinge um ponto na superfície e como essa energia é refletida, transmitida ou absorvida [referencia]. Na renderização, essas funções são implementadas por meio programas especializados das unidades de processamento gráfico (GPUs), esses programmas são chamados de shaders, e cada API de rederização disponibiliza etapas diferentes onde esses executaveis podem ser mudados durante o processo de renderização. Esses shaders concedem a capacidade de cada objeto renderizado ter sua aparência configurada por meio de um código que implementa uma BRDF.


\section{Motivação}

Apesar da disponibilidade de linguagens específicas para a programação de shaders, que possibilitam a modificação procedimentos que representam uma BRDF, a aplicação de BRDFs na geração de shaders requer conhecimento especializado em programação [referencia?]. Essa barreira técnica pode restringir a exploração dos efeitos visuais por profissionais de áreas não relacionadas à programação. Diante disso, surge a necessidade de ferramentas mais acessíveis para a criação de shaders.

No meio acadêmico, as BRDFs são, comumente, descritas por uma fórmula escrita em LaTeX, uma abordagem promissora para atender a essa necessidade é o desenvolvimento de um compilador capaz de traduzir BRDFs em LaTex para shaders, assim democratizando a visualização dessas BRDFs. Dado que as fórmulas são equações matemáticas, precisamos retrigir repretsentação da linguagem de entrada para o compilador afim de garatir um projeto útil em tempo ábil.

\section{Objetivo}
Este trabalho visa projetar e implementar um compilador que, a partir de funções de distribuição de refletância bidirecional escrita como equações em LaTeX, seja capaz de gerar código de shading na linguagem alvo da API OpenGL (referencia). A saída será um shader capaz de reproduzir as características de reflexão da função de refletância original, considerando a precedencia de operadores, em uma superfície tridimensional, ou, ao menos, alcançar uma aproximação satisfatória dessas características, considerando as limitações da linguagem de shading da API principalmente as representações de dados de forma discreta.

\section{Metodologia}
Para alcançar o objetivo, a sequencia das etapas adotadas serão as seguintes.


\begin{enumerate}

   \item Realizar uma análise abrangente das áreas relacionadas ao desenvolvimento da ferramenta proposta;
   \item Investigar o estado da arte no campo da compilação de BRDFs em linguagens de shading;
   \item Definir a linguagem de entrada e a linguagem de saída do compilador;
   \item Elaborar testes com equações LaTeX de entrada pareado com a saída em shader GLSL esperado;
   \item Implementar o compilador utilizando uma linguagem de programação e tecnicas recursivas de parsing
   \item Realizar a renderização de cenas utilizando o shader gerado pelo compilador.

\end{enumerate}

% Apesar da importância de usar técnicas confiáveis para avaliar um BRDF, há uma falta de trabalhos na literatura que reúnam e comparem essas técnicas.
% Este artigo propõe uma compilação de técnicas usadas para avaliar representações de BRDF, juntamente com suas definições formais. Essas técnicas foram classificadas em três grupos diferentes - funções de comparação, imagens renderizadas e gráficos - e, para ilustrar seu uso, três modelos clássicos e amplamente adotados e uma representação de BRDF de ponta foram avaliados quanto à sua capacidade de preservar a aparência de materiais medidos. Com base em nossa pesquisa sobre funções de comparação, uma técnica de avaliação de BRDF estável e robusta é proposta. Observou-se tanto durante a revisão da literatura quanto nos experimentos que cada grupo de técnicas fornece informações complementares sobre os BRDFs avaliados, o que sugere que pelo menos um modelo de cada categoria deve ser adotado durante a escolha de critérios para avaliar um BRDF.

%%%%%%%%%%%%%%%%%%%%%%%%%%%%%%%%%%%%%%%%%%%%%%%%%%%%%%%%%%%%%%%%%%%%%%%%%%
% Revisão Bibliográfica 
%%%%%%%%%%%%%%%%%%%%%%%%%%%%%%%%%%%%%%%%%%%%%%%%%%%%%%%%%%%%%%%%%%%%%%%%%%
\chapter{Revisão Bibliográfica}

Para esta seção, será conduzida uma revisão literária abrangente com o objetivo de explorar trabalhos relacionados ao desenvolvimento de compiladores para tradução de BRDFs expressas em LaTeX para a linguagem de shading, empregando, técnicas de parsing.
O processo de busca será conduzido em duas etapas distintas. Primeiramente, será realizado um levantamento dos trabalhos existentes nas bases IEEE Xplorer Digital Library, Elsevier Scopus e Google Acadêmico, esse foram escolhidos por serem acessíveis gratuitamente pela afiliação à Universidade Federal de Sergipe, já o google scholar foi escolhido para agregar pesquisas em outras bases que possam ter trabalhos relevantes.

Por fim, será realizada uma busca por produtos ou ferramentas similares no mercado, utilizando strings de busca específicas em repositórios digitais, especificamente GitHub, e GitLab. Esses processos de busca permitirão identificar referências relevantes e estabelecer um panorama do estado da arte no campo dos compiladores de BRDFs para shaders, contribuindo para a compreensão do contexto acadêmico e prático no qual este trabalho se insere.

\subsection{Mapeamento Sistemático}

Foram elaboradas questões de pesquisa específicas e conduzidas buscas em diversas bases de dados, utilizando frases-chave que refletiam os principais aspectos do tema em questão. A partir desse processo de busca criteriosa, foram identificados e selecionados os trabalhos que melhor atendiam às questões propostas, garantindo maior relevância para o estudo em questão.


\subsection{Questões de Pequisa}

\begin{enumerate}
  \item Quais são as abordagens mais comuns utilizadas na criação de compiladores para tradução de BRDFs expressas em alguma linguagem de texto, com LaTeX, para shaders?

  \item Quais as técnicas de parsing que têm sido aplicadas no desenvolvimento de compiladores para linguagens matemáticas como LaTeX?

  \item O trabalho utiliza arvores, ou gramáticas livre de contexto para representar uma BRDF?

 \item Quais são os principais desafios enfrentados ao traduzir funções matemáticas complexas, como as BRDFs, em shaders?

\end{enumerate}

Quais são as ferramentas e recursos disponíveis para auxiliar no desenvolvimento de compiladores para BRDFs e shaders, e como elas podem ser integradas ao processo de desenvolvimento?

\subsection{Descrição dos Trabalhos Relacionados}

\subsection{Pesquisa por Repositórios online}


\chapter{Conceitos}

\section{BRDFs}
% \url{https://www.youtube.com/watch?v=kPIqO929pIc&list=PL2zRqk16wsdpyQNZ6WFlGQtDICpzzQ925&index=3}

\section{Compilador}
\section{Análise Léxica}
\section{Análise Sintática}
\subsection{Pratt Parsing}
\subsubsection{precedencia de expressões}
\section{Análise Sintática}
\section{Pratt Parsing}
\section{Shaders}

%
(("Full Text & Metadata":brdf) OR  ("Full Text & Metadata":reflectance))
AND (("Full Text & Metadata":shader) OR  ("Full Text & Metadata":shading))
AND (("Full Text & Metadata":compiler) OR  ("Full Text & Metadata":parsing) OR  ("Full Text & Metadata":parser) OR  ("Full Text & Metadata":grammar))

%
% ("Full Text & Metadata":brdf)
% AND (("Full Text & Metadata":shader) OR  ("Full Text & Metadata":shading))
% AND (("Full Text & Metadata":compiler) OR  ("Full Text & Metadata":parsing) OR  ("Full Text & Metadata":parser) OR  ("Full Text & Metadata":grammar))
%


% \include{Conteudo/02_Comandos}
% \include{Conteudo/03_ConteudoEspecifico}
% \include{Conteudo/04_Outros}
% \include{Conteudo/05_Customizacao}
% \include{Conteudo/06_Conclusao}

\phantompart
\bibliography{Bibliografia}

%%%%%%%%%%%%%%%%%%%%%%%%%%%%%%%%%%%%%%%%%%%%%%%%%%%%%%%%%%%%%%%%%%%%%%%%%%
% ELEMENTOS PÓS-TEXTUAIS
%%%%%%%%%%%%%%%%%%%%%%%%%%%%%%%%%%%%%%%%%%%%%%%%%%%%%%%%%%%%%%%%%%%%%%%%%%

\postextual

\renewcommand{\chapnumfont}{\chaptitlefont}
\renewcommand{\afterchapternum}{}
\include{Pos_Textual/Apendices}
\include{Pos_Textual/Anexos}

\end{document}
