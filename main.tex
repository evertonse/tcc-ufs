%%%%%%%%%%%%%%%%%%%%%%%%%%%%%%%%%%%%%%%%%%%%%%%%%%%%%%%%%%%%%%%%%%%%%%%%%%

% abnTeX2: Modelo de Trabalho Acadêmico em conformidade com 
% as normas da ABNT

%%%%%%%%%%%%%%%%%%%%%%%%%%%%%%%%%%%%%%%%%%%%%%%%%%%%%%%%%%%%%%%%%%%%%%%%%%

\documentclass[english, 
               brazil, 
               bsc] %Opções bsc (TCC) e msc (Mestrado)
               {dcomp-abntex2}


%%%%%%%%%%%%%%%%%%%%%%%%%%%%%%%%%%%%%%%%%%%%%%%%%%%%%%%%%%%%%%%%%%%%%%%%%%
% Área para adição de pacotes extras
%%%%%%%%%%%%%%%%%%%%%%%%%%%%%%%%%%%%%%%%%%%%%%%%%%%%%%%%%%%%%%%%%%%%%%%%%%

\usepackage{lipsum} %Retirar para a versão final do documento

%Utilize aqui seu pacote preferido para algoritmos
\usepackage[linesnumbered]{algorithm2e}

%%%%%%%%%%%%%%%%%%%%%%%%%%%%%%%%%%%%%%%%%%%%%%%%%%%%%%%%%%%%%%%%%%%%%%%%%%

%Compila o indice
\makeindex

\begin{document}

% Seleciona o idioma do documento (conforme pacotes do babel)
\selectlanguage{brazil}

% Retira espaço extra obsoleto entre as frases.
\frenchspacing 

%%%%%%%%%%%%%%%%%%%%%%%%%%%%%%%%%%%%%%%%%%%%%%%%%%%%%%%%%%%%%%%%%%%%%%%%%%
% ELEMENTOS PRÉ-TEXTUAIS
%%%%%%%%%%%%%%%%%%%%%%%%%%%%%%%%%%%%%%%%%%%%%%%%%%%%%%%%%%%%%%%%%%%%%%%%%%

\pretextual

\titulo{Desenvolvimento de um Compilador de BRDFs em LaTeX para linguagem de shading GLSL, através da técnica Pratt Parsing } 
\autor{Everton Santos de Andrade Júnior}
\orientador{Beatriz Trinchão Andrade}
\coorientador{Gastao Florencio Miranda Junior}
\curso{Ciência da Computação}

\inserirInformacoesPDF

\imprimircapa
\imprimirfolhaderosto*

\begin{dedicatoria}
   \vspace*{\fill}
   \centering
   \noindent
   \textit{Esta página foi deixada em branco, não-ironicamente, de propósito} \vspace*{\fill}
\end{dedicatoria}
% ---

\include{Pre_Textual/Agradecimentos}
\include{Pre_Textual/Epigrafe}
% resumo em português
\setlength{\absparsep}{18pt} % ajusta o espaçamento dos parágrafos do resumo
\begin{resumo}
 
% Segundo a \citeonline[3.1-3.2]{NBR6028:2003}, o resumo deve ressaltar o objetivo, o método, os resultados e as conclusões do documento. A ordem e a extensão destes itens dependem do tipo de resumo (informativo ou indicativo) e do tratamento que cada item recebe no documento original. O resumo deve ser precedido da referência do documento, com exceção do resumo inserido no próprio documento. (\ldots) As palavras-chave devem figurar logo abaixo do resumo, antecedidas da expressão Palavras-chave:, separadas entre si por ponto e finalizadas também por ponto.

% O presente trabalho propõe o desenvolvimento de um compilador de funções de distribuição de reflexão bidirecional (BRDFs) expressas em LaTeX para a linguagem de shading GLSL, utilizando a técnica de parsing de Pratt. O objetivo é automatizar o processo de tradução de funções complexas de materiais, frequentemente descritas em LaTeX, para o código GLSL utilizado em programação de shaders para OpenGL. Para isso, será empregada a técnica de parsing de Pratt, uma abordagem eficiente e flexível para analisar e traduzir expressões matemáticas e lógicas. O trabalho incluirá a implementação do compilador, a análise de desempenho e precisão da tradução, e a comparação com métodos tradicionais de tradução manual. Ao final, espera-se oferecer uma ferramenta útil para desenvolvedores e pesquisadores na área de computação gráfica, facilitando a utilização e compreensão de modelos de materiais complexos em aplicações gráficas. Palavras-chave: Compilador, BRDFs, LaTeX, GLSL, Pratt Parsing.

O presente trabalho propõe o desenvolvimento de um compilador de funções de distribuição de reflexão bidirecional (BRDFs) expressas em LaTeX para a linguagem de shading GLSL, utilizando a técnica de parsing de Pratt. O objetivo é automatizar o processo de tradução de funções complexas de materiais, frequentemente descritas em LaTeX, para o código GLSL utilizado em programação de shaders para OpenGL. Ao fornecer essa ferramenta, pretende-se não apenas simplificar o trabalho dos desenvolvedores e pesquisadores na área de computação gráfica, mas também democratizar o acesso e compreensão de modelos de materiais complexos. Além disso, ao permitir que as BRDFs sejam expressas em uma forma mais familiar e acessível, como a notação matemática, o compilador reduz a barreira de entrada para aqueles que não estão familiarizados com linguagens programação. Isso pode facilitar a colaboração interdisciplinar entre profissionais de diferentes áreas, como artistas visuais, designers e cientistas de materiais, que desejam explorar e entender o comportamento visual de materiais em suas aplicações.

 \textbf{Palavras-chave}: Compilador, BRDFs, LaTeX, GLSL, Shading, Pratt Parsing.
\end{resumo}

\include{Pre_Textual/Abstract}


\mostrarlistadeILUSTRACOES
\mostrarlistadeQUADROS
\mostrarlistadeTABELAS
\mostrarlistadeCODIGOS
\mostrarlistadeALGORITMOS
 
\include{Pre_Textual/Abreviaturas}
\include{Pre_Textual/Simbolos}
    
\mostrarSUMARIO

%%%%%%%%%%%%%%%%%%%%%%%%%%%%%%%%%%%%%%%%%%%%%%%%%%%%%%%%%%%%%%%%%%%%%%%%%%
% ELEMENTOS TEXTUAIS
%%%%%%%%%%%%%%%%%%%%%%%%%%%%%%%%%%%%%%%%%%%%%%%%%%%%%%%%%%%%%%%%%%%%%%%%%%

\textual

\chapter{Introdução}


\section{Contexto}

Na computação gráfica, a representação realista de cenas tridimensionais depende fortemente da modelagem da luz. A interação da luminosidade incidente na matéria, bem como os materiais que compõem esses objetos, são aspectos críticos a serem considerados na geração dessas cenas [referencia]. Na prática, essa interação é frequentemente modelada por meio de funções de distribuição de refletância bidirecional, conhecidas como BRDFs.


As BRDFs, essencialmente, calculam a proporção entre a energia luminosa que atinge um ponto na superfície e como essa energia é refletida, transmitida ou absorvida [referencia]. Na renderização, essas funções são implementadas por meio programas especializados das unidades de processamento gráfico (GPUs), esses programmas são chamados de shaders, e cada API de rederização disponibiliza etapas diferentes onde esses executaveis podem ser mudados durante o processo de renderização. Esses shaders concedem a capacidade de cada objeto renderizado ter sua aparência configurada por meio de um código que implementa uma BRDF.


\section{Motivação}

Apesar da disponibilidade de linguagens específicas para a programação de shaders, que possibilitam a modificação procedimentos que representam uma BRDF, a aplicação de BRDFs na geração de shaders requer conhecimento especializado em programação [referencia?]. Essa barreira técnica pode restringir a exploração das capacidades visuais por profissionais de áreas não relacionadas à programação. Diante disso, surge a necessidade de ferramentas mais acessíveis para a criação de shaders.

Quase universalmente, as BRDFs são descritas por uma fórmula escrita em LaTeX, uma abordagem promissora para atender a essa necessidade é o desenvolvimento de um compilador capaz de traduzir BRDFs em LaTex para shaders, assim democratizando a visualização que essas BRNo entanto, devido à complexidade matemática das BRDFs, é necessário simplificar sua representação para utilizá-las como linguagem de entrada para o compilador.

\section{Objetivo}
Este estudo visa projetar e implementar um compilador que, a partir de funções de distribuição de refletância bidirecional, seja capaz de gerar código de sombreamento na linguagem alvo. O código resultante deverá reproduzir as características de reflexão da função de refletância original em uma superfície tridimensional, ou, no mínimo, alcançar uma aproximação satisfatória dessas características, conforme as capacidades da linguagem de sombreamento alvo.

Para alcançar o objetivo geral, as seguintes etapas serão realizadas:

Realizar uma análise abrangente das áreas relacionadas ao desenvolvimento da ferramenta proposta;
Investigar o estado da arte no campo da compilação de BRDFs em linguagens de sombreamento;
Definir a linguagem de entrada e a linguagem de saída do compilador;
Selecionar as ferramentas mais adequadas para o desenvolvimento do compilador;
Elaborar exemplos de código de entrada;
Implementar o compilador utilizando as ferramentas selecionadas;
Testar o compilador com uma variedade de exemplos de código de entrada;
Realizar a renderização de cenas utilizando o código de sombreamento gerado pelo compilador.

@Fix

\section{Metodologia}


% Apesar da importância de usar técnicas confiáveis para avaliar um BRDF, há uma falta de trabalhos na literatura que reúnam e comparem essas técnicas.
% Este artigo propõe uma compilação de técnicas usadas para avaliar representações de BRDF, juntamente com suas definições formais. Essas técnicas foram classificadas em três grupos diferentes - funções de comparação, imagens renderizadas e gráficos - e, para ilustrar seu uso, três modelos clássicos e amplamente adotados e uma representação de BRDF de ponta foram avaliados quanto à sua capacidade de preservar a aparência de materiais medidos. Com base em nossa pesquisa sobre funções de comparação, uma técnica de avaliação de BRDF estável e robusta é proposta. Observou-se tanto durante a revisão da literatura quanto nos experimentos que cada grupo de técnicas fornece informações complementares sobre os BRDFs avaliados, o que sugere que pelo menos um modelo de cada categoria deve ser adotado durante a escolha de critérios para avaliar um BRDF.


% \include{Conteudo/02_Comandos}
% \include{Conteudo/03_ConteudoEspecifico}
% \include{Conteudo/04_Outros}
% \include{Conteudo/05_Customizacao}
% \include{Conteudo/06_Conclusao}

\phantompart
\bibliography{Bibliografia}

%%%%%%%%%%%%%%%%%%%%%%%%%%%%%%%%%%%%%%%%%%%%%%%%%%%%%%%%%%%%%%%%%%%%%%%%%%
% ELEMENTOS PÓS-TEXTUAIS
%%%%%%%%%%%%%%%%%%%%%%%%%%%%%%%%%%%%%%%%%%%%%%%%%%%%%%%%%%%%%%%%%%%%%%%%%%

\postextual

\renewcommand{\chapnumfont}{\chaptitlefont}
\renewcommand{\afterchapternum}{}
\include{Pos_Textual/Apendices}
\include{Pos_Textual/Anexos}

\end{document}
