

\section{Plano de Continuação} \label{continuacao}


A continuação deste trabalho envolve várias tarefas-chave destinadas a completar o desenvolvimento do compilador proposto para converter equações \LaTeX  que descrevem BRDFs em código de \textit{shader} GLSL. As tarefas incluem: atualizar o \textit{lexer} e o \textit{parser} escritos em Odin para aceitar equações \LaTeX; testar o \textit{lexer} para garantir o reconhecimento correto dos \textit{tokens}; testar o \textit{parser} para garantir que a árvore sintática está com precedência correta; definir símbolos predefinidos e constantes matemáticas; implementar o processo de geração de código GLSL usando a árvore sintática com o padrão de \textit{design} visitante (\textit{Visitor}); expandir os casos de teste para cobrir uma melhor variedade de BRDFs; testar o código gerado quanto à correção e eficácia, incluindo as visualizações das BRDFs em algumas cenas; preparar a apresentação final; escrever o documento do trabalho de conclusão (TCC). Cada etapa e seus períodos de execução são ilustrados na \autoref{chart} e listados a seguir:


\begin{enumerate}
\item 06/05/2024 - 10/06/2024: Atualizar o \textit{lexer} e o \textit{parser} para aceitar equações \LaTeX.
\item 10/05/2024 - 18/05/2024: Testar o \textit{lexer} para garantir o reconhecimento correto de todos os \textit{tokens} \LaTeX .
\item 18/05/2024 - 10/06/2024: Testar o \textit{parser} para garantir a árvore de sintaxe com a precedência correta.
\item 10/06/2024 - 25/06/2024: Definir símbolos pré-definidos como constantes matemáticas e outras quantidades.
\item 25/06/2024 - 24/08/2024: Implementar o processo de geração de código GLSL.
\item 10/07/2024 - 17/07/2024: Expandir os casos de teste para cobrir uma ampla gama de BRDFs.
\item 17/07/2024 - 24/08/2024: Testar o código gerado quanto à correção e eficácia, tanto em código quanto em visualização e corrigir se necessário.
\item 01/06/2024 - 06/09/2024: Escrever o documento da tese.
\item 25/08/2024 - 06/09/2024: Preparar a apresentação final.
\end{enumerate}


% \begin{adjustwidth}{-1.3cm}{-1cm}
\begin{landscape}
\begin{figure}[htb]
\caption{\label{chart}\small Chart das tarefas previstas. }
\begin{center}
\begin{ganttchart}[
vgrid,
hgrid,
x unit=1.12mm,
time slot format=isodate,
% time slot format=isodate-yearmonth,
bar label font=\footnotesize,
group label font=\footnotesize,
milestone label font=\footnotesize,
]{2024-05-06}{2024-9-10}
    \gantttitlecalendar{year, month=shortname} \\
    \ganttbar{1. Atualizar lexer e parser}{2024-05-06}{2024-06-10} \\
    \ganttbar{2. Testar lexer para todos os tokens}{2024-05-10}{2024-05-18} \\
    \ganttbar{3. Testar parser para precedência}{2024-05-18}{2024-06-10} \\
    \ganttbar{4. Definir símbolos matemáticos}{2024-06-10}{2024-6-25} \\
    \ganttbar{5. Implementar geração de GLSL}{2024-06-25}{2024-08-24} \\
    \ganttbar{6. Expandir casos de teste para GLSL}{2024-07-10}{2024-07-17} \\
    \ganttbar{7. Testar correção do código}{2024-07-17}{2024-08-24} \\
    \ganttbar{8. Escrita do trabalho}{2024-06-01}{2024-09-06} \\
    \ganttbar{9. Preparar apresentação final}{2024-08-25}{2024-09-06}
\end{ganttchart}
\end{center}
\end{figure}
\end{landscape}
