\chapter{Resultados}
\label{chapter.resultados}


Este capítulo apresenta os resultados dos experimentos com BRDFs distintas. Esses experimentos servem de validação e visualização da capacidade do compilador desenvolvido neste trabalho. Cada BRDF escolhida teve sua excolha feita com foco em explorar diferentes expressões matemáticas com diferentes niveis de complexidades;aspectos importantes para esse projeto.

Todos os experimentos seguem uma ordem para apresentação de todos os experimentos. Primeiro, apresentamos a BRDF do experimento, incluindo a referencia e, para as mais imporantes, uma breve explicação sobre ela. Depois mostrandos o código-fonte descreve a BRDF em \texttt{EquationLang}, jutamente com a sua representação em PDF \LaTeX. Traduzimos o código fonte para GLSL usando o compilador desenvolvido neste trabalho. Por fim, utilizamos o código em linguagem shading gerado para ser carregado na ferramenta BRDF Disney. Mostramos o grafico 3D e 2D da distribuição de reflexão especular e difusa da BRDF similar aos representados na \autoref{@@}. Para demonstrar a eficácia do GLSL gerado, mostramos a renderização de três objeto tridimensionais com iluminação provida pelo código gerado para BRDF em questão através de tecnica de \textit{raytracing} fornecido pela ferramenta Disney, ( breve explicação sobre raytracing pode ser visto em \autoref{apendice}). Cada um dos três objeto possuem os angulos em cordenadas polares fixas. Todos as imagems do experimento seguem o formato da \autoref{@@}; da esquerda para direita os três objeto tem as dupla angulo de elevação ($\theta_i$)  e angulo azimutal ($\phi_i$) da luz incidente: $\left(33,8941, 145,826\right)$, $\left( ,\right)$, $\left( ,\right)$, respectivamente. Gamma e exposição também são fixadas em $2,112$ e $-1,248$ respectivamente. Adicionalmente mostramos o efeito da BRDF em uma esfera com renderização projetiva padrão para observar a iluminação em um objeto simples.
Deve-se dizer que o gráfico polar e 3D da distribuição de reflexão é reference a todas os três canais de cores ao mesmo tempo, então pode ter overlap entre as cores vermelho azul e verde na visualização pois a distruição de cada um desses canais podem ser os mesmos em um dado experimento.



Apesar de conter uma breve explicação sobre a BRDF, o mais importante é ver o código gerado e seu funcionamento na ferramenta Disney, pois o foco principal permanece no compilador e sua representação fidedigna à BRDFs descrita.  Vale ressaltaar que o código é gerado pelo computador e nbão é muito legivel para o humano, se comprar a código shading escrito a mão, então incluimos o GLSL gerada para fins de desmontração e completude, mas não necessariamente para leitura. Também, o código gerado pode ser longo e divido em duas partes, então recomenda-se olhar rapidamente para adquirir uma noção da forma em qual o código é gerado e se torna mais produtivo pular para o a imagem renderizada pelo código gerado ou para proximo experimento.

Existem mais de uma maneira de expressar as BRDF. Parte dos resultados é realizar experimentos de versões diferentes da mesma BRDF, com não só parametros diferentes mas também expressões matemáticas diferente para expressa-la. Sendo asism, provemos duas versões para algumas dos experimentos.


Por conveniencia, deixamos a tabela \autoref{@@ table} para navegar rapidamente cada imagem e código de todos os experimentos, e  do mesmos.


@TABLE@
% Esses experimentos, proporcionando uma abordagem que explora tanto as BRDFs quanto relacionados ao desenvolvimento de compiladores e à aplicação de conceitos como BRDFs


\section{Opnião}
Os resultados são satisfátórios, captura nuances importantes das BRDFs, mesmo em materiais com estruturas que usam X e Y @@@. Se mostrou capaz de permitir várias parametrização baseada em nas equações e, com o nivel atual o compilador, permite modelar uma ampla gama de comportamentos de superfície.
