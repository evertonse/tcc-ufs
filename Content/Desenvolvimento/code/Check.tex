\begin{codigo}[H]
    \caption{\small Recorte da função \texttt{check\_expr}. }
    \label{eq-function-check-expr}
\begin{lstlisting}[language=C, frame=none, inputencoding=utf8]
check_expr :: proc(expr: ^ast.Expr) {
    // Primeiro inferimos o tipo
    infer_type(expr)
    // Código omitido de preambulo
    // Partimos checar os casos para cada tipo de expressão
        case ^Expr_Identifier:
            // Check for using undefined indetifiers
            identifier_key := key_from_identifier(e)
            if !is_defined(e, false) {
                error(e.identifier,  "Identifier `%v` is not defined in the current scope.", identifier_key)
            }

        case ^Expr_Grouped:
            // Basta vallidar o corpo da expressão agrupada
            check_expr(e.expr)

        case ^Expr_Prefix:
            // Validamos o lado dereito da expressão
            check_expr(e.right)
            if e.op.kind == .Sqrt {
                if !is_type_equal(e.right.ty_inferred, ty_number) {
                    error(e.op,  "Square root can only accept Numbers but instead got `%v`.", format_type(e.right.ty_inferred))
                }
            }
            // Código omitido ...
        case ^Expr_Infix:
            // Validamos dois lados da expressão binária e
            // checamos se são compativeis entre si
            check_expr(e.right)
            check_expr(e.left)
            // Código omitido ...

        case ^Expr_Vector_Literal:
            for number in e.numbers {
                check_expr(number)
            }

        case ^Expr_Number:
            // base case
        // Outros casos omitidos
    }
}

\end{lstlisting}
\end{codigo}
