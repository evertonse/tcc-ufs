\begin{codigo}[H]
    \caption{\small Recorte da função \texttt{check\_expr}. }
    \label{eq-function-check-expr}
\begin{lstlisting}[language=C, basicstyle=\ttfamily\footnotesize, frame=none, inputencoding=utf8]
check_expr :: proc(expr: ^ast.Expr) {
    // Primeiro inferimos o tipo
    infer_type(expr)
    // Código omitido de preambulo
        case ^Expr_Identifier:
            // Check for using undefined indetifiers
            identifier_key := key_from_identifier(e)
            if !is_defined(e, false) {
                error(e.identifier,  "Identifier `%v` is not defined in the current scope.", identifier_key)
            }
        case ^Expr_Function_Call:
            check_expr(e.left)
            fn_ident, fn_ident_ok := e.left.expr_derived.(^ast.Expr_Identifier)
            if !fn_ident_ok {
                error(e.open,  "Tried to call an expression that is not an identifier")
            }
            fn_string := key_from_identifier(fn_ident)
            fn_sym, fn_sym_ok := scope_get(fn_string)
            fn_type, fn_type_ok := e.left.ty_inferred.derived.(^ast.Type_Function)
            if !fn_type_ok {
                error(e.open,  "Tried to call`%v`, which is not a function. Its type is `%v`", fn_string, format_type(e.left.ty_inferred))
            }
            if len(e.exprs) != len(fn_type.params) {
                error(
                    e.open,  "Args number mismatch. `%v` function expects  `%v` arguments but `%v` were given.",
                    fn_string, len(fn_type.params), len(e.exprs)
                )
            }
        // Outros casos omitidos
    }
}

\end{lstlisting}
\end{codigo}
