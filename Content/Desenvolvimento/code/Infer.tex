\begin{codigo}[htb]
    \caption{\small Parte do switch da inferencia de tipos. }
    \label{cod-type-inference}
\begin{lstlisting}[language=C, frame=none, inputencoding=utf8]
infer_type :: proc(expr: ^ast.Expr, allow_invalid := false, default : ^ast.Type = ast.ty_invalid ) -> ^Type {
    /// Código omitido por brevidade  ///
    #partial switch e in expr.derived {
    case ^Expr_Identifier:
        // Alway set by the parser if indentifier starts with `\vec{}`
        if e.is_vector {
            type = new_type_vector(ty_number, 3) // vector 3 of type number (real)
        } else if e.identifier.kind == .Pi || e.identifier.kind == .Epsilon {
            type = ty_number
        } else {
            key := key_from_identifier(e)
            sym, ok := scope_get(key)
            if ok {
                if sym.type != nil {
                    type = sym.type
                }
            }
            /// Código omitido por brevidade  ///
        }

    case ^Expr_Prefix:
        right_type := infer_type(e.right, allow_invalid, default)
        /// Código omitido, mas aqui fazemos a validação da subexpressão direita e atribuimos o tipo correto para Expr_Prefix   ///

    case ^Expr_Infix:
        ty_left  := infer_type(e.left, allow_invalid, default)
        ty_right := infer_type(e.right, allow_invalid, default)
        /// Código omitido
        //  Inferimos tipo da expressão esquerda e direto dessa operação binária
        //  Depois validamos compatibilidade considerando a operação sendo usada nessas duas expressões

    /// Outros casos ...  ///
}
\end{lstlisting}
\end{codigo}
