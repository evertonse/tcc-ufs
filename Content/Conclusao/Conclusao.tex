
\chapter{Conclusão}

Este trabalho atinge as tarefas que setamos para fazer, cada pedaço, temos uma serie de teste que incluem não só visualização das BRDFs com uma serie de erros muito bem formatados para informar o usuário, temos testes com varias BRDFs que podemos visualizar em Latex, a linaguem gerada, uma ferramenta disney para visualizar, Realmente iria ajudar mt pessoas na area que não tem conhecimento de compilador ou shjading, isso faciita demais a vida slk. Agora é só esperar.
Entretando poderiamos ter melhores erros com mais contexto ainda, poderiamos aumentar as capacidades do compilador ao permitir mais construções matematicas como somatório através da notação $\Sigma$, poderiamos permitir definição de derivadas e integrais e utilizar algortimos numericos para calcular o valor desses expressões na lingaugem shading.  Apesar de não encontrar essas outras expressões na BRDFs exploradas neste trabalho, podesmos ainda assim aumentar o poder do compilador. Poderiamos ter geração de código para outros tipos de shader, seria um back-end para unity que é uma ferramenta para ciração de gamers onde também é usado para visualizar e eles teem linguagem de shading propria. Poderiamos desenvolver um editor que automaticamente compila seu shader e mostra o resultado no mesmo aplicativo, entre outras melhors. 

@@ Look at other conclusions to be write better @@

Este sistema fornece uma base suficiente para implementação de BRDFs complexas, permitindo que o usuário se concentre na lógica específica do modelo de reflectância enquanto mantém consistência nas transformações de coordenadas e cálculos geométricos fundamentais.

Esta implementação é particularmente relevante para simulações de iluminação física em computação gráfica, onde a precisão nos cálculos de ângulos e vetores é crucial para 
a correta representação do comportamento da luz
