
\chapter{Conclusão} \label{chapter-conclusion}

Este trabalho alcançou o seu objetivo, que é desenvolver um compilador capaz de traduzir funções de distribuição de refletância bidirecional para código de linguagem de \textit{shading}. Para renderizar cenas de alta qualidade, são essenciais os \textit{shaders}, componentes intrínsecos à pipeline gráfica, conforme discutido na \autoref{shading}. Esses \textit{shaders} podem conter a implementação de BRDFs que determinam a maneira como a luz interage com os materiais. Entretanto, a complexidade das BRDFs em suas equações, bem como a necessidade de conhecimentos na área de programação com \textit{shaders}, tornam desafiadora a tradução de BRDFs para código.

A ferramenta reduz significativamente a barreira técnica que poderia restringir a exploração de efeitos visuais por profissionais fora da área de programação, pois fornece um sistema que transforma um documento \LaTeX{} contendo equações de BRDF diretamente em um arquivo GLSL, pronto para ser carregado no visualizador de BRDFs gratuito da Disney. Essa funcionalidade atende principalmente ao meio acadêmico, onde as BRDFs são frequentemente descritas por equações em \LaTeX{}, democratizando o acesso à criação de efeitos visuais complexos.

O projeto faz uso de conhecimentos técnicos de áreas multidisciplinares, como gramáticas livres de contexto, geração de código, algortimos com árvores e grafos, renderização projetiva e por \textit{raytracing}, programação com \textit{shaders}, além de fundamentos teóricos sobre refletância e conceitos de radiometria para implementar os recursos da ferramenta. Esses recursos incluem: mensagens de erro bem estruturadas e informativas; suporte a definições de equação em qualquer ordem; geração de código para diferentes BRDFs diretamente a partir de equações \LaTeX{}; integração com a ferramenta Disney para renderização visual das BRDFs; e visualização da árvore sintática gerada como arquivo SVG. Todas as etapas do compilador, apresentadas na arquitetura da \autoref{fig-estrutura-geral-compilador}, foram implementadas, incluindo análise léxica, sintática, semântica e emissão de código.

O compilador mantém a consistência na ordem dos operadores, permitindo o uso de convenções como símbolos matemáticos comuns ($\pi$, $\epsilon$) e outros símbolos específicos da área, como $\omega_i$ e $\omega_o$. Também suporta expressões matemáticas essenciais para simulações de iluminação realista em computação gráfica, como produto interno, produto vetorial, funções trigonométricas, exponenciação, entre outras, permitindo cálculos com ângulos e vetores. Esses detalhes são cruciais para representar fielmente o comportamento da luz.

% Dessa forma, a ferramenta promove uma experiência simplificada para visualização de BRDFs.

A partir dos experimentos do \autoref{chapter.resultados}, o sistema demonstrou que provê uma base sólida para a implementação de BRDFs complexas a partir de suas equações. Isso confirma que usuários podem focar na lógica específica de modelagem de refletância, sem precisar lidar com detalhes técnicos de baixo nível, como aspectos específicos da linguagem de \textit{shading}. Embora o sistema seja funcional e os experimentos tenham sido bem-sucedidos, existem oportunidades para melhorias e expansão. Algumas direções promissoras são indicadas nos seguintes items:

\begin{itemize}
    \item Ampliar o suporte para construções matemáticas adicionais, como somatórios ($\Sigma$) e produtos acumulados ($\Pi$);
    \label{items-melhorias}
    \item Adicionar funcionalidades para definição e cálculo de derivadas e integrais, utilizando algoritmos numéricos para avaliar essas expressões diretamente na linguagem de \textit{shading};
    \item Expandir as capacidades para suportar diferentes linguagens de \textit{shading}, como as utilizadas nos motores gráficos Unity\footnote{\url{https://unity.com/}} e Unreal\footnote{\url{https://www.unrealengine.com/en-US}};
    \item Desenvolver um editor de texto \LaTeX{} integrado, que permita a compilação e visualização simultâneas da BRDFs.
\end{itemize}

Ademais, aprimoramentos no tratamento de erros podem proporcionar maior contextualização e clareza, auxiliando os usuários na resolução de problemas. Embora não tenham sido identificadas nas BRDFs exploradas certas construções matemáticas (como notação $\Pi$ e $\Sigma$) ou integrais não analíticas - o que impulsionaria a necessidade de um algoritmo numérico para resolução -, a implementação dessas funcionalidades ampliaria significativamente o potencial do compilador.

As perspectivas futuras deste sistema apontam para uma ferramenta cada vez mais versátil e acessível, com potencial para transformar a forma como desenvolvedores e pesquisadores trabalham com BRDFs. A democratização do acesso a técnicas avançadas de computação gráfica representa uma oportunidade para facilitar e agilizar a modelagem dessas funções em simulações científicas.
