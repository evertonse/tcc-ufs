\chapter{Resultados}
\label{chapter.resultados}


Este capítulo apresenta os resultados dos experimentos com diferentes BRDFs, servindo como validação e visualização da capacidade do compilador desenvolvido. A escolha de cada BRDF foi direcionada para explorar expressões matemáticas com diversos níveis de complexidade, aspectos importante para testar a capacidade do compilador desenvolvido neste projeto.

Os experimentos seguem uma metodologia padronizada. Inicialmente, apresenta-se a BRDF do experimento, incluindo sua referência, quando relevante e uma breve explicação conceitual. Em seguida, demonstra-se o código fonte que descreve a BRDF em \texttt{EquationLang}, acompanhado de sua representação em PDF \LaTeX{}. Utilizando o compilador desenvolvido, o código fonte é traduzido para linguagem de \textit{shading} GLSL e carregado na ferramenta Disney BRDF Explorer.

A análise inclui gráficos 3D e 2D da distribuição de reflexão especular e difusa da BRDF. Para demonstrar a eficácia do código GLSL gerado, são renderizados três objetos tridimensionais utilizando técnicas de \textit{ray tracing} fornecidas pela ferramenta Disney. Os objetos possuem ângulos em coordenadas polares fixas, com condições padronizadas de iluminação: ângulos de elevação ($\theta_i$) e azimutal ($\phi_i$) da luz incidente predefinidos em $33,8941$ e $145,826$ respectivamente; gamma fixado em $2,112$ e exposição em $-1,248$.
% Adicionalmente, apresenta-se o efeito da BRDF em uma esfera com renderização projetiva padrão.

O \textit{plot}\footnote{O termo plot é comumente usado em ferramentas de visualização e análise para se referir a gráficos ou representações visuais de dados. Aqui, refere-se a representações 2D (polar) ou 3D usadas para ilustrar os componenetes especular e difusos de cada canal de cor da BRDF.} 3D de BRDFs na ferramenta Disney Explorer oferece uma visualização que fixa uma direção de luz incidente ($\omega_i$) e coleta valores da direções de visualização ($\omega_o$) em um hemisfério como entrada para BRDF. Para cada direção de visualização, renderiza-se um primitivo proporcional ao valor da função BRDF.

O \textit{plot} polar, por sua vez, representa um corte bidimensional, fixando a direção de luz incidente ($\omega_i$) e o ângulo azimutal de saída ($\phi_o$), variando apenas o ângulo polar de saída ($\theta_o$), similar ao mostrado nas figuras da \autoref{brdfmodels}. Cada ponto representa o valor médio das componentes da BRDF, visualizando o comportamento da refletância em diferentes ângulos de observação. Em alguns casos, fatores logarítmicos são utilizados para melhor visualização do gráfico.

É importante observar que os gráficos polares e 3D representam simultaneamente os três canais de cores, como na \autoref{fig-ashikhmin-shirley-alternative-plots}, podendo haver sobreposição entre vermelho, azul e verde na visualização, já que a distribuição de cada canal pode ser idêntica em um dado experimento.

% Embora os experimentos contenha uma explicação sobre a BRDF, o foco principal permanece na representação fidedigna das equações em GLSL provida pelo compilador. 
% E Ainda, toda explicação que esteja fora do ambiente equatio é ignorada pelo compilaodr, então realmente é apensar para ilustrar a compilação de um arquivo por completo, por isso incluimos as explicações rudimentares ( muitos vezes em inles) para demonstrar isso. Recomenda-se observar rapidamente o código gerado para compreender sua estrutura, reconhecendo que o código GLSL gerado por computador não é tão legível quanto código \textit{shading} escrito manualmente.

Embora os experimentos contenham uma explicação sobre a BRDF, o foco principal permanece na representação fidedigna das equações em GLSL fornecida pelo compilador. Além disso, qualquer explicação presente no arquivo de entrada (\verb".tex") que esteja fora do ambiente \texttt{equation} é ignorada pelo compilador, sendo incluída apenas para ilustrar a compilação de um arquivo completo. Por essa razão, inserimos explicações rudimentares (muitas vezes em inglês) para demonstrar esse aspecto. Recomenda-se observar rapidamente o código gerado para compreender sua estrutura, reconhecendo que o código GLSL gerado pelo computador não é tão legível quanto o código \textit{shading} escrito manualmente.

Alguns experimentos exploram múltiplas formas de expressar a mesma BRDF, não apenas com parâmetros distintos, mas também com expressões matemáticas alternativas. Para facilitar a navegação, a \autoref{table-experiments} é disponibilizada para acesso rápido às imagens e códigos dos experimentos.



\begin{table}[H]
\centering
\begin{tabular}{|c|c|c|c|c|}
\hline
    \textbf{Experimento} & \textbf{Equações}                                                & \textbf{Objetos 3D}                                       & \textbf{\textit{Plots}}                                  & \textbf{GLSL}  \\ \hline
    Blinn-Phong          & \autoref{fig-blinn-phong-eqlang-latex}                           & \autoref{fig-blinn-phong-eqlang}                          & \autoref{fig-blinn-phong-plots}                          &                  \autoref{cod-blinn-phong-glsl-pt-1}              \\ \hline
    Cook-Torrance        & \autoref{fig-cook-torrance-eqlang-latex}                         & \autoref{fig-cook-torrance-eqlang}                        & \autoref{fig-cook-torrance-plots}                        &             \autoref{cod-cook-torrance-glsl-pt-1}              \\ \hline
    Ward                 & \autoref{fig-ward-eqlang-latex}                                  & \autoref{fig-ward-objetcs}                                & \autoref{fig-ward-plots}                                 &                     \autoref{cod-ward-glsl-pt-1}               \\ \hline
    Ashikhmin-Shirley    & \autoref{fig-ashikhmin-shirley-close-to-original-eqlang-latex}   & \autoref{fig-ashikhmin-shirley-close-to-original-eqlang}  & \autoref{fig-ashikhmin-shirley-close-to-original-plots}  &                  \autoref{cod-ashikhmin-shirley-close-to-original-glsl-pt-1}              \\ \hline
    Oren-Nayar           & \autoref{fig-oren-nayar-eqlang-latex}                            & \autoref{fig-oren-nayar-eqlang}                           & \autoref{fig-oren-nayar-plots}                           &                \autoref{cod-oren-nayar-glsl-pt-1}              \\ \hline
    Ashikhmin-Shirley$_2$& \autoref{fig-ashikhmin-shirley-alternative-eqlang-latex}         & \autoref{fig-ashikhmin-shirley-alternative-eqlang}        & \autoref{fig-ashikhmin-shirley-alternative-plots}        &                  \autoref{cod-ashikhmin-shirley-alternative-glsl-pt-1}              \\ \hline
    Cook-torrance$_2$    & \autoref{fig-cook-torrance-alternative-eqlang-latex}             & \autoref{fig-cook-torrance-alternative-eqlang}            & \autoref{fig-cook-torrance-alternative-plots}            & \autoref{cod-cook-torrance-alternative-glsl-pt-1}              \\ \hline
    Dür                  & \autoref{fig-duer-eqlang-latex}                                  & \autoref{fig-duer-eqlang}                                 & \autoref{fig-duer-plots}                                 &                      \autoref{cod-duer-glsl-pt-1}              \\ \hline
    Edwards-2006         & \autoref{fig-edwards-2006-eqlang-latex}                          & \autoref{fig-edwards-2006-eqlang}                         & \autoref{fig-edwards-2006-plots}                         &              \autoref{cod-edwards-2006-glsl-pt-1}              \\ \hline
    Kajiya-Kay-1989$_*$  & \autoref{fig-kajiya-eqlang-latex}                                & \autoref{fig-kajiya-objects}                              & \autoref{fig-kajiya-plots}                               &                  \autoref{cod-kajiya-glsl-pt-1}              \\ \hline
    Minnaert             & \autoref{fig-minnaert-eqlang-latex}                              & \autoref{fig-minnaert-eqlang}                             & \autoref{fig-minnaert-plots}                             &                  \autoref{cod-minnaert-glsl-pt-1}              \\ \hline
\end{tabular}
\caption{Tabela dos Experimentos}
\label{table-experiments}
\end{table}


%%%
Concluímos que os experimentos realizados apresentaram resultados satisfatórios. O compilador desenvolvido demonstra flexibilidade ao capturar as nuances das diferentes BRDFs, inclusive em materiais com estruturas complexas. O sistema permite diversas parametrizações e equações alternativas para representar os comportamentos da superfície.

Os resultados obtidos não apenas validam a abordagem metodológica adotada, mas também abrem perspectivas para futuras extensões e refinamentos da ferramenta. Após o último experimento, seguimos diretamente para o capítulo de conclusão (\autoref{chapter-conclusion}), onde são discutidas as possíveis direções para a continuidade deste trabalho.

%%%%%%%%%%%ABOVE VALIDADED%%%%%%%%%%%%%%%
%%%%%%%%%%%%%%%%%%%%%%%%%%%%%%%%%%%%%%%%%

