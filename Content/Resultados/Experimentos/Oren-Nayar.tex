
\section{Experimento BRDF Oren-Nayar} \label{section-experiment-oren-nayar}

Este experimento é baseado no trabalho de Oren e Nayar \cite{oren1994generalization}, no qual propuseram uma generalização do modelo de refletância de Lambert, desenvolvendo uma modelagem matemática que oferece maior precisão na difusão de luz em superfícies rugosas. As equações que descrevem esse experimento se encontram na \autoref{fig-oren-nayar-eqlang-latex}. O código-fonte em \texttt{EquationLang}, usado como entrada para o compilador, está dividido em duas partes: a primeira no \autoref{cod-oren-nayar-eqlang-pt-1} e a segunda no \autoref{cod-oren-nayar-eqlang-pt-2}. A renderização dos objetos 3D utilizando essa BRDF está na \autoref{fig-oren-nayar-eqlang}, e os \textit{plots} correspondentes estão na \autoref{fig-oren-nayar-plots}. A saída em GLSL está dividida em três partes: a parte 1 no \autoref{cod-oren-nayar-glsl-pt-1}, a parte 2 no \autoref{cod-oren-nayar-glsl-pt-2}, e a parte 3 no \autoref{cod-oren-nayar-glsl-pt-3}.


%%%%%%%%%%%%%%%%%%%%%%%%%%%%%%%%%%%%%%%%%%%%%%%%%
\subsection{Representação em documento \LaTeX{}}
%%%%%%%%%%%%%%%%%%%%%%%%%%%%%%%%%%%%%%%%%%%%%%%%%
\begin{figure}[H]
    \caption{\label{fig-oren-nayar-eqlang-latex}
    \small Equações da BRDF do experimento Oren-Nayar em documento \LaTeX{}.}
    \begin{center}
        \includegraphics[scale=0.82]{./Imagens/brdfs/oren-nayar.pdf}
    \end{center}
\end{figure}

%%%%%%%%%%%%%%%%%%%%%%%%%%%%%%%%%%%%%%%%%%%%%%%%%
\subsection{Código Fonte em \texttt{EquationLang}}
%%%%%%%%%%%%%%%%%%%%%%%%%%%%%%%%%%%%%%%%%%%%%%%%%
\begin{codigo}[H]
    \caption{\small Código fonte da BRDF do experimento Oren-Nayar (parte 1 de 2).}
    \label{cod-oren-nayar-eqlang-pt-1}
\begin{lstlisting}[language=tex, frame=none, inputencoding=utf8]
\begin{equation}
    \rho = 0.9
\end{equation}
\begin{equation}
    \sigma = 30*\pi/180
\end{equation}
\begin{equation}
    \theta_r = \arccos((V \cdot N))
\end{equation}
\begin{equation}
    \phi_{\text{diff}} = ( \text{normalize}(V-N*((V \cdot N))) \cdot \text{normalize}(L - N*((L \cdot N))) )
\end{equation}
\begin{equation}
    N = \vec n
\end{equation}

\begin{equation}
    L = \vec \omega_i
\end{equation}

\begin{equation}
    V = \vec \omega_o
\end{equation}

\begin{equation}
    \theta_i = \arccos ((L \cdot N))
\end{equation}

\begin{equation}
    \alpha = \max (\theta_i, \theta_r)
\end{equation}

\begin{equation}
    \beta = \min (\theta_i, \theta_r)
\end{equation}

\begin{equation}
    C_1 = 1 - 0.5 * (\sigma^2) / ((\sigma^2) + 0.33)
\end{equation}

\end{lstlisting}
\end{codigo}

\begin{codigo}[H]
    \caption{\small Código fonte da BRDF do experimento Oren-Nayar (parte 2 de 2).}
    \label{cod-oren-nayar-eqlang-pt-2}
\begin{lstlisting}[language=tex, frame=none, inputencoding=utf8]
\begin{equation}
    C_2 = \frac{0.45 * (\sigma^2)}{ ((\sigma^2) + 0.09)}
        * (
            \text{step}(\phi_{\text{diff}})       * (\sin(\alpha))
            + (1 - \text{step}(\phi_{\text{diff}}))  * (\sin(\alpha) - \frac{2*\beta}{\pi}^3)
        )
\end{equation}
\begin{equation}
    C_3 = 0.125 * (\sigma^2) / ((\sigma^2)+0.09) * ((4*\alpha*\beta)/(\pi*\pi))^2
\end{equation}

\begin{equation}
    L_1 = \rho/\pi * (C_1 + \phi_{\text{diff}} * C_2 * \tan(\beta) + (1 - \text{abs}(\phi_{\text{diff}})) * C_3 * \tan((\alpha+\beta)/2))
\end{equation}

\begin{equation}
    L_2 = 0.17 * \rho*\rho / \pi * (\sigma^2)/((\sigma^2)+0.13) * (1 - \phi_{\text{diff}}*(4*\beta*\beta)/(\pi*\pi))
\end{equation}

\hspace{60px} The \emph{BRDF} output
\begin{equation}
    f = L_1 + L_2
\end{equation}

\hspace{60px} Utility Functions
\begin{equation}
  \text{normalize}(\vec{u}) = \frac{\vec{u}}{\sqrt{\vec{u} \cdot \vec{u}}}
\end{equation}

\begin{equation}
  \text{abs}(v) = \max(v,-v)
\end{equation}

\hspace{60px} \small{Step function that returns 1 when x \verb">=" 0 and 0 when \verb"x < 0"}
\begin{equation}
  \text{step}(x) = \min(1, \max(0, x / (\text{abs}(x) + \epsilon)))
\end{equation}
\end{lstlisting}
\end{codigo}
%%%%%%%%%%%%%%%%%%%%%%%%%%%%%%%%%%%%%%%%%%%%%%%%%
\subsection{Código GLSL Gerado}
%%%%%%%%%%%%%%%%%%%%%%%%%%%%%%%%%%%%%%%%%%%%%%%%%
\begin{codigo}[H]
    \caption{\small Saída do compilador: código GLSL da BRDF do experimento Oren-Nayar (parte 1 de 3).}
    \label{cod-oren-nayar-glsl-pt-1}
\begin{lstlisting}[language=C, inputencoding=utf8]
analytic ::begin parameters
#[type][name][min val][max val][default val]
::end parameters
::begin shader
//////////// START OF BUILTINS DECLARTION ////////////
vec3 var_0_vec_h;
vec3 var_3_vec_n;
float var_10_theta_h;
float var_11_theta_d;
float var_1_pi;
float var_2_epsilon;
vec3 var_4_vec_omega_i;
float var_5_theta_i;
float var_6_phi_i;
vec3 var_7_vec_omega_o;
float var_8_theta_o;
float var_9_phi_o;
//////////// END OF BUILTINS DECLARTION ////////////
//////////// START OF USER DECLARED ////////////
float var_14_sigma;
vec3 var_19_V;
vec3 var_20_N;
vec3 var_21_L;
float var_22_phi_text_diff;
float var_23_theta_i;
float var_24_theta_r;
float var_25_alpha;
float var_26_beta;
float var_27_C_3;
float var_28_C_1;
float var_29_rho;
float var_30_C_2;
float var_31_L_1;
float var_32_L_2;
float var_33_f;
//////////// END OF USER DECLARED ////////////
//////////// START FUNCTIONS DECLARATIONS ////////////
float var_12_text_abs(float var_13_v) { return max(var_13_v, (-var_13_v)); }
float var_15_text_step(float var_16_x) {
  return min(1.0, max(0.0, (var_16_x / ((var_12_text_abs(var_16_x) + var_2_epsilon)))));
}
vec3 var_17_text_normalize(vec3 var_18_vec_u) {
  return (var_18_vec_u / sqrt(dot(var_18_vec_u, var_18_vec_u)));
}
//////////// END FUNCTIONS DECLARATIONS ////////////
\end{lstlisting}
\end{codigo}

\begin{codigo}[H]
    \caption{\small Saída do compilador: código GLSL da BRDF do experimento Oren-Nayar (parte 2 de 3).}
    \label{cod-oren-nayar-glsl-pt-2}
\begin{lstlisting}[language=C, inputencoding=utf8]
vec3 BRDF(vec3 L, vec3 V, vec3 N, vec3 X, vec3 Y) {
  //////////// START OF BUILTINS INITIALIZATION ////////////
  var_0_vec_h = normalize(L + V);
  var_3_vec_n = normalize(N);
  var_1_pi = 3.141592653589793;
  var_2_epsilon = 1.192092896e-07;
  var_4_vec_omega_i = L;
  var_5_theta_i = atan(var_4_vec_omega_i.y, var_4_vec_omega_i.x);
  var_6_phi_i = atan(sqrt(var_4_vec_omega_i.y * var_4_vec_omega_i.y +
                          var_4_vec_omega_i.x * var_4_vec_omega_i.x),
                     var_4_vec_omega_i.z);
  var_7_vec_omega_o = V;
  var_8_theta_o = atan(var_7_vec_omega_o.y, var_7_vec_omega_o.x);
  var_9_phi_o = atan(sqrt(var_7_vec_omega_o.y * var_7_vec_omega_o.y +
                          var_7_vec_omega_o.x * var_7_vec_omega_o.x),
                     var_7_vec_omega_o.z);
  var_10_theta_h = acos(dot(var_0_vec_h, N));
  var_11_theta_d = acos(dot(var_0_vec_h, var_4_vec_omega_i));
  //////////// END OF BUILTINS INITIALIZATION ////////////
  var_14_sigma = ((30.0 * var_1_pi) / 180.0);
  var_19_V = var_7_vec_omega_o;
  var_20_N = var_3_vec_n;
  var_21_L = var_4_vec_omega_i;
  var_22_phi_text_diff = (dot(var_17_text_normalize(
               (var_19_V - (var_20_N * ((dot(var_19_V, var_20_N)))))),
           var_17_text_normalize(
               (var_21_L - (var_20_N * ((dot(var_21_L, var_20_N))))))));
  var_23_theta_i = acos(((dot(var_21_L, var_20_N))));
  var_24_theta_r = acos(((dot(var_19_V, var_20_N))));
  var_25_alpha = max(var_23_theta_i, var_24_theta_r);
  var_26_beta = min(var_23_theta_i, var_24_theta_r);
  var_27_C_3 = (((0.125 * (pow(var_14_sigma, 2.0))) /
        (((pow(var_14_sigma, 2.0)) + 0.09))) *
       pow((((((4.0 * var_25_alpha) * var_26_beta)) / ((var_1_pi * var_1_pi)))),
           2.0));
  var_28_C_1 = (1.0 - ((0.5 * (pow(var_14_sigma, 2.0))) / (((pow(var_14_sigma, 2.0)) + 0.33))));
  var_29_rho = 0.9;
\end{lstlisting}
\end{codigo}

\begin{codigo}[H]
    \caption{\small Saída do compilador: código GLSL da BRDF do experimento Oren-Nayar (parte 3 de 3).}
    \label{cod-oren-nayar-glsl-pt-3}
\begin{lstlisting}[language=C, inputencoding=utf8]
  var_30_C_2 = (((0.45 * (pow(var_14_sigma, 2.0))) / (((pow(var_14_sigma, 2.0)) + 0.09))) *
       (((var_15_text_step(var_22_phi_text_diff) * (sin((var_25_alpha)))) +
         (((1.0 - var_15_text_step(var_22_phi_text_diff))) *
          ((sin((var_25_alpha)) -
            pow(((2.0 * var_26_beta) / var_1_pi), 3.0)))))));
  var_31_L_1 = ((var_29_rho / var_1_pi) *
       (((var_28_C_1 + ((var_22_phi_text_diff * var_30_C_2) * tan((var_26_beta)))) +
         ((((1.0 - var_12_text_abs(var_22_phi_text_diff))) * var_27_C_3) *
          tan(((((var_25_alpha + var_26_beta)) / 2.0)))))));
  var_32_L_2 = ((((((0.17 * var_29_rho) * var_29_rho) / var_1_pi) *
         (pow(var_14_sigma, 2.0))) /
        (((pow(var_14_sigma, 2.0)) + 0.13))) *
       ((1.0 - ((var_22_phi_text_diff * (((4.0 * var_26_beta) * var_26_beta))) /
                ((var_1_pi * var_1_pi))))));
  var_33_f = (var_31_L_1 + var_32_L_2);
  return vec3(var_33_f);
}

\end{lstlisting}
\end{codigo}


%%%%%%%%%%%%%%%%%%%%%%%%%%%%%%%%%%%%%%%%%%%%%%%%%
\subsection{Visualização do Resultado}
%%%%%%%%%%%%%%%%%%%%%%%%%%%%%%%%%%%%%%%%%%%%%%%%%
\begin{figure}[H]
    \caption{\small{\textit{Plots} da distribuição de reflexão especular e difusa do experimento Oren-Nayar.}}
    \label{fig-oren-nayar-plots}
\minipage{0.48\textwidth}
    \vspace{42px}
  \includegraphics[width=\linewidth]{./Imagens/brdfs/oren-nayar-3D-plot}
    % \vspace{0.1px}
    \legend{ \small (a) 3D \textit{plot}}
\endminipage\hfill
\minipage{0.48\textwidth}
  \includegraphics[width=\linewidth]{./Imagens/brdfs/oren-nayar-polar-plot.png}
    \legend{ \small (b) \textit{Polar plot}}
\endminipage\hfill
\end{figure}

\begin{figure}[H]
    \caption{\small{Objetos 3D renderizados pelo experimento Oren-Nayar.}}
    \label{fig-oren-nayar-eqlang}
\minipage{0.32\textwidth}
  \includegraphics[width=\linewidth]{./Imagens/brdfs/oren-nayar-teapot.png}
    \legend{ \small (a) \textit{Teapot}}
\endminipage\hfill
\minipage{0.32\textwidth}
  \includegraphics[width=\linewidth]{./Imagens/brdfs/oren-nayar-dragon.png}
    \legend{ \small (b) Dragão de Stanford}
\endminipage\hfill
\minipage{0.32\textwidth}%
  \includegraphics[width=\linewidth]{./Imagens/brdfs/oren-nayar-goblin.png}
    \legend{ \small (c) Goblin}
\endminipage
\end{figure}

