\section{Conclusão}

% Slide de Conclusão Geral
\begin{frame}{Conclusão}
    \begin{itemize}
        \item Objetivo alcançado: desenvolvimento de um compilador para traduzir BRDFs para código GLSL.
        \item Facilita a criação de \textit{shaders} a partir de descrições matemáticas em \LaTeX{}, democratizando o acesso a técnicas de computação gráfica avançada.
        \item Integração com ferramentas gratuitas como o visualizador da Disney.
        \item Projeto interdisciplinar envolvendo:
        \begin{itemize}
            \item Gramáticas livres de contexto.
            \item Traversia em Árvores.
            \item Programação com \textit{shaders}.
            \item Fundamentos teóricos de refletância e radiometria.
        \end{itemize}
    \end{itemize}
\end{frame}

% Slide de Principais Contribuições
\begin{frame}{Principais Contribuições}
    \begin{itemize}
        \item Tradução de equações \LaTeX{} em código GLSL para BRDFs:
        \item Visualização das árvores sintáticas geradas (SVG).
        \item Geração de mensagens de erro informativos para facilitar depuração.
        \item Suporte a operações fundamentais em computação gráfica:
        \begin{itemize}
            \item Produto interno, vetorial.
            \item Funções trigonométricas e exponenciação.
        \end{itemize}
    \end{itemize}
\end{frame}

% Slide de Resultados
\begin{frame}{Resultados Obtidos}
    \begin{itemize}
        \item Redução da barreira técnica para implementação de BRDFs.
        \item Sistema funcional e bem-sucedido em experimentos realizados.
        \item Possibilidade de focar na modelagem de refletância sem detalhes de baixo nível.
    \end{itemize}
    % \begin{figure}
    %     \centering
    %     \includegraphics[scale=0.4]{./Imagens/estrutura-geral-compilador.png}
    %     \caption{\small Arquitetura do compilador.}
    % \end{figure}
\end{frame}

% Slide de Perspectivas Futuras
\begin{frame}{Perspectivas Futuras}
    \begin{itemize}
        \item Ampliar suporte a novas construções matemáticas:
        \begin{itemize}
            \item Somatórios ($\Sigma$) e produtos acumulados ($\Pi$).
            \item Vetores de mais dimensões do que apenas 3.
        \end{itemize}
    \item Adicionar derivadas ($\frac{d}{dx}$) e integrais ($\int$) para resolução numérica.
    \item Suporte a outras linguagens de \textit{shading} como a usada por Unity\footnote{HLSL} e Unreal.
    \item Desenvolver um editor \LaTeX{} integrado com visualização simultânea.
    \item Melhorar o tratamento de erros para maior clareza e contextualização.
    \end{itemize}
    % \begin{figure}
    %     \centering
    %     \includegraphics[scale=0.4]{./Imagens/futuro-expansao.png}
    %     \caption{\small Expansão planejada do projeto.}
    % \end{figure}
\end{frame}

\begin{frame}{Encerramento}
    \begin{center}
        \textbf{\Huge Fim}
    \end{center}
    \begin{itemize}
        \item O projeto abre caminho para a democratização de técnicas avançadas em computação gráfica.
        \item\href{mailto:evertonse.junior@gmail.com}{evertonse.junior@gmail.com}.
    \end{itemize}
\end{frame}

