\section{Conclusão}
\begin{frame}{Conclusão}
    Este trabalho alcançou seu objetivo de \textbf{desenvolver um compilador capaz de traduzir BRDFs descritas em \LaTeX{} para código GLSL}. De forma que:

    \begin{itemize}
        \item Reduz a \textbf{barreira técnica} para a criação de \textit{shaders}, permitindo que profissionais de áreas não relacionadas à programação explorem efeitos visuais complexos.
        \item Aumenta a \textbf{interatividade} no design das BRDFs.
        \item Suporta a \textbf{integração} com a ferramenta gratuita da Disney para renderização visual das BRDFs compiladas.
        \item Permite que os usuários \textbf{focalizem na modelagem} de refletância, sem a necessidade de lidar com detalhes de baixo nível.
        \item Demonstra-se capaz por meio de experimentos de renderização bem-sucedidos.
    \end{itemize}
\end{frame}


\begin{frame}{Conclusão: Conhecimentos Usados}
    Utilizamos conhecimentos interdisciplinares para realizar este trabalho, tais como:
    \begin{itemize}
        \item Gramáticas livres de contexto e geração de código.
        \item Algoritmos com árvores e grafos.
        \item Renderização projetiva, \textit{raytracing} e programação com \textit{shaders}.
        \item Fundamentos teóricos de refletância e radiometria.
    \end{itemize}
\end{frame}

% Slide de Principais Contribuições
\begin{frame}{Conclusão: Contribuições}
    \begin{itemize}
        \item Tradução de equações \LaTeX{} com suporte as convenções mais usadas na definição de BRDFs ( $\omega_i,\omega_o, \theta_i, \vec{n}, \pi ...$) :
        \item Visualização das árvores sintáticas geradas (SVG).
        \item Geração de mensagens de erro informativos para facilitar depuração.
        \item Suporte a operações fundamentais em computação gráfica:
        \begin{itemize}
            \item Produto interno, vetorial.
            \item Funções trigonométricas e exponenciação.
        \end{itemize}
    \end{itemize}
\end{frame}

% Slide de Perspectivas Futuras
\begin{frame}{Conclusão: Perspectivas Futuras}
    \begin{itemize}
        \item Ampliar suporte a novas construções matemáticas:
        \begin{itemize}
            \item Somatórios ($\Sigma$) e produtos acumulados ($\Pi$).
            \item Vetores de mais dimensões do que apenas 3.
        \end{itemize}
    \item Adicionar derivadas ($\frac{d}{dx}$) e integrais ($\int$) para resolução numérica.
    \item Suporte a outras linguagens de \textit{shading} como a usada por Unity\footnote{HLSL} e Unreal.
    \item Desenvolver um editor \LaTeX{} integrado com visualização simultânea.
    \item Melhorar o tratamento de erros para maior clareza e contextualização.
    \end{itemize}
    % \begin{figure}
    %     \centering
    %     \includegraphics[scale=0.4]{./Imagens/futuro-expansao.png}
    %     \caption{\small Expansão planejada do projeto.}
    % \end{figure}
\end{frame}

\begin{frame}{Conclusão: Encerramento}
    \begin{center}
        \textbf{\Huge Fim}\footnote{\href{mailto:evertonse.junior@gmail.com}{evertonse.junior@gmail.com}.}
    \end{center}
\end{frame}

