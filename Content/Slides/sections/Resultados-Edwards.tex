
%%%%%%%%%%%%%%%%%%%%%%%%%%%%%%%%%%%%%%%%%%%%%%%%%
% \subsection{Experimento Edwards-2006}
%%%%%%%%%%%%%%%%%%%%%%%%%%%%%%%%%%%%%%%%%%%%%%%%%

\begin{frame}
\begin{figure}
    \frametitle{Experimento Edwards-2006: Equações da BRDF em documento \LaTeX{}}
    \footnote{\tiny{Experimento adicionado aos slides para demonstrar definição de funções no formato $f(x,y) = x+y$.}}
    \begin{center}
        \includegraphics[scale=0.52]{./Imagens/brdfs/edwards-2006.pdf}
    \end{center}
\end{figure}
\end{frame}

%%%%%%%%%%%%%%%%%%%%%%%%%%%%%%%%%%%%%%%%%%%%%%%%%
\begin{frame}[fragile]
    \frametitle{Experimento Edwards-2006: Código fonte \LaTeX{} da BRDF}
%%%%%%%%%%%%%%%%%%%%%%%%%%%%%%%%%%%%%%%%%%%%%%%%%
    \vspace{-0.3cm}
    \hspace{2cm}
\begin{lstlisting}[basicstyle=\ttfamily\scriptsize,language=tex, frame=none, inputencoding=utf8]
\begin{equation}
n = 10
\end{equation}
\begin{equation}
R = 1
\end{equation}

\begin{equation}
\text{lump}(\vec{h}, R, n) = (n+1)/(\pi*R*R) * (1-(\vec{h} \cdot \vec{h})/(R*R)^ n)
\end{equation}

Scaling projection
\begin{equation}
    uH = \vec{\omega_i}+\vec{\omega_o} % // unnormalized H
\end{equation}
\begin{equation}
    h = (\vec{n} \cdot \vec{\omega_o}) / (\vec{n} \cdot uH) * uH
\end{equation}
\begin{equation}
    huv = h - (\vec{n} \cdot \vec{\omega_o}) * \vec{n}
\end{equation}
Specular term (D and G)
\begin{equation}
    p = \text{lump}(huv, R, n)
\end{equation}
\begin{equation}
    f = p * ((\vec{n} \cdot \vec{\omega_o})^ 2)
    / (4 * (\vec{n} \cdot \vec{\omega_i})
    * (\vec{\omega_i} \cdot \vec{h}) * ((\vec{n} \cdot \vec{h})^ 3))
\end{equation}
\end{lstlisting}
\end{frame}

\begin{frame}[fragile]
    \frametitle{Experimento Edwards-2006: Código GLSL da BRDF deste experimento (1 de 2)}
\begin{clang}
analytic ::begin parameters
#[type][name][min val][max val][default val]
::end parameters
::begin shader
//////////// START OF BUILTINS DECLARTION ////////////
vec3 var_0_vec_h;
vec3 var_3_vec_n;
float var_10_theta_h;
float var_11_theta_d;
float var_1_pi;
float var_2_epsilon;
vec3 var_4_vec_omega_i;
float var_5_theta_i;
float var_6_phi_i;
vec3 var_7_vec_omega_o;
float var_8_theta_o;
float var_9_phi_o;
//////////// END OF BUILTINS DECLARTION ////////////
//////////// START OF USER DECLARED ////////////
vec3 var_15_uH;
vec3 var_16_h;
float var_14_n;
vec3 var_17_huv;
float var_13_R;
float var_18_p;
float var_19_f;
//////////// END OF USER DECLARED ////////////
//////////// START FUNCTIONS DECLARATIONS ////////////
float var_12_text_lump(vec3 var_0_vec_h, float var_13_R, float var_14_n) {
  return ((((var_14_n + 1.0)) / (((var_1_pi * var_13_R) * var_13_R))) *
          ((1.0 - ((dot(var_0_vec_h, var_0_vec_h)) /
                   pow(((var_13_R * var_13_R)), var_14_n)))));
}
\end{clang}
\end{frame}

\begin{frame}[fragile]
    \frametitle{Experimento Edwards-2006: Código GLSL da BRDF deste experimento (2 de 2)}
\begin{clang}
//////////// END FUNCTIONS DECLARATIONS ////////////
vec3 BRDF(vec3 L, vec3 V, vec3 N, vec3 X, vec3 Y) {
  //////////// START OF BUILTINS INITIALIZATION ////////////
  var_0_vec_h = normalize(L + V);
  var_3_vec_n = normalize(N);
  var_1_pi = 3.141592653589793;
  var_2_epsilon = 1.192092896e-07;
  var_4_vec_omega_i = L;
  var_5_theta_i = atan(var_4_vec_omega_i.y, var_4_vec_omega_i.x);
  var_6_phi_i = atan(sqrt(var_4_vec_omega_i.y * var_4_vec_omega_i.y +
                          var_4_vec_omega_i.x * var_4_vec_omega_i.x), var_4_vec_omega_i.z);
  var_7_vec_omega_o = V;
  var_8_theta_o = atan(var_7_vec_omega_o.y, var_7_vec_omega_o.x);
  var_9_phi_o = atan(sqrt(var_7_vec_omega_o.y * var_7_vec_omega_o.y +
                          var_7_vec_omega_o.x * var_7_vec_omega_o.x), var_7_vec_omega_o.z);
  var_10_theta_h = acos(dot(var_0_vec_h, N));
  var_11_theta_d = acos(dot(var_0_vec_h, var_4_vec_omega_i));
  //////////// END OF BUILTINS INITIALIZATION ////////////
  var_15_uH = (var_4_vec_omega_i + var_7_vec_omega_o);
  var_16_h = (((dot(var_3_vec_n, var_7_vec_omega_o)) /
    (dot(var_3_vec_n, var_15_uH))) * var_15_uH);
  var_14_n = 10.0;
  var_17_huv = (var_16_h - ((dot(var_3_vec_n, var_7_vec_omega_o)) * var_3_vec_n));
  var_13_R = 1.0;
  var_18_p = var_12_text_lump(var_17_huv, var_13_R, var_14_n);
  var_19_f = ((var_18_p * (pow((dot(var_3_vec_n, var_7_vec_omega_o)), 2.0))) /
              ((((4.0 * (dot(var_3_vec_n, var_4_vec_omega_i))) *
                 (dot(var_4_vec_omega_i, var_0_vec_h))) * (pow((dot(var_3_vec_n, var_0_vec_h)), 3.0)))));
  return vec3(var_19_f);
}
\end{clang}
\end{frame}

\begin{frame}{Experimento Edwards-2006: Plots de Distruição de Reflexão}
\begin{figure}[H]
  
\caption{\small{\textit{Plots} da distribuição de reflexão especular e difusa do experimento edwards-2006.}}
    \label{fig-edwards-2006-plots}
\minipage{0.48\textwidth}
    \vspace{42px}
  \includegraphics[width=\linewidth]{./Imagens/brdfs/edwards-2006-3D-plot}
    % \caption{\small{(a)}}\label{fig:awesome_image1}
    % \vspace{0.1px}
    % \legend{ \small (a) 3D \textit{plot}}
\endminipage\hfill
\minipage{0.48\textwidth}
  \includegraphics[width=\linewidth]{./Imagens/brdfs/edwards-2006-polar-plot.png}
    % \legend{ \small (b) \textit{Polar plot}}
    % \caption{\small{(b)}}\label{fig:awesome_image1}
\endminipage\hfill
\end{figure}
\end{frame}

\begin{frame}{Experimento Edwards-2006: Objetos 3D renderizados para esta BRDF}
\begin{figure}[H]
    \label{fig-edwards-2006-eqlang}
\minipage{0.32\textwidth}
  \includegraphics[width=\linewidth]{./Imagens/brdfs/edwards-2006-teapot.png}
    % \legend{ \small (a) \textit{Teapot}}
\endminipage\hfill
\minipage{0.32\textwidth}
  \includegraphics[width=\linewidth]{./Imagens/brdfs/edwards-2006-dragon.png}
    % \legend{ \small (b) Dragão de Stanford}
\endminipage\hfill
\minipage{0.32\textwidth}%
  \includegraphics[width=\linewidth]{./Imagens/brdfs/edwards-2006-goblin.png}
    % \legend{ \small (c) Goblin}
\endminipage
\end{figure}
\end{frame}

%%%%%%%%%%%%%%%%%%%%%%%%%%%%%%%%%%%%%%%%%%%%%%%%%
% \subsection{Experimento Edwards-2006}
%%%%%%%%%%%%%%%%%%%%%%%%%%%%%%%%%%%%%%%%%%%%%%%%%
