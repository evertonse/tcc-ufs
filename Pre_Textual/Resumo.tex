% resumo em português
\setlength{\absparsep}{18pt} % ajusta o espaçamento dos parágrafos do resumo
\begin{resumo}
 
% Segundo a \citeonline[3.1-3.2]{NBR6028:2003}, o resumo deve ressaltar o objetivo, o método, os resultados e as conclusões do documento. A ordem e a extensão destes itens dependem do tipo de resumo (informativo ou indicativo) e do tratamento que cada item recebe no documento original. O resumo deve ser precedido da referência do documento, com exceção do resumo inserido no próprio documento. (\ldots) As palavras-chave devem figurar logo abaixo do resumo, antecedidas da expressão Palavras-chave:, separadas entre si por ponto e finalizadas também por ponto.


  O presente trabalho propõe o desenvolvimento de um compilador de funções de distribuição de reflexão bidirecional (BRDFs) expressas em \LaTeX{}  para a linguagem de \textit{shading} GLSL, utilizando a técnica de Pratt \textit{Parsing} e linguagem de programação Odin.
  O objetivo é automatizar o processo de tradução de funções complexas de materiais, descritas em equações \LaTeX{}, para o código GLSL utilizado na programação de \textit{shaders} para OpenGL.
  Ao fornecer essa ferramenta, pretende-se não apenas simplificar o trabalho dos desenvolvedores e pesquisadores na área de computação gráfica, mas também democratizar o acesso e compreensão de modelos de materiais complexos. Além disso, ao permitir que as BRDFs sejam expressas em uma forma mais familiar e acessível, como a notação matemática, o compilador reduz a barreira de entrada para aqueles que não estão familiarizados com linguagens programação, de modo a facilitar a colaboração interdisciplinar entre profissionais de diferentes áreas. A validação dos \textit{shaders} de saída do compilador proposto será feita através da ferramenta Disney BRDF Explorer, que possibilita a visualização e análise de BRDFs.

  \textbf{Palavras-chave}: Compilador, BRDFs, LaTeX, GLSL, Shading, Pratt \textit{Parsing}.
\end{resumo}
