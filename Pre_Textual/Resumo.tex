% resumo em português
\setlength{\absparsep}{18pt} % ajusta o espaçamento dos parágrafos do resumo
\begin{resumo}
 
% Segundo a \citeonline[3.1-3.2]{NBR6028:2003}, o resumo deve ressaltar o objetivo, o método, os resultados e as conclusões do documento. A ordem e a extensão destes itens dependem do tipo de resumo (informativo ou indicativo) e do tratamento que cada item recebe no documento original. O resumo deve ser precedido da referência do documento, com exceção do resumo inserido no próprio documento. (\ldots) As palavras-chave devem figurar logo abaixo do resumo, antecedidas da expressão Palavras-chave:, separadas entre si por ponto e finalizadas também por ponto.

% O presente trabalho propõe o desenvolvimento de um compilador de funções de distribuição de reflexão bidirecional (BRDFs) expressas em LaTeX para a linguagem de shading GLSL, utilizando a técnica de parsing de Pratt. O objetivo é automatizar o processo de tradução de funções complexas de materiais, frequentemente descritas em LaTeX, para o código GLSL utilizado em programação de shaders para OpenGL. Para isso, será empregada a técnica de parsing de Pratt, uma abordagem eficiente e flexível para analisar e traduzir expressões matemáticas e lógicas. O trabalho incluirá a implementação do compilador, a análise de desempenho e precisão da tradução, e a comparação com métodos tradicionais de tradução manual. Ao final, espera-se oferecer uma ferramenta útil para desenvolvedores e pesquisadores na área de computação gráfica, facilitando a utilização e compreensão de modelos de materiais complexos em aplicações gráficas. Palavras-chave: Compilador, BRDFs, LaTeX, GLSL, Pratt Parsing.

O presente trabalho propõe o desenvolvimento de um compilador de funções de distribuição de reflexão bidirecional (BRDFs) expressas em LaTeX para a linguagem de shading GLSL, utilizando a técnica de parsing de Pratt. O objetivo é automatizar o processo de tradução de funções complexas de materiais, frequentemente descritas em LaTeX, para o código GLSL utilizado em programação de shaders para OpenGL. Ao fornecer essa ferramenta, pretende-se não apenas simplificar o trabalho dos desenvolvedores e pesquisadores na área de computação gráfica, mas também democratizar o acesso e compreensão de modelos de materiais complexos. Além disso, ao permitir que as BRDFs sejam expressas em uma forma mais familiar e acessível, como a notação matemática, o compilador reduz a barreira de entrada para aqueles que não estão familiarizados com linguagens programação. Isso pode facilitar a colaboração interdisciplinar entre profissionais de diferentes áreas, como artistas visuais, designers e cientistas de materiais, que desejam explorar e entender o comportamento visual de materiais em suas aplicações.

 \textbf{Palavras-chave}: Compilador, BRDFs, LaTeX, GLSL, Shading, Pratt Parsing.
\end{resumo}
